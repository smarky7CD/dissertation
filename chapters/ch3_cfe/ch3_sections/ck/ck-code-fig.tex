\begin{figure}[htp]
	\Wider[3em]{
		\centering
		\begin{pcvstack}[boxed,center,space=0.5em]
			\begin{pchstack}
				\begin{pcvstack}[space=0.45em]
					\procedure[linenumbering, headlinecmd={\vspace{.1em}\hrule\vspace{.2em}}]{$\Rep_K(\setS)$}{%
						M \gets \zeros(k,m)\\
%						\pcgraycomment{$k\times m$ (fp,cnt) 2-d array}\\
						\pcfor i \in [k] \\
						\t A[i] \gets [(\star,0)]\times m\\
						\repr \gets \langle M,A\rangle\\
						\pcfor x \in \setS \\
						\t \repr \, {\getsr} \Up_K(\repr,{\up_{x}})\\
						\pcreturn \repr
					}
					\procedure[linenumbering, headlinecmd={\vspace{.1em}\hrule\vspace{.2em}}]{$\Up_K(\repr,\up_x)$}{%
						\langle M,A\rangle \gets \repr\\
						M  \getsr \Up^\CMS_{K}(M ,\up_x)\\
						A \getsr \Up^\HK_{K}(A ,\up_x)\\
						\pcreturn \repr {\gets}  \langle M,A\rangle
					}
				\end{pcvstack}
				\begin{pcvstack}[space=0.45em]
					\procedure[linenumbering, headlinecmd={\vspace{.1em}\hrule\vspace{.2em}}]{$\Qry_K(\repr,\qry_x)$}{%
						\langle M, A \rangle \gets \repr\\
						(p_1,\ldots,p_k) \gets R(K,x), \, \mathrm{fp}_{x} \gets T(K,x)\\
						\Theta_1,\Theta_2 \gets \infty\\
						\pcgraycomment{CMS only overestimates}\\
						\cnt_{\text{UB},x}\gets \Qry^\CMS_{K}(M,\qry_x)\label{line:ck:cms}\\
						\pcgraycomment{HK only underestimates}\\
						\cnt_{\text{LB},x} \gets \Qry^\HK_{K}(A,\qry_x) \label{line:ck:finalunderest}\\
						\pcgraycomment{return upperbound if equal to lowerbound}\\
						\pcif \cnt_{\text{UB},x} =  \cnt_{\text{LB},x}\label{line:UB=LB:start}\\
						\t \pcreturn \cnt_{\text{UB},x} \label{line:UB=LB:finish}\\
						\pcfor i \in [k] \label{line:ck:startoverestdjust}\\
						\t \pcgraycomment{if never observed}\\
						\t \pcif A[i][p_i].\mathrm{fp} = \star\\
						\t \t \cnt_{\text{UB},x} \gets  0\\
						\t \t \pcreturn 0 \label{line:ck:est0}\\
						\t \pcgraycomment{upper bound adjustment}\\
						\t \pcgraycomment{x does not own counter}\\
						\t \pcelse \pcif A[i][p_i].\mathrm{fp} \not= \fp_x\\
						\t \t \Theta \gets \frac{M[i][p_i] {-} A[i][p_i].\cnt {+}1}{2}\\
						\t \t \Theta_1 {\gets} {\min}\left\{ \Theta_1, \Theta \right\}\\
%						\t \t \Theta_1 {\gets} 
%						{\min}\left\{ 
%						\Theta_1, \frac{M[i][p_i] {-} A[i][p_i].\cnt {+}1}{2}
%						\right\}\\
						\t \pcgraycomment{x owns counter}\\
						\t \pcelse \pcif A[i][p_i].\mathrm{fp} = \fp_x\\
						\t \t \Theta \gets \frac{M[i][p_i] {+} A[i][p_i].\cnt}{2}\\
						\t \t \Theta_2 {\gets} 
						{\min}\left\{ 
						\Theta_2, \Theta\right\}\\
%                         \t \t \Theta_2 {\gets} 
%						{\min}\left\{ 
%						\Theta_2, \frac{M[i][p_i] {+} A[i][p_i].\cnt}{2}
%						\right\}\\
						\cnt_{\text{UB},x} {\gets} \floor(\min\left\{ \Theta_1, \Theta_2 \right\}) \label{line:ck:finaloverest}\\
						\pcreturn \cnt_{\text{UB},x}
					}
				\end{pcvstack}
			\end{pchstack}	
		\end{pcvstack}
	}
	\caption{Keyed structure $\CK[R,T,m,k]$ supporting point-queries for any potential stream element~$x$ ($\qry_x$).
		$\Qry^\CMS_{K}, \Up^\CMS_{K}$, resp. $\Qry^\HK_{K},  \Up^\HK_{K}$, denote query and update algorithms of keyed structure $\CMS[R,T,m,k]$ (Figure \ref{fig:cms}), resp. $\HK[R,T,m,k,1]$ (Figure \ref{fig:hk}, but note $d=1$). 
		The parameters are a function $R: \keys\by\bits^* \to [m]^k$, a function $T: \keys\by\bits^* \to \bits^n$ for some desired fingerprint length~$n$, and integers $m,k \geq 0$. A concrete scheme is given by a particular choice of parameters.}
	\label{fig:ck}
\end{figure}
%\eject


%%%%%%%%%%%%%%%%%%%%%%%%%%%%%%%%%%%%%%%%%
%%%%%%%%%%%%%%%%%%%%%%%%%%%%%%%%%%%%%%%%%
\ignore{
\begin{center}
\mpage{1}{
	\Wider[5em]{
		\centering
		\begin{pcvstack}[boxed,space=0.5em]
			\begin{pchstack}
				\begin{pcvstack}
					\procedure[linenumbering, headlinecmd={\vspace{.1em}\hrule\vspace{.2em}}]{$\Rep_K(\setS)$}{%
						M \gets \zeros(k,m)\\
						\pcgraycomment{initialise $k\times m$ (fp,cnt) 2-d array}\\
						\pcfor i \in [k] \\
						\t A[i] \gets [(\star,0)]\times m\\
						\repr \gets \langle M,A\rangle\\
						\pcfor x \in \setS \\
						\t \repr \getsr \Up_K(\repr,\up_{x})\\
						\pcreturn \repr
					}
					\procedure[linenumbering, headlinecmd={\vspace{.1em}\hrule\vspace{.2em}}]{$\Up_K(\repr,\up_x)$}{%
						\langle M,A\rangle \gets \repr\\
						M  \getsr \Up_{K,\text{CMS}}(M ,\up_x)\\
						A \getsr \Up_{K,\text{HK}}(A ,\up_x)\\
						\pcreturn \repr \gets \langle M,A\rangle
					}
				\end{pcvstack}
				\begin{pcvstack}
					\procedure[linenumbering, headlinecmd={\vspace{.1em}\hrule\vspace{.2em}}]{$\Qry_K(\repr,\qry_x)$}{%
						\langle M, A \rangle \gets \repr\\
						(p_1,\ldots,p_k) \gets R(K,x), \, \mathrm{fp}_{x} \gets T(K,x)\\
						\Theta_1,\Theta_2 \gets \infty\\
						\pcgraycomment{CMS only overestimates}\\
						\cnt_{\text{UB},x}\gets \Qry_{K,\text{CMS}}(M,\qry_x)\label{line:ck:cms}\\
						\pcgraycomment{HK only underestimates}\\
						\cnt_{\text{LB},x} \gets \Qry_{K,\text{HK}}(A,\qry_x) \label{line:ck:finalunderest}\\
						\pcgraycomment{return upperbound if equal to lowerbound}\\
						\pcif \cnt_{\text{UB},x} =  \cnt_{\text{LB},x}\label{line:UB=LB:start}\\
						\t \pcreturn \cnt_{\text{UB},x} \label{line:UB=LB:finish}\\
						\pcfor i \in [k] \label{line:ck:startoverestdjust}\\
						\pcgraycomment{if never observed}\\
						\t \pcif A[i][p_i].\mathrm{fp} = \star:\\
%						line added to make presentation simpler?
						\t \t \cnt_{\text{UB},x} \gets  0\\
%						\cnt_{\text{LB},x} = 0\\
						\t \t \pcreturn 0 \label{line:ck:est0}\\
						\pcgraycomment{upper bound adjustment}\\
						\pcgraycomment{x does not own counter}\\
						\t \pcelse \pcif A[i][p_i].\mathrm{fp} \not= \fp_x:\\
%						\t \t \fp' \gets A[i][p_i].\mathrm{fp}\\
%						\t \t \cnt'_{\text{LB}}\gets
%						\underset{A[j][\ell].\fp = \fp'}{\underset{j \in [k], \ell \in [m]}{\max}}
%						\{A[j][\ell].\cnt\}\\
%						\t \t \Theta_0 \gets 
%						\min\left\{ 
%						\Theta_0, M[i][p_i] - \cnt'_{\text{LB}}
%						\right\}\\
						\t \t \Theta_1 \gets 
						\min\left\{ 
						\Theta_1, \frac{M[i][p_i] - A[i][p_i].\cnt +1}{2}
						\right\}\\
						\pcgraycomment{x owns counter}\\
						\t \pcelse \pcif A[i][p_i].\mathrm{fp} = \fp_x:\\
                                                \t \t \Theta_2 \gets 
						\min\left\{ 
						\Theta_2, \frac{M[i][p_i] + A[i][p_i].\cnt}{2}
						\right\}\\
						\cnt_{\text{UB},x} \gets \floor(\min\left\{ \Theta_1, \Theta_2 \right\}) \label{line:ck:finaloverest}\\
						\pcreturn \cnt_{\text{UB},x}
%						\left\lceil\frac{\cnt_{\text{UB},x} +\cnt_{\text{LB},x}}{2} \right\rceil  \label{line:ck:qryout}
					}
					\vspace{1cm}
				\end{pcvstack}
			\end{pchstack}	
		\end{pcvstack}
	}    
}
\captionof{figure}{Keyed structure $CK[R,T,m,k]$ supporting point-queries for any potential stream element~$x$ ($\qry_x$).
$\Qry_{K,\text{CMS}}, \Up_{K,\text{CMS}}$, resp. $\Qry_{K,\text{HK}},  \Up_{K,\text{HK}}$, denote query and update algorithms of keyed structure $CMS[R,T,m,k]$ (Figure \ref{fig:cms}), resp. $HK[R,T,m,k,1]$ (Figure \ref{fig:hk}). 
The parameters are a function $R: \keys\by\bits^* \to [m]^k$, a function $T: \keys\by\bits^* \to \bits^n$ for some desired fingerprint length~$n$, and integers $m,k \geq 0$. A concrete scheme is given by a particular choice of parameters.}

\end{center}
}