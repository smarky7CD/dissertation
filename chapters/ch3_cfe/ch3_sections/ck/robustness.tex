%
\begin{figure*}[ht!]
	\Wider[3em]{
		\centering
		\begin{pcvstack}[boxed,center,space=0.5em]
			\begin{pchstack}
				\begin{pcvstack}[space=0.45em]
					\procedure[linenumbering, headlinecmd={\vspace{.1em}\hrule\vspace{.2em}}]{$\Rep_K(\setS)$}{%
						M \gets \zeros(k,m)\\
%						\pcgraycomment{$k\times m$ (fp,cnt) 2-d array}\\
						\pcfor i \in [k] \\
						\t A[i] \gets [(\star,0)]\times m\\
						\repr \gets \langle M,A\rangle\\
						\pcfor x \in \setS \\
						\t \repr \, {\getsr} \Up_K(\repr,{\up_{x}})\\
						\pcreturn \repr
					}
					\procedure[linenumbering, headlinecmd={\vspace{.1em}\hrule\vspace{.2em}}]{$\Up_K(\repr,\up_x)$}{%
						\langle M,A\rangle \gets \repr\\
						M  \getsr \Up^\CMS_{K}(M ,\up_x)\\
						A \getsr \Up^\HK_{K}(A ,\up_x)\\
						\pcreturn \repr {\gets}  \langle M,A\rangle
					}
				\end{pcvstack}
				\begin{pcvstack}[space=0.45em]
					\procedure[linenumbering, headlinecmd={\vspace{.1em}\hrule\vspace{.2em}}]{$\Qry_K(\repr,\qry_x)$}{%
						\langle M, A \rangle \gets \repr\\
						(p_1,\ldots,p_k) \gets R(K,x), \, \mathrm{fp}_{x} \gets T(K,x)\\
						\Theta_1,\Theta_2,\Delta \gets \infty\\
                        \text{flag} \gets \mathsf{False}\\
                        N \gets \sum_{j=1}^{m} M[1][j]\\
						%\pcgraycomment{CMS only overestimates}\\
						\cnt_{\text{UB},x}\gets \Qry\CMS_{K}(M,\qry_x)\\
						%\pcgraycomment{HK only underestimates}\\
						\cnt_{\text{LB},x} \gets \Qry^\HK_{K}(A,\qry_x)\\
						%\pcgraycomment{return upperbound if equal to lowerbound}\\
						\pcif \cnt_{\text{UB},x} =  \cnt_{\text{LB},x}\\
						\t \pcreturn \cnt_{\text{UB},x},\text{flag}\\
						\pcfor i \in [k] \\
						%\t \pcgraycomment{if never observed}\\
						\t \pcif A[i][p_i].\mathrm{fp} = \star\\
						\t \t \cnt_{\text{UB},x} \gets  0\\
						\t \t \pcreturn 0,\text{flag}\\
						%\t \pcgraycomment{upper bound adjustment}\\
						%\t \pcgraycomment{x does not own counter}\\
						\t \pcelse \pcif A[i][p_i].\mathrm{fp} \not= \fp_x\\
						\t \t \Theta \gets \frac{M[i][p_i] {-} A[i][p_i].\cnt {+}1}{2}\\
						\t \t \Theta_1 {\gets} {\min}\left\{ \Theta_1, \Theta \right\}\\
                        \t \t \hat{\Delta} \gets \frac{M[i][p_i] {-} A[i][p_i].\cnt {+} 1}{2}\\
                        \t \t \Delta {\gets} 
						{\min}\left\{ 
						\Delta, \hat{\Delta}\right\}\\
%						\t \t \Theta_1 {\gets} 
%						{\min}\left\{ 
%						\Theta_1, \frac{M[i][p_i] {-} A[i][p_i].\cnt {+}1}{2}
%						\right\}\\
						%\t \pcgraycomment{x owns counter}\\
						\t \pcelse \pcif A[i][p_i].\mathrm{fp} = \fp_x\\
						\t \t \Theta \gets \frac{M[i][p_i] {+} A[i][p_i].\cnt}{2}\\
						\t \t \Theta_2 {\gets} 
						{\min}\left\{ 
						\Theta_2, \Theta\right\}\\
                        \t \t \hat{\Delta} \gets \frac{M[i][p_i] {-} A[i][p_i].\cnt}{2}\\
                        \t \t \Delta {\gets} 
						{\min}\left\{ 
						\Delta, \hat{\Delta}\right\}\\
%                        \t \t \Theta_2 {\gets} 
%						{\min}\left\{ 
%						\Theta_2, \frac{M[i][p_i] {+} A[i][p_i].\cnt}{2}
%						\right\}\\
						\cnt_{\text{UB},x} {\gets} \floor(\min\left\{ \Theta_1, \Theta_2 \right\}) \\
                        \pcif \Delta \geq \psi N\\
                        \t \text{flag} \gets \mathsf{True}\\
						\pcreturn \cnt_{\text{UB},x}, \text{flag}
					}
				\end{pcvstack}
			\end{pchstack}	
		\end{pcvstack}
	}
	\caption{Keyed structure $\CK[R,T,m,k,\psi]$ supporting point-queries for any potential stream element~$x$ ($\qry_x$) and the ability to raise a flag on ``bad'' frequency estimation.
		$\Qry^\CMS_{K}, \Up^\CMS_{K}$, resp. $\Qry^\HK_{K},  \Up^\HK_{K}$, denote query and update algorithms of keyed structure $\CMS[R,T,m,k]$ (Figure \ref{fig:cms}), resp. $\HK[R,T,m,k,1]$ (Figure \ref{fig:hk}). 
		The parameters are a function $R: \keys\by\bits^* \to [m]^k$, a function $T: \keys\by\bits^* \to \bits^n$ for some desired fingerprint length~$n$, integers $m,k \geq 0$, and flag parameter~$\psi \in \left( 0,1 \right)$. A concrete scheme is given by a particular choice of parameters.}
	\label{fig:flag-ck}
\end{figure*}

Corollary~\ref{cor:esterror:CMSHK} shows that the error in $\CK(x)$ is largest when $\HK(x) \ll \CMS(x)$.  In particular, when~$x$ does not own any of its counters $\HK(x)$ takes on its minimal value of zero.  But we can say something a bit more refined, by examining what is computed on the way to the returned value $\CK(x)$.  

Specifically, recall that $\CK(x) = \lfloor\min\{\Theta_1,\Theta_2\}\rfloor$, where $\Theta_1$ is the smallest upperbound on~$n_x$ that we can determine by looking only at the rows that~$x$ does not own, and $\Theta_2$ is the smallest upperbound on~$n_x$ that we can determine by looking only at the rows that~$x$ does own.  Let $\Delta=|\CK(x)-n_x|$ be the potential error in the estimate $\CK(x)$.  Dropping the floor for brevity, if $\CK(x)=\Theta_1$ then Lemma~\ref{lma:fx:MA:Theta2} tells us that $\Delta \leq (M[i^*][p_{i^*}] - A[i^*][p_{i^*}].\cnt+1)/2$, where $i^* \in \{j \mid \Theta^j_1 = \min_{i \in \hat{I}_x}\{\Theta^i_1\}\}$. 

Likewise, if $\CK(x)=\Theta_2$ then by Lemma~\ref{lma:fx:MA:Theta2} we have~$n_x \leq (M[i^*][p_{i^*}] + A[i^*][p_{i^*}].\cnt)/2$, where now $i^* \in \{j \mid \Theta^j_2=\min_{i \in I_x}\{\Theta^i_2\} \}$.  In this case $A[i^*][p_{i^*}].\cnt \leq n_x$, so we know that $\Delta \leq (M[i^*][p_{i^*}] + A[i^*][p_{i^*}].\cnt)/2 - A[i^*][p_{i^*}].\cnt= (M[i^*][p_{i^*}] - \allowbreak A[i^*][p_{i^*}].\cnt)/2$.  Adding 1/2 to this upperbound gives the same expression as in the previous case.

Thus, we can augment the basic version of CK so that $\Qry(\qry_x)$ 
%~$\Qry_K(\pub,\qry_x)$ 
computes~$\Delta$, and returns a boolean value~$\mathrm{flag}$ along with the estimate of~$n_x$.  The value of~$\mathrm{flag}$ would be set to 1 iff $\Delta \geq \psi N$, where~$N$ is the length of currently inserted stream and $\psi$ is a parameter.  We choose this condition because the non-adaptive correctness guarantees of CMS have a similar form: with $k$~rows and $m$~counters per row, the estimate $\CMS(x)$ is such that $\Prob{\CMS(x) - n_x \leq \epsilon N} \geq 1-\delta$ when $\epsilon = e/m$, $\delta=e^{-k}$. 

Observe that when the frequency estimation error on an element~$x$ is large, then row~$i^*$ will be such that~$M[i^*][p_{i^*}]$ will have a large value and~$A[i^*][p_{i^*}].\cnt$ will have a value very small relative to the value in~$M[i^*][p_{i^*}]$. In the worst case~$A[i][p_{i^*}].\cnt=1$ -- in our attacks we force this to be the case. Taking~$A[i^*][p_{i^*}].\cnt \approx 0$, observe that whether $\CK(x)$ is determined by~$\Theta_1$ or~$\Theta_2$, we see~$\CK(x) \approx (1/2)M[i^*][p_{i^*}] \approx (1/2)\CMS(x)$ in this high error case. Then rolling in the non-adaptive CMS correctness guarantee we see $\Pr[\Delta > (1/2)(\epsilon N) - (1/2)n_x] \leq \delta$ and certainly $\Pr[\Delta > 1/2(\epsilon)N] \leq \Pr[\Delta > 1/2(\epsilon)N - (1/2)n_x]$, thus setting~$\psi = (1/2)\epsilon$ (where we can derive $\epsilon$ from parameter~$m$) can be a useful starting point for setting~$\psi$. As a caveat, however, as~$N$ becomes large, an adversarial stream may be able to induce significant error by setting~$\psi$ in this way (due to the looseness of the CMS bound).  Depending on the deployment scenario, smaller values of~$\psi$, or even sublinear functions of~$N$, may be more appropriate for detecting abnormal streams.

Nonetheless, we implemented an version of CK with flag-raising (see Figure~\ref{fig:flag-ck}), and set~$m=1024,k=4$. This corresponds to~$\epsilon=0.00265,\delta=0.0183$. We then set~$\psi = 0.0012 < \frac{1}{2} \epsilon$. Against it, we ran~$100$ trials of the public hash, public representation attack with~$q_U = 2^{16}$, and with per-trial random target elements~$x$. The average error was~$8203.71$, and in \emph{every} trial the warning flag was raised on the frequency estimation of the target element.

For comparison, we also ran 100 trials, with the same parameters, using the non-adversarial streams from Section~\ref{sub-sec:experiments}. In each trial, the entire stream was processed, and then we queried for the frequency of \emph{every} element in the stream, counting the number of estimates that raised the flag.  Over all 100 trials, or nearly 7.7 million estimates in total, only \emph{three} flags were raised.  These initial findings suggest that the potential for CK to flag suspicious estimates may be of significant benefit to systems employing compact frequency estimators.

%When a flag is set, signals that the estimate has potentially high error and that adversarial behavior may be occurring. Therefore, systems that rely on estimations from CK can use the flag as a signal to regard any action they take based on the estimate with caution, even adjusting what actions are taken based on if a flag is raised. We present a version of the CK that supports this flag raising behavior in 

%%%---start of ignore block---%%%
\ignore{
As shown in Section \smnote{a pointer to the CK attacks section---add in when paper structure is more finalized} when comparing attacks between the structures, the CK allows for less error than the CMS and HK for a fixed sized structure and set of adversarial resources. Further recall, that error is maximized in the attack on some target~$x$ when all other the counters that~$x$ maps to have large value in the~$M$ substructure of the CK and low value~$\approx 1$ in the~$A$ substructure of the CK. In our attacks it is such that~$x$ will not own any of the counters in~$A$, but if the above condition is met then the error in the frequency estimation on~$x$ will be large regardless. 

This condition provides an additional layer of adversarial robustness available to the CK that is not available to the other structures. Precisely, we can raise a \emph{flag} on returns from the~$\Qry$ algorithm when we deem a particular estimation to be unreliable. Intuitively, we can be suspicious of a frequency estimate of an element~$x$ when all the counters it maps to have the large~$M$,small~$A$ pattern described above. We can however, make this idea more precise by digging in to the internals of the~$\Qry$ algorithm of the CK and our analysis of the structure. We will show this flag is raised without fail when adaptive adversarial input is being inserted into the CK such that intolerable error is being induced on some target~$x$, while being suppressed during non-adversarial operation. 

Further, observe that we are not able to raise suspicion on a call to~$\QRYO$ on a particular element~$x$ for the CMS or HK, because unlike the CK there is no information pattern contained in the representation to exploit. The CK due to its two substructures allow us to elucidate information about the set of elements that collide with a particular element in each row, and in turn allow us to determine how these colliding elements are effecting our estimate. \smnote{Does this need to be expanded upon?}. 

We present a small update to the standard CK in Figure~\ref{fig:flag-ck} that adds the ability to raise a flag when the potential overestimation error when~$\Qry$ is called on a particular element~$x$ exceeds a user defined threshold. We set this threshold by providing an additional parameter~$\psi \in \left( 0,1 \right)$ to the structure. We also change the~$\Qry$ algorithm to return a pair~$(\hat{n}_x,\text{flag})$ as opposed to solely the frequency estimation~$\hat{n}_x$ for an element~$x$. The return value flag takes on a boolean value that is set to true if~$\Delta \geq \psi N$ and false otherwise, where~$\Delta \leq |\hat{n_x} - n_x|$ is the potential overestimation amount on any frequency estimate, and~$N$ is the total length of the stream processed thus far. 

When we make an estimate on an element~$x$ using CK we first compute its row positions $(p_1,\ldots,p_k) \gets R(K,x)$  and its fingerprint $\text{fp}_x \gets T(K,x)$. We then go row by row and make our estimation based on our internal sub-estimators $\Theta_1,\Theta_2$. Specifically, we return $\hat{n_x} \gets \floor(\min\left\{ \Theta_1, \Theta_2 \right\})$, assuming that $\cnt_{\text{UB},x} \neq \cnt_{\text{LB},x}$ or for any~$i \in k$ it is such that~$A[i][p_i].\fp \neq \star$ -- in these cases a perfect estimation is made and we of course do not raise a flag. Call the row we select the sub-estimator to make the final estimation from~$i*$. Recall, that this is the row we minimize the amount of unknown frequency contribution in. 

In the case~$\Theta_1$ is the sub-estimator we base our final point query estimate upon we observe that~$\Delta 
\leq (M[i*][p_i*] - A[i*][p_i*].\cnt+1)/2$ as if~$x$ does not own the counter~$A[i*][p_i*]$ with its fingerprint then~$n_x \geq 0$ and by Lemma~\ref{lma:fx:MA:Theta1} we have~$n_x \leq (M[i*][p_i*] - A[i*][p_i*].\cnt+1)/2$ capping our potential estimate error. In the other case in which~$\Theta_2$ is the sub-estimator we base our final point query estimate upon we observe that we observe that~$\Delta \leq (M[i*][p_i*] - A[i*][p_i*].\cnt)/2$ as if~$x$ does own the counter~$A[i*][p_i*]$ with its fingerprint then~$n_x \geq A[i*][p_i*]$ and by Lemma~\ref{lma:fx:MA:Theta2} we have~$n_x \leq (M[i*][p_i*] + A[i*][p_i*].\cnt)/2$. Subtracting this upper-bound from the lower-bound caps our potential estimate error.

A CMS provides the guarantee in the non-adaptive setting that on a frequency estimation for any element~$x$ that~$\hat{n}_x \leq n_x + \epsilon N$ with probability~$1 - \delta$ where~$m = \left \lfloor\frac{e}{\epsilon}\right \rfloor$ (where~$e$ is Euler's number) and~$k = \left \lfloor \mathrm{ln}(\frac{1}{\delta}) \right \rfloor$. If we fix~$m,k$ first we can of course derive~$\epsilon,\delta$ by solving for them according to the previous equations. }
%%%---end of ignore block---%%%