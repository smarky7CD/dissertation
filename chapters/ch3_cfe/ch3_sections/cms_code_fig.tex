\begin{figure}[h]
    %	\Wider[4em]{
            \centering
            \begin{pchstack}[boxed,center,space=0.5em]
                \begin{pcvstack}[space=0.45em]
                        \procedure[linenumbering, headlinecmd={\vspace{.1em}\hrule\vspace{.2em}}]{$\Rep_{K}(\setS)$}{%
                            M \gets \zeros(k,m)\\
                            \pcfor x \in \setS \\
                            \t M \gets \Up_{K}(M,\up_{x})\\
                            \pcreturn M
                        }
                \end{pcvstack}	
                \begin{pcvstack}[space=0.45em]
                        \procedure[linenumbering, headlinecmd={\vspace{.1em}\hrule\vspace{.2em}}]{$\Up_{K}(M,\up_x)$}{%
                            (p_1,\ldots,p_k) \gets R(K,x)\\
                            \pcfor i \in [k]\\
                            \t M[i][p_i] \,{+}{=}\,1\\
                            \pcreturn M 
                        }
                        \procedure[linenumbering, headlinecmd={\vspace{.1em}\hrule\vspace{.2em}}]{$\Qry_{K}(M,\qry_x)$}{%
                            (p_1,\ldots,p_k) \gets R(K,x)\\
                            \pcreturn  \min_{i \in [k]}\{M[i][p_i]\}
                        }
                \end{pcvstack}	
            \end{pchstack}
    %	}
      \caption[The Count-min Sketch Structure.]{Keyed count-min sketch structure $\mathrm{CMS}[R,m,k]$ admitting point queries for any~$x \in \univ$. The parameters are integers $m,k \geq 0$, and a keyed function $R: \keys\by\univ \to [m]^k$ that maps data-object elements (encoded as strings) to a vector of positions in the array~$\v.M$. A concrete scheme is given by a particular choice of
      parameters.
      } 
      \label{fig:cms}
    \end{figure}
    
    