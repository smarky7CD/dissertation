Say $\hat{n}_x$ is the CMS estimate.
As noted in Section \ref{sec:cms}, the estimate $\hat{n}_x = n_x + \min_{i \in [k]}\{\sum_{y \in V^i_x }n_y\}$; thus $\hat{n}_x = n_x$ if there exists an $i\in[k]$ such that $\sum_{y \in V^i_x }n_y=0$.  Since $n_y > 0$ for any $y \in V^{i}_x$, we
can restate this as $\hat{n}_x > n_x$ if and only if $V^{1}_x,\ldots,V^{k}_x$ are all non-empty.  When this is the case, 
the union $\set{C} = \bigcup_{i\in[k]} V^{i}_x$ contains a set of stream elements that ``cover'' the counters $M[i][p_i]$ associated to~$x$. 
Since the presence of a covering~$\set{C}$ within the stream is necessary (and sufficient)
for creating a frequency estimation error for the CMS, we formalize the idea of a ``cover'' in the following definition.

\begin{definition} Let~$\set{U}$ be the universe of possible stream elements.  Fix $x \,{\in}\, \set{U}$, $r\,{\in}\,\mathbb{Z}$, and $\set{Y}\,{\subseteq}\,U$. Then a set~$\set{C}\,{=}\,\{y_1,y_2,{\dots},y_t\}$ is an $(\set{Y},x,r)$-\emph{cover} if: (1) $\set{C}\subseteq \set{Y}{\setminus}\{x\}$, and (2)$\forall i \,{\in}\, [k]\,$ $\exists j_1,\ldots,j_r\,{\in}\,[t]$ such that $R(K,x)[i]\,{=}\,R(K,y_{j_1})[i],{\ldots},R(K,x)[i]\,{=}\,R(K,y_{j_r})[i]$.
	\myenddef
\end{definition}

For the CMS, we will be interested in $\set{Y}\,{=}\,\set{U}, r\,{=}\,1$, and we will shorten the notation to calling this a 1-cover (for~$x$), or just a cover. 
For the HK, we will still be interested in $r\,{=}\,1$, but with a different set~$\set{Y}$.   In particular, HK has a fingerprint function $T(K,\cdot)$, and we define the set $\set{FP}(K,x)\,{=}\,\{y \in \set{U} \,|\, T(K,y) \,{\not=}\, T(K,x)\}$. We will typically write $\fp_x$ as shorthand for the result of computing $T(K,x)$, dropping explicit reference to the key~$K$;   

In analyzing their HK structure, Yang et al.~\cite{yang2019heavykeeper}, rely on there being ``no fingerprint collisions", to ensure that HK have only one-sided error. (In general, the HK returned estimates may over- or underestimate the true frequency.)  But, no precise definition of this term is given.  We define it (by negation) as follows: stream~$\streamvar{S}$ does not satisfy the \emph{no-fingerprint collision} (NFC) condition with respect to~$x$ (and key~$K$) if there exists $y,z \in \streamvar{S}\|x$ such that~$T(K,y)=T(K,z)$ \emph{and}~$\exists i$ such that~$R(K,y)[i] = R(K,z)[i]$; otherwise $\streamvar{S}$ does satisfy the NFC condition with respect to~$x$ (and~$K$). In other words,~$\streamvar{S}\|x$ cannot contain distinct elements that have the same fingerprint and share a counter position.  Our analysis treats the fingerprint function~$T(K,\cdot)$ and position hash functions~$R(K,\cdot)[i]$ as random oracles, the particular value of~$K$ will not matter, only whether or not it is publicly known.  As such, explicit mention of~$K$ can be elided without loss of generality, and we shorten $\set{FP}(K,x)$ to $\set{FP}_x$.  Further, in the random oracle model the fingerprint computation and row position computation are independent, so the probability of their conjunction is much smaller than the simple ``birthday bound'' event on fingerprint collisions. Anyway, for our HK analysis (Section~\ref{sec:hk-attacks}), we will be interested in $(\set{FP}_x,x,1)$-covers, which are just $(\set{U},x,1)$-covers under NFC condition. 


When analyzing our new CK structure (Section~\ref{sec:ck}), which inherits the fingerprint function from HK, we will be interested in $(\set{FP}_x,x,2)$-covers, as $r\,{=}\,1$ will no longer enable attacks to drive up estimation error.

\paragraph{Exploring time-to-cover}
Observe that even when the stream elements and the target~$x$ are independent of the internal randomness of the structure, a sufficiently long stream will almost certainly contain a cover for~$x$. For example, for CMS, this results in $\hat{n}_x$ being an overestimate of~$n_x$.  How long the stream needs to be for this to occur is what we explore next. 
Each of CMS, HK and CK use a mapping $R(K,\cdot)$ to determine the positions to which stream elements are mapped.
Let~$L^r_i$ be the number of \textit{distinct-element} evaluations of $R(K,.)$  
needed to find elements covering the target's counter in the~$i^{\mathrm{th}}$ row $r$ times. 
Then~$L^r_i$ is a negative binomial random variable with success probability $p\,{=}\, \frac{1}{m}$ and $\Pr[L^r_i \,{=}\,  z]\,{=}\,\binom{z-1}{z-r}(1-p)^{z-r}p^{r}$. 
\ignore{
The binomial coefficient in the formula comes from choosing $t\,{=}\, z\,{-}\, r$ evaluations which do not cover the targeted counter of the $i$-th row.  The choice is out of $z\,{-}\,1$ (and not $z$) evaluations. This is because $z$ counts the \textit{minimal} number of evaluations to cover the $i$-th row counter $r$ times, i.e., at the $z$-th evaluation the $i$-th row counter transitions from being covered $(r\,{-}\, 1)$ to $r$ times.
}
This is because $L^r_i$ counts the \textit{minimal} number of evaluations needed to find~$r$ elements $y_1, {\dots}, y_r$ with $R(K,y_j)[i]=p_i$.
This holds for any~$i \,{\in}\,  [k]$, and all~$L^r_i$ are independent.
%
Thus, letting $L^r\,{=}\, \max\{L^r_1,L^r_2,{\dots},L^r_k\}$, we have 

\begin{equation}	
\Pr[ L^r \leq z ]{=}\, \prod_{i\,{=}\, 1}^{k} \Pr[L^r_i \leq z]{=}\, \left(p^{r} \sum_{t\,{=}\, 0}^{z-r}\binom{t+r-1}{t}(1-p)^t\right)^k{.}
\label{eqn:pr-of-r-cover-z}
\end{equation}
%

Note that relation~\eqref{eqn:pr-of-r-cover-z} fully defines~$z$ for any fixed values of $\Prob{L^r \leq z}$, $m, k, r$.  
Thus, we will be able to relate $\Prob{\rvcoverx{r}{x}}$ and $\Prob{L^r=z}$ via the resources used in attacks, e.g., $\rvcoverx{r}{x}$ occurs iff $L^r \leq f_{m,k,r}(q_H,q_U,q_Q)$ for some function $f_{m,k,r}$ of the adversarial resources. 

When~$r\,{=}\, 1$, this simplifies to
The~$L^1_i$ are geometric random variables with success probability~$p$, and 
\begin{eqnarray}
\Pr[ L^1 \leq z ] \,{=}\,  \left((1-q)(1+q+q^2+\cdots + q^{z-1})\right)^k \,{=}\,(1-q^z)^k
\label{eqn:pr-of-1-cover-z}
\end{eqnarray}
with $q\,{=}\, 1-p$.
When $r\,{=}\,2$ we arrive at a more complicated expression 
\begin{equation}
\Pr[L^2 \leq z] = (1- zq^{z-1} +(z-1)q^z  )^k.
\label{eqn:pr-of-2-cover-z}
\end{equation}

One can show that~$\mathbb{E}[L^1]\,{=}\, \sum_{z=0}^{\infty} (1 - (1-q^z)^k)$; for typical values of~$m$, we have the very good approximation $ \mathbb{E}[L^1] \approx m H_k$, $H_k$ being the $k$-th harmonic number.
\footnote{Concretely, when $k=5,m=1000$ we have $\mathbb{E}[L^1]\approx 2283$. Experimentally, we verified this result over~$10,000$ trials with an average of~$2281$ insertions needed to find a cover set for a per-trial randomly chosen element~$x$.}  
This constant depends only on parameters~$m$ and~$k$.