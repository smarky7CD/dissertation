\label{sec:cms-attacks}
\noindent
In our attack model, if the mapping~$R(K,\cdot)$ is public, we may use the $\HASHO$ oracle (only) to find a cover set for the target~$x$ ``locally'', i.e., the step is entirely offline.  When this is not the case, we use a combination of queries to the $\UPO$ and $\QRYO$ oracles to signal when a cover set \emph{exists} among the current stream of insertions; then we make additional queries to learn a subset of stream elements that yield a cover. 

Before exploring each setting, we build up some general results. 
Let $\rvcoverx{r}{x}$ be the event that in the execution of $\ExpErrFe{\struct}{\advA}{u,v}$, the adversary queries the $\UPO$-oracle with $\up_{e_i},{\dots},\up_{e_t}$ and $e_1,{\dots},e_t$ is an $r$-cover for the target.  For concision, define random variable $\rverr=\ExpErrFe{\struct}{\advA}{u,v}$.  We will mainly focus on $\mathbb{E}[\rverr]$ when analyzing the behavior of structures, so here we observe that the non-negative nature of~$\rverr$ allows us to write $\mathbb{E}[\rverr]=\sum_{\xi > 1} \Prob{\rverr \geq \xi}$.  In determining the needed probabilities, it will be beneficial to condition on $\rvcoverx{r}{x}$, 
as this event (for particular values of~$r$) will be crucial for creating errors. 


Our attacks against CMS (and, later, HK and CK) have two logical stages.  The first stage finds the necessary type of cover for the target~$x$, and the second stage uses the cover to drive up the estimation error.  The first stage is the most interesting, as the second will typically just insert the cover as many times as possible for a given resource budget $(q_H,q_U,q_Q)$.  We note that whether or not the first stage is adaptive depends on the public/private nature of the structure's representation and hash functions, whereas the second stage will always be adaptive.  

Say $\UPO$-query budget (i.e., number of adversarial stream elements) is fixed to~$q_U$, and for the moment assume that the other query budgets are infinite.
Let some $q'_{U} \leq q_U$ of the $\UPO$-queries be used in the first stage of the attack. 
The number $q'_U$ is a random variable, call it $Q$, with distribution determined by the randomness of the structure and coins of the attacker. 
So, ~$\mathbb{E}\left[\rverr\right]$ may depend on the value of~$Q$, and then we calculate the expectation as $\mathbb{E}[\mathbb{E}[\rverr\,|\,Q]]$. 
After a cover~$\set{C}$ is found by the first stage (so $\rvcoverx{1}{x}$ holds), the second stage can insert $\set{C}$ until the resource budget is exhausted. 
Note that each insertion of~$\set{C}$ will increase the CMS estimation-error by one. Our attacks ensure that $|\set{C}| \leq k$, and so the number of $\set{C}$-insertions in the second stage is at least $\floor(\frac{q_U - Q}{k})$.  This implies that $\mathbb{E}\left[\rverr \,|\, Q\right] \geq \sum_{\xi=1}^{\lfloor(q_U - Q)/k\rfloor} \Prob{\rverr \geq \xi \,|\, Q, \rvcoverx{1}{x}} \Prob{\rvcoverx{1}{x} \,|\, Q}$
Letting $0 \leq T \leq q_U$ be the maximum number of $\UPO$-queries allowed in the first stage (i.e. $Q \leq T$), we have
\begin{equation*}
	\label{eqn:exp-err-cms}
	\mathbb{E}\left[{\rverr}\right] {\geq}\sum_{q'_U{=}0}^{T} \left\lfloor\frac{q_U {-} q'_U}{k}\right\rfloor\Prob{\rvcoverx{1}{x} \,|\, Q{=}q'_U}\Prob{Q{=}q'_U}{.}
\end{equation*}


% Public Hash Setting
\paragraph{Public hash and representation setting}
The public hash setting allows to find a cover using the $\HASHO$ oracle only  (i.e., $Q{=}0$).
This step introduces no error;
$\mathbb{E}\left[\rverr\right]{=}\allowbreak \left\lfloor\frac{q_U}{k}\right\rfloor \Prob{\rvcoverx{1}{x} \,|\, Q{=}0}$.
Given our definition of $L^1$ as the minimal number of $R(K,{\cdot})$ evaluations to find a cover,
the cover-finding step of the attack requires $k(1{+}L^1)$ $\HASHO$-queries: $k$ to evaluate $R(K,x)$, and then $kL^1$ to find a cover.
Say~$q_H$ is the $\HASHO$-oracle budget for the attack. A cover is then found iff $L^1{\leq}\frac{q_H{-}k}{k}$. 
Assuming $q_U > k$ (so that a found cover is inserted at least once) and using \eqref{eqn:pr-of-1-cover-z} we arrive at
\begin{align}\label{eqn:pub-pub-pr-cover-qH}
	\Pr\left[\rvcoverx{1}{x} \,|\, Q{=}0\right] 
	= \left(1{-}\left(1{-}{1}/{m}\right)^{\frac{q_H}{k}{-}1}\right)^{k}
\end{align}
implying $\mathbb{E}[\rverr] \geq \left\lfloor\frac{q_U}{k}\right\rfloor \left(1 - \left(1 - 1/m\right)^{\frac{q_H}{k}-1}\right)^{k}
$. For $q_H/k \gg 1$, which is likely as $q_H$ is \emph{offline} work and practical~$k$ are small, $\mathbb{E}[\rverr]\approx q_U/k$. 
The full attack can be found in Figure~\ref{fig:attack-cms-hfcs}.
\begin{figure*}[ht!]
	\Wider[2em]{
	\centering
	\begin{pchstack}[boxed,center,space=0.4em]
		\procedure[linenumbering, headlinecmd={\vspace{.1em}\hrule\vspace{.2em}}]{$\text{CoverAttack}^{\HASHO,\UPO,\QRYO}(x, K, {\repr})$}{%
			\textrm{cover} \, {\gets} \textrm{FindCover}^{\HASHO}(1,x,K)\\
			\pcuntil q_U \ \UPO \text{-queries made:}\\
			\t \pcfor e \in \textrm{cover}{:} \ \UPO(e)\\
			\pcreturn \textrm{done}
		}
		\procedure[linenumbering, headlinecmd={\vspace{.1em}\hrule\vspace{.2em}}]{$\text{FindCover}^{\HASHO}(r, x, K)$}{%
		%			\text{\textsc{Inputs}: target element~$x$, CMS representation~$M$, key~$\key$}\\
		\textrm{cover} \gets \emptyset; \,
		\textrm{found} \gets \mathsf{False}\\
		\set{I} \gets \emptyset; \,\textrm{tracker} \gets \zeros(k)\\
		\hspace{-.5em}
		\pcgraycomment{$R(K,x)[i]=\HASHO(\encode{i,K,x})$}\\
		(p_1,p_2,\ldots,p_k) \gets R(K,x)\\
		\pcwhile \textrm{not found}\\
		\t \pcif q_H \ \HASHO\text{-queries made}\\
		\t \t \pcreturn \emptyset\\
		\t y \getsr \set{U}\setminus (\set{I} \cup \{x\})\\
		\t \set{I} \gets \set{I} \cup \{y\}\\
		\t (q_1,q_2,\ldots,q_k) \gets R(K,y)\\
		\t \pcfor i \in [k]\\
		\t \t \pcif p_i = q_i~\textbf{and}~\textrm{tracker}[i] < r\\
		\t \t \t \textrm{cover} \gets \textrm{cover} \cup \{y\}\\
		\t \t \t \textrm{tracker}[i]~+= 1\\
		\t \pcif \mathsf{sum}(\textrm{tracker}) = rk\\
		\t \t \textrm{found} \gets \mathsf{True}\\
		\pcreturn \textrm{cover}
	}
	\end{pchstack}
}
\caption{Cover Set Attack for the CMS in public
	hash function setting. 
We use $R(K,x)$ to mean $(\HASHO(\encode{1,K,x}),\HASHO(\encode{2,K,x},\ldots,\HASHO(\encode{k,K,x})))$.
The attack is parametrized with  the update and $\HASHO$ query budget $q_U$ and $q_H$.
}
\label{fig:attack-cms-hfcs}

\end{figure*}

\paragraph{Private hash and private representation setting}
This is the most challenging setting to find a cover: the privacy of hash functions effectively makes local hashing useless, and the private representation prevents the adversary from learning anything about the result of online hash computations.
\begin{figure*}[ht!]
	%\Wider[2em]{
	%	\centering
		\begin{pchstack}[boxed,center,space=0.5em]
			\begin{pcvstack}[space = 0.45em]
				\procedure[linenumbering, headlinecmd={\vspace{.1em}\hrule\vspace{.2em}}]{$\text{CoverAttack}^{\UPO,\QRYO}(x, \bot, \bot)$}{%
					\textrm{cover} \gets \textrm{FindCover}^{\UPO,\QRYO}(x)\\
					\pcuntil q_U \ \UPO \text{-queries made:}\\
					\t \pcfor e \in \textrm{cover}{:} \ \UPO(e)\\
					\pcreturn \textrm{done}
				}
				\procedure[linenumbering, headlinecmd={\vspace{.1em}\hrule\vspace{.2em}}]{$\text{MinUncover}^{\UPO,\QRYO}(x, a', \text{cover})$}{%
					b' \gets -1\\
					\pcwhile a' \not= b'\\
					\t \pcif (q_U-|\textrm{cover}| + 1)\UPO\text{-} \\
					\t \t \text{ or } q_Q \ \QRYO\text{-queries made:}\\
					\t \t \t \pcreturn \textrm{cover}\\ %\textrm{False}\\
					\t b' \gets a'\\
					\t \pcfor y \in \text{cover}: \UPO(y)\\
					\t a' \gets \QRYO(x)\\
					\pcreturn a'
				}
			\end{pcvstack}
			\procedure[linenumbering, headlinecmd={\vspace{.1em}\hrule\vspace{.2em}}]{$\text{FindCover}^{\UPO,\QRYO}( x)$}{%
				\pcgraycomment{find $1$-cover for x}\\
				\textrm{cover} \gets \emptyset\\
				\textrm{found} \gets \mathsf{False}\\
				\streamvar{I} \gets \emptyset; a \gets \QRYO(x) \\
				\hspace{-.5em}
				\pcwhile \textrm{not found} \\
				\t \pcif q_U \ \UPO \text{- or } q_Q \ \QRYO\text{-queries made}\\
				\t \t \pcreturn \textrm{cover} \\
				\t y \getsr \univ \setminus (\streamvar{I} \cup \{x\})\\
				\t \streamvar{I} \gets \streamvar{I} \cup \{y\}\\
				\t \UPO(y); \ a' \gets \QRYO(x)\\
				\t \pcif a' \not= a: \\
				\t \t \textrm{cover} \gets \{y\}\\
				\t \t \textrm{found} \gets \mathsf{True}\\
	%			\t \t \pcif |\textrm{cover}| = k: \pcreturn \textrm{cover}\\
				\pcfor i \in [2, 3, \dots, k]\label{alg:CMS:CK:change}\\
				\t a \gets \text{MinUncover}^{\UPO,\QRYO}(x, a', \text{cover}) \\
				\t \pcif a = \textrm{cover}: \pcreturn \textrm{cover}\\
				\t \pcfor y \in \set{I} \ \pcgraycomment{in order of insertion to $\set{I}$}\\
				\t \t \pcif q_U \ \UPO \text{- or } q_Q \ \QRYO\text{-queries made}\\
				\t \t \t \pcreturn \textrm{cover}\\
				\t \t \UPO(y); a' \gets \QRYO(x) \\
				\t \t \pcif a' \not= a: \\
				\t \t \t \textrm{cover} \gets \textrm{cover} \cup \{y\}\\
	%			This is just here so the algorithm is exactly the same for the CK!
				\t \t \t \streamvar{I} \gets \set{I} \setminus \{y\}\\
				\t \t \t \textbf{break}\\
				\pcreturn \textrm{cover}
			}
		\end{pchstack}
	%}
	\caption{Cover Set Attack for the CMS in private
		hash function and private representation setting. 
		The attack is parametrised with  the update and query query budget $q_U$ and $q_Q$.
	}
	\label{fig:attack-cms-iqcsa}
\end{figure*}


In Figure~\ref{fig:attack-cms-iqcsa} we give an attack for the private hash and private representation setting.  This is the most challenging setting for finding a cover set: the privacy of the hash functions makes local hash computations effectively useless, and the privacy of the representation prevents the adversary from using it to view the result of online hash computations.  The attack begins by querying~$x$ to learn its current frequency estimate; let $(p_1,p_2,\ldots,p_k) \gets R(K,x)$ and let $M[1][p_1]=c_1,\ldots,M[k][p_k]=c_k$ be the values of the counters associated to~$x$ at this time, i.e., $\min_{i\in[k]}\{c_i\}=a \geq 0$. 

The attack then inserts distinct random elements that are not equal to~$x$, checking the estimated frequency after each insertion until the estimated frequency for~$x$ increases to $a+1$, as this signals that a cover set for~$x$ has been inserted. Let $\streamvar{I}$ be the stream of inserted elements at the moment that this happens.  At this point, we begin the first ``round" of extracting from~$\streamvar{I}$ a 1-cover.  Say the last inserted element was $z_1$. As this caused the CMS estimate to increase, $z_1$ must share at least one counter with~$x$.  Moreover, any counter covered by~$z_1$ must have been minimal, i.e., still holding its initial value~$c_i$, at the time that~$z_1$ was inserted.  Thus, we set our round-one candidate cover set~$\set{C}_1\gets\{z_1\}$.  Notice that by definition, the insertion of~$z_1$ increases the estimation error by one.

Let $\set{M}(\set{C}_1)=\{i \in [k] \,|\, \exists z \in \set{C}: R(K,z)[i]=p_i\}$, i.e., the set of rows whose $x$-counters are covered by $\set{C}_1$, and let $\delta_1 = \min_{j \not\in \set{M}(\set{C}_1)}\{M[j][p_j]\} - \min_{i \in \set{M}(\set{C}_1)}\{M[i][p_i]\}$.  Notice that $\delta_1$ is the gap between the smallest counter(s) \emph{not} covered by~$\set{C}_1$, and the smallest counter(s) that are covered by~$\set{C}_1$.  (Observe that~$z_1$ may also cover non-minimal $x$-counters.)  Thus, if we now reinsert~$\set{C}_1$ a total of $\delta_1$ times, this gap shrinks to zero; reinserting it once more will cause some $x$-counter that is \emph{not} covered by~$\set{C}_1$ to become minimal, and we can observe this by making an estimation query (i.e. a $\QRYO$ call) after each reinsertion.  

\noindent\textit{Example: }
Say we have~$k=4$, and prior to the first insertion of~$z_1$ (as part of~$\streamvar{I}$) we have $M[1][p_1]=2$, $M[2][p_2]=3$, $M[3][p_3]=5$ and $M[4][p_4]=0$. Now, say that~$z_1$ covers the $x$-counters in rows 1,4: then upon first inserting $z_1$, we have $M[1][p_1]=3$, $M[2][p_2]=3$, $M[3][p_3]=5$ and $M[4][p_4]=1$.  We create~$\set{C}_1=\{z_1\}$, and compute $\delta_1 = 3-1=2$.  If we were to insert~$\set{C}_1$ twice more, we would have $M[1][p_1]=5$, $M[2][p_2]=3$, $M[3][p_3]=5$ and $M[4][p_4]=3$; if we had checked the CMS estimate for~$n_x$ after each insertion, we would have observed responses $2$ and $3$.  After $\delta_1+1=3$ re-insertions of~$\set{C}_1$, we would have $M[1][p_1]=6$, $M[2][p_2]=3$, $M[3][p_3]=5$, $M[4][p_4]=4$, and the CMS estimate of~$n_x$ would remain $3$ because now $M[2][p_2]$ is minimal.\hfill$\circ$


Notice that the $\delta_1+1$ re-insertions of~$\set{C}_1$ will increase the CMS estimate of~$n_x$ by exactly $\delta_1$. At this point we begin round 2, searching for $z_2 \in \streamvar{I}\setminus\set{C}_1$ that covers the newly minimal $x$-counters.  Recall that the elements of~$\streamvar{I}$ are distinct (by design), so if we reinsert $\streamvar{I}\setminus\set{C}_1$ \emph{in order} we are guaranteed to hit some satisfying~$z_2 \neq z_1$, and this can be observed by checking the CMS estimate of $n_x$ after each element is reinserted.  As was the case for~$z_1$, we know that~$z_2$ covers the currently minimal $x$-counters, and that prior to reinserting~$z_2$ these counters had not changed in value since the end of round 1.  Thus, reinserting~$z_2$ increases the estimation error by one.  We set $\set{C}_2 \gets \set{C}_1 \union \{z_2\}$, and then switch to reinserting $\set{C}_2$ a total of $\delta_2+1$ times (where $\delta_2$ is defined analogously to $\delta_1$) to end round 2.  Again, this increases the estimation error by $\delta_2$.

Continuing this way, after some $\ell \leq k$ rounds we will have found a complete 1-cover for~$x$.  There can be at most~$k$ rounds, because each round~$i$ adds exactly one new element~$z_i$ to the incomplete cover $\set{C}_{i-1}$, and there are only~$k$ counters to cover.  Notice that in round~$\ell$, when we reinsert~$\set{C}_\ell$ we will never observe that some new $x$-counter has become minimal: all $x$-counters are covered by~$\set{C}_\ell$, so all will be increased by each reinsertion.  Nonetheless, each reinsertion of~$\set{C}_\ell$ adds one to the estimation error, and these re-insertions may continue until the resource budget is exhausted, i.e., until a total of~$q_U$ elements have been inserted (via $\UPO$) as part of the attack.

The number of $\UPO$-queries (i.e. insertions) required to reach the complete cover~$\set{C}_\ell$ is 
\[
q'_U \leq \ell|\streamvar{I}| + \sum_{i=1}^{\ell-1}(\delta_i + 1)(i) = \ell|\streamvar{I}| + \frac{\ell(\ell-1)}{2} + \sum_{i=1}^{\ell-1} i \delta_i
\]
and so~$\set{C}_\ell$ can potentially be reinserted at least $\lfloor (q_U - q'_U)/\ell \rfloor$ times, each time adding one to the estimation error.  We say \emph{potentially} because the~$\QRYO$-query budget may be the limiting factor; we'll return to this in a moment.  For now, assuming~$q_Q$ is not the limiting factor, the error introduced by the attack is
\begin{align*}
	\rverr &\geq \left\lfloor \left( \ell + \sum_{i=1}^{\ell-1}\delta_i \right) + \left( \frac{q_U - \ell|\streamvar{I}| - \frac{\ell(\ell-1)}{2} - \sum_{i=1}^{\ell-1} i \delta_i}{\ell} \right) \right\rfloor\\
	%&= \left\lfloor \frac{1}{\ell} \left( \ell^2 + \ell\sum_{i=1}^{\ell-1}\delta_i +  q_U - \ell|\streamvar{I}| - \frac{\ell(\ell-1)}{2} - \sum_{i=1}^{\ell-1} i \delta_i \right) \right\rfloor\\
	%&= \left\lfloor\frac{1}{\ell} \left( \frac{\ell(\ell+1)}{2} + q_U - \ell|\streamvar{I}| + \sum_{i=1}^{\ell-1}(\ell-i)\delta_i \right) \right\rfloor\\
	&= \left\lfloor \left( \frac{\ell+1}{2} + \frac{1}{\ell}\left(q_U + \sum_{i=1}^{\ell-1}(\ell-i)\delta_i \right) - L^1 \right) \right\rfloor
\end{align*}
where the final line holds because $|\streamvar{I}|$ is, by construction, precisely~$L_1$.  We note that $\rverr$ is a function of several random variables: $L^1, \ell, \{\delta_i\}_{i \in [\ell-1]}$.  

%\footnote{\mia{Idea (feel free to ignore). If we are making all of these approximations later on, we could consider switching from $k L^1$ to $\sum_{i=1}^{k} L_i^1$ (at some point or just in final approximations). Why is $\sum_{i=1}^{k} L_i^1$ an upper bound on the number of insertions of elements in $\streamvar{I}$ in all rounds? 
%		Round $i$ finds \textit{the first} element in $\streamvar{I}$ that covers the current minimal counter. The length of the search over $\streamvar{I}$ is equal to $L^1_j$ for some $j$ - by definition. Minimal counters at the beginning of round $i$ are not covered by $\set{C}_i$. So, $L^1_j$ was not `used' to model the length of any of the previous searches.\\
%		Potentially - this could help us to compare CK and CMS errors.
%}}

We would like to develop an expression for $\mathbb{E}[\rverr]$, so we observe that for practical values of~$k,m$ (e.g., $k=4$, with $m \gg k$) it is likely that $\ell=k$.  We have $\ell < k$ only if one or more of the covering elements cover multiple $x$-counters, and for small~$k \ll m$ this is unlikely.  We approximate $\rverr$ with $\widehat{\rverr}$ by replacing $\ell$ with~$k$, dropping the flooring operation, arriving at
\[
\mathbb{E}[\rverr]\approx\mathbb{E}[\widehat{\rverr}]
\approx \left( \frac{k+1}{2} + \frac{1}{k}\left(q_U + \sum_{i=1}^{k-1}(k-i)\mathbb{E}[\delta_i] \right) - \mathbb{E}[L^1] \right)
\]
Rearranging and using the very tight approximation $\mathbb{E}[L^1]\approx mH_k$, we have
\begin{align*}
	\mathbb{E}[\widehat{\rverr}] &\approx \left( \frac{q_U}{k} - mH_k \right) + \frac{k+1}{2} + \left(\frac{1}{k}\sum_{i=1}^{k-1}(k-i) \mathbb{E}[\delta_i] \right) 
	%&=\left(\frac{q_U}{k} - mH_k \right) + O(k)
\end{align*}
%where the second line follows because bounding $\mathbb{E}[\delta_i]$ by any constant results in $\frac{1}{k}\sum_{i=1}^{k-1}(k-i) \mathbb{E}[\delta_i] = O(k)$.  
We do not have a crisp way to describe the distribution of the $\delta_i$ random variables, but we can make some educated statements about them.  The expected value of \emph{any} counter~$M[i][j]$ after~$\streamvar{I}$ has been inserted is $|\streamvar{I}|/m \approx mH_k/m=H_k$, and $H_k < 4$ for $k \leq 30$ (and practical values of $k$ are typically much less than 30); moverover, standard balls-and-bins arguments tell us that as the number of balls approaches $m \ln m$, the \emph{maximum} counter value in any row approaches the expected value.  Since $\delta_1 \leq \max_{j \not\in \set{M}(\{z_1\})}\{M[j][p_j]\} - \min_{i \in \set{M}(\{z_1\})}\{M[i][p_i]\}$, we can safely assume that $\mathbb{E}\left[\delta_1\right]$ is upper-bounded by a constant that is small relative to $m, q_U/k$.  
%moreover, a standard balls and bins result tells us that the maximum counter value in any row is at most $\frac{(H_k)\cdot 3\ln m}{\ln \ln m}$ with probability at least $1-1/m$.  For $k \leq 30, m=2048$, this maximum-value upperbound is 45.

After inserting~$\set{C}_1{=}\,\{z_1\}$ a total of~$\delta_1\,{+}\,1$ times, we switch to reinserting $\streamvar{I}\setminus \set{C}_1$ until we find a $z_2$ that covers the currently minimal $x$-counters.  When we begin to reinsert~$\set{C}_2$, we know by construction that $\delta_2 \leq \min_{j \not\in \set{M}(\{z_1,z_2\})}\{M[j][p_j]\} - \left(\min_{j \not\in \set{M}(\{z_1\})}\{M[j][p_j]\}+1\right)$.  For the first term in the difference, we ``roll back" one round; say that $\alpha=\max_{j \not\in \set{M}(\{z_1\})}\{M[j][p_j]\}$. Then, being very pessimistic, we know that $\min_{j \not\in \set{M}(\{z_1,z_2\})}\{M[j][p_j]\} \leq 2\alpha + (\delta_1\,{+}\,1)$: in finding~$z_2$, we reinsert at most all of $\streamvar{I}\setminus\{z_1\}$, which would add another (at most) $\alpha$ to that maximum counter value, and the repeated insertions of~$\set{C}_1$ could have added at most~$\delta_1\,{+}\,1$ to said maximum counter.  However, the second term in the difference is at least $\delta_1\,{+}\,1$, so $\delta_2 \leq 2\alpha$ and we have already argued that $\alpha$ is in the neighborhood of~$H_k \,{<}\, 4$.  Continuing this this way, we reach the conclusion that 
the dominant term in $\mathbb{E}[\widehat{\rverr}] \approx \mathbb{E}[\rverr]$ will be $\frac{q_U}{k}-mH_k$. This is observed experimentally in Table~\ref{tab:attack-comp}. For realistic values of~$k$, significant error will be created when $q_U \gg (mk)H_k$. For example, when $k\,{=}\,4, m\,{=}\,2048$ we require $q_U \gg 17067$; this is likely not a real restriction in most practical use-cases of CMS, e.g., computing the heavy hitter flows traversing a router.

%\tsnote{Can we say that this is observed experimentally?  Will any of our experiments surface this?}
%\mia{
	%It depends how off you want to be with your estimate. If being off by $m$ is OK, then yes. I would say this depends on the relation between $q_U/k$ and $m$.
	%}
%But almost certainly there is at least one of the $\delta_i \gg 0$.  To see this, recall that $\delta_1$ is the gap between the minimum counter covered $\set{C}_1$ and the rest of the $x$-counters, and that the minimum counter covered by $\set{C}_1=\{z_1\}$ is just one greater than its initial value.  In particular, during the insertion of~$\streamvar{I}\setminus z_1$, the $x$-counters that are \emph{not} covered by~$z_1$ would have increased.  This is because $z_1$ must cover the last $x$-counters holding their initial values, and the most likely case is that $z_1$ covers just one counter, as the probability of covering more counters decreases exponentially in $1/m$.  This suggests that $\delta_1 \geq 1$, at least.  And this is amplified as the rounds go on, because the counters in $\set{C}_i$ are not increased by reinsertions of $\streamvar{I}\setminus\set{C}_i$, whereas the $x$-counters not covered by~$\set{C}_i$ would have been increased with \emph{every} prior reinsertion of $\streamvar{I}\set{C}_j$ for $j < i$.  

Returning to the matter of exhausting the $\QRYO$-budget, the total number of $\QRYO$-queries for the attack depends somewhat heavily on whether or not $\ell=k$.  If $\ell = k$ then $|\set{C}_k|=k$, and we know that a complete cover has been found.  Thus, we do not need to make any $\QRYO$-queries during reinsertions of~$\set{C}_k$.  If $\ell < k$, however, then we must make $\QRYO$-queries during reinsertions of~$\set{C}_\ell$, because we do not know that $\set{C}_\ell$ contains a complete cover.  

Either way, the number of $\QRYO$-queries need to reach~$\set{C}_\ell$ is 
%\begin{align*}
%[\ell=k]:\quad q'_Q &\leq k(|\streamvar{I}|+1) + \sum_{i=1}^{k-1}\delta_i  \\
%[\ell<k]:\quad q'_Q &\leq \ell(|\streamvar{I}|+1) + \sum_{i=1}^{\ell-1}\delta_i  + (q_U-q'_U)/\ell\\
%&\leq (\ell-1)|\streamvar{I}|+ \frac{\ell+1}{2} + \frac{1}{\ell}\sum_{i=1}^{\ell-1}(\ell-i)\delta_i +  \frac{q_U}{\ell}
%\end{align*}
$q'_Q \leq 1 + \ell|\streamvar{I}| + \sum_{i=1}^{\ell-1}(\delta_i +1)$, and the expected gap between $q'_U$ and $q'_Q$ is 
\begin{align*}
	\mathbb{E}[q'_U - q'_Q] &\approx \mathbb{E}\left[\sum_{i=1}^{\ell-1}i(\delta_i +1) - \sum_{i=1}^{\ell-1}(\delta_i +1)\right] \\
	%&= \mathbb{E}\left[\sum_{i=1}^{\ell-1}(i-1)(\delta_i +1)\right] \\
	%&= \mathbb{E}\left[\sum_{i=1}^{\ell-1}(i-1)(\delta_i) + \sum_{i=1}^{\ell-1}(i-1)\right] \\
	%&\leq \mathbb{E}\left[\sum_{i=1}^{\ell-1}\delta_i \right] + \mathbb{E}\left[\frac{(\ell-1)(\ell-2)}{2}\right]\\
	&\leq \mathbb{E}\left[\sum_{i=1}^{\ell-1}\delta_i \right] + \frac{(k-1)(k-2)}{2}\\
	& \leq k\mathbb{E}\left[\max_{i\in[\ell-1]}\{\delta_i\}\right] + \frac{(k-1)(k-2)}{2}
\end{align*}
By the arguments just given about the~$\delta_i$, we can safely bound $\mathbb{E}\left[\max_{i\in[\ell-1]}\{\delta_i\}\right]$ by $kH_k$.  So $\mathbb{E}[q'_U - q'_Q] = O(k^2)$ with a small hidden constant.
Thus, the expected numbers of $\UPO$-queries and $\QRYO$-queries expended to find the complete cover~$\set{C}_\ell$ are similar, especially for realistic values of~$k$.  

Now, in the most likely case that $\ell=k$, no further $\QRYO$-queries are needed. Hence, when $\ell=k$, the overall error induced by the attack will be determined by the insertion/$\UPO$-budget ($q_U$) when the \emph{total} $\QRYO$-budget~$q_Q$ is approximately the insertion-budget required for finding the cover.   When $\ell < k$, in order for the overall error to be determined by the insertion budget, the total $\QRYO$-budget needs to accommodate $q'_Q + (q_U - q'_U)/\ell$ queries. The second summation comes from the fact that while accumulating error via re-insertions of~$\set{C}_\ell$, we must make one $\QRYO$-query per reinsertion.  
%
This is a potentially large jump in the number of estimation queries required, from $\ell=k$ to \ $\ell<k$.  But in reality the jump might be less important than it appears: if $\ell < k$ then given our intuition about the $\delta_i$, it seems likely that if some~$\set{C}_i$ is taking a large number of insertions, one can likely assume that~$\set{C}_i$ is a complete cover, cease making estimation queries and switch to an insertion only strategy.


%%% Pub hash, priv rep %%%
 \paragraph{Public hash and private representation setting}
Observe that the public representation is never used in our attack in the public hash and public representation setting. Therefore, in this public hash and private representation setting, the same attack can be used. The same analysis applies.

%%% priv has, pub rep %%%
\paragraph{Private hash and public representation setting}
The public representation allows for an attack similar to our attack in the public hash settings (Figure~\ref{fig:attack-cms-iocsa}). Here, we use the $\UPO$-oracle instead of the $\HASHO$-oracle to find a cover. By comparing the state before and after adding an element 
it is easy to deduce the element's counters (as they are the only ones to change).
Our attack first adds the target to get its counters. Then, we keep inserting \textit{distinct} elements, comparing the state before and after until a cover ~$\set{C}$ is found.
By the definition of $L^1$, the cover is found with $(q'_U  = 1+ L^1)$ $\UPO$-queries, and is after reinserted $\lfloor (q_U - q_U')/|\set{C}| \rfloor$ times, each time adding one to the estimation error. Hence, 
$\rverr \geq \lfloor (q_U - 1 - L^1)/|\set{C}|\rfloor \geq \lfloor (q_U -1 - L^1)/k \rfloor$  and
%\begin{align*}
$\mathbb{E}[\rverr] \geq \frac{q_U-1-\mathbb{E}\left[L^1\right]}{k} \approx \frac{q_U - mH_k}{k}.$
%\end{align*}
\begin{figure*}[ht!]
	\centering
	\begin{pchstack}[boxed,space=0.5em]
		\procedure[linenumbering, headlinecmd={\vspace{.1em}\hrule\vspace{.2em}}]{$\text{CoverAttack}^{\UPO}(x, \bot, \repr)$}{%
			\textrm{cover} \gets \textrm{FindCover}^{\UPO}(1,x, \repr)\\
			\pcuntil q_U \ \UPO \text{-queries made:}\\
			\t \pcfor e \in \textrm{cover}{:} \ \UPO(e)\\
			\pcreturn \textrm{done}
		}
		\procedure[linenumbering, headlinecmd={\vspace{.1em}\hrule\vspace{.2em}}]{$\text{FindCover}^{\UPO}(r, x, \repr)$}{%
			\textrm{cover} \gets \emptyset; \,
			\textrm{found} \gets \mathsf{False}\\
			\set{I} \gets \emptyset; \,\textrm{tracker} \gets \zeros(k)\\
			\repr' \gets \UPO(x)\\
			\pcgraycomment{compute $x$'s indices}\\
			\pcfor i \in [k] \\
			\t \pcfor  j \in [m]\\
			\t \t \pcif \repr'[i][j] \not= \repr[i][j]\\
			\t \t \t  p_i \gets j; \textbf{break};\\
			\hspace{-.5em}
			\pcwhile \textrm{not found}\\
			\t \pcif q_U \ \UPO\text{-queries made}:\pcreturn \emptyset\\
			\t y \getsr \set{U}\setminus (\set{I} \cup \{x\})\\
			\t \set{I} \gets \set{I} \cup \{y\}\\
			\t  \repr \gets \repr'\\
			\t \repr' \gets \UPO(y) \\
			\t \pcgraycomment{compute $y$'s indices}\\
			\t \pcfor i \in [k] \\
			\t \t \pcfor  j \in [m]\\
			\t \t \t \pcif \repr'[i][j] \not= \repr[i][j]\\
			\t \t \t  \t q_i \gets j; \textbf{break};\\
			\t \pcfor i \in [k]\\
			\t \t \pcgraycomment{compare $x$'s and $y$'s indices row by row}\\
			\t \t \pcif p_i = q_i~\textbf{and}~\textrm{tracker}[i] < r\\
			\t \t \t \textrm{cover} \gets \textrm{cover} \cup \{y\}\\
			\t \t \t \textrm{tracker}[i]~+= 1\\
			\t \pcif \mathsf{sum}(\textrm{tracker}) = rk\\
			\t \t \textrm{found} \gets \mathsf{True}\\
			\pcreturn \textrm{cover}
		}
	\end{pchstack}
	\caption{Cover Set Attack for the CMS in private
		hash function and public representation setting. 
		The attack is parametrized with  the update query budget $q_U$.
	}
	\label{fig:attack-cms-iocsa}
\end{figure*}
 
 