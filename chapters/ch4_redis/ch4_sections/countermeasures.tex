In this section, we outline some countermeasures that limit the effectiveness of our attacks. For remarks on the Bloom filter and Cuckoo filter we again refer the reader to~\cite{cryptoeprint:2024/1312}.

Protecting the count-min sketch and Top-K against frequency estimation attacks is challenging. Recall, that both the Count-Min sketch and the Top-K are a class of probabilistic data structures called \emph{compact frequency estimators} (CFE). In Chapter 3 we explore both of these structures in detail, and show that even when switching the hash functions to a keyed primitive (e.g. a PRF) and keeping the internal state of the structure efficient attacks that cause massive frequency estimation errors are still possible~\cite{markelon23}. That is the leakage from insertions and queries to a black-boxed structure is sufficient to carry out the style of attacks we present in this paper. The choices of Redis make these attacks easier to carry out, but findings are negative in any case. 

It is of great interest to explore secure PDS for frequency estimate queries that are tenable for real world applications. One could of course disallow queries to the structure, or use some public-key infrastructure to only allow insertions from authenticated parties. However, this clearly limits both the usability and performance. Another possibility is to explore new ways of constructing frequency estimation PDS, such as the Count-Keeper introduced in~\cite{markelon23}. While this structure remains susceptible to the types of attacks we present here, they are less effective, and the Count-Keeper has a native ability to flag suspicious frequency estimates. 

\subsection{Concluding Remarks}

We made a comprehensive security analysis of the Redis PDS suite, developing 10 different attacks across four PDS. 
Our attacks can be used to cause severe disruptions to the performance of systems relying on these PDS, ranging from mis-estimation of data statistics to triggering denial-of-service attacks. 
Our work illustrates the importance of low-level algorithmic choices and the dangers of using weak hash functions in PDS. 

Our work opens up interesting directions for future work. Various other PDS suites exist in the wild, such as in Google BigQuery and Apache Spark, and could also be subjected to detailed security analysis as we have done for Redis here. 
Methods to provably protect PDS against attacks have been proposed in~\cite{NaorY15,clayton2019,FPUV22,PatersonR22}. However, these analyses tend to focus on textbook versions of the PDS. 
Adapting these analyses to cater to the specifics of different implementations would help improve confidence in the deployed variants.

At a higher level, there still seems to be a lack of understanding  in the broader developer community about the risks of using PDS in potentially adversarial settings. 
Work is needed to educate developers about these risks; we hope this paper can play a part in this effort. 
As an alternative, in an effort to shield developers from these risks, one could develop new PDS implementations that are secure by default and package them in the form of easily consumed libraries with safe APIs. Such an effort could leverage the experience that the research community has gained from developing ``safe by default'' cryptographic libraries.
