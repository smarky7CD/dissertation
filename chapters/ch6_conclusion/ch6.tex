\chapter{Conclusion and Future Work}\label{chap:conclusion}

This work contributes to a growing body of work that seeks to formalize the security of data structures, bridging the gap between theoretical designs and practical implementations. By rigorously analyzing compact frequency estimators and probabilistic skipping-based data structures under adversarial models, this work demonstrates that the assumptions underlying traditional performance guarantees often collapse in adversarial environments.

The results show that existing data structures, including Count-min sketch, HeavyKeeper, hash tables, skip lists, and treaps, are susceptible to adaptive attacks that can severely degrade correctness and performance. The development of Count-Keeper and the modified PSDS variants provides a concrete pathway toward provably secure, efficient, and practically deployable structures. These constructions are not merely theoretical but are supported by formal proofs and empirical evaluations, highlighting their resilience against adaptive adversaries.

The Redis case study underscores the importance of bridging theory and practice. By exposing vulnerabilities in Redis’s CFE implementations, this work demonstrates how small design decisions, such as the use of non-cryptographic hash functions, can open pathways for sophisticated attacks. The proposed countermeasures offer actionable steps toward more secure deployments of widely-used data structures.

In sum, this work not only identifies vulnerabilities in widely adopted data structures but also presents formalized, provably secure alternatives. It offers a rigorous foundation for future work in designing efficient and secure data structures and serves as a call to incorporate adversarial considerations into core data structure design. Below, we highlight several promising directions inspired by this work.

Our work on CFEs suggests that nearly all existing sketch-based frequency estimators may be susceptible to the kinds of attacks we present. While our proposed Count-Keeper structure mitigates the extent of damage an adaptive adversary can inflict and enables the flagging of potentially adversarially influenced estimates, it does not prevent adversaries from causing large frequency estimation errors. This aligns with recent theoretical work: Hardt and Woodruff~\cite{hardt2013}, Cohen et al.\cite{cohen2022robust}, and Ben-Eliezer et al.\cite{BenEliezer2022} have shown that linear sketches (including CMS but not HK) are fundamentally non-robust against well-resourced adaptive attacks, particularly in various $L_p$-norm estimation tasks such as solving the $k$-heavy-hitters problem relative to the $L_2$-norm.

Thus, a significant open problem is the design of a CFE that not only detects adversarial manipulation but also outright prevents it. While Hassidim et al.~\cite{HassidimKMMS20} explore enhancing streaming algorithms with differential privacy to achieve robustness, their approach is impractical for real-world deployments in our setting. Either the query responses would become excessively lossy, or the number of queries allowed would need to be severely restricted to deter attacks. Developing a generically robust CFE structure remains an open and highly relevant challenge. Additionally, it may be fruitful to explore further compositions of CPDS, as we did with Count-Keeper. For instance, could one design more performant and robust approximate set-membership structures by combining (say) Bloom filters and Cuckoo filters?

The Redis case study also opens avenues for further investigation. Many other PDS suites are deployed in real-world systems, such as Google BigQuery and Apache Spark, and could be subjected to the same detailed security analysis applied here. While methods for provably securing PDS against attacks have been proposed~\cite{NaorY15,clayton2019,FPUV22,PatersonR22,markelon23,filic2025deletions}, these analyses often focus on textbook versions of CPDS. Extending these analyses to the specific implementations used in practice could greatly enhance confidence in their security.

More broadly, there is still a lack of understanding in the developer community about the risks of using CPDS in adversarial environments. Further work is needed to educate practitioners about these risks, and we hope this work can contribute to this effort. Alternatively, to shield developers from these risks, one could develop new CPDS implementations that are secure by default, packaging them into user-friendly libraries with safe APIs. This effort could benefit from the experience of the cryptographic research community, which has developed “safe-by-default” cryptographic libraries.

Regarding PSDS, while our theoretical bounds provide rigorous security guarantees, there is scope to tighten these bounds further. Future work could extend our analysis to related data structures, such as zip trees~\cite{tarjan2021zip}, zip-zip trees~\cite{gila2023zip}, skip graphs~\cite{aspnes2007skip}, and randomized meldable heaps~\cite{gambin1998randomized}. Additionally, exploring more sophisticated deletion strategies may reduce memory overhead while preserving security guarantees. One promising direction involves localized re-initializations, allowing portions of a structure to be safely rebuilt without compromising robustness. Finally, our Adaptive Adversary Property Conservation (AAPC) framework could be applied to broader classes of data structures and properties, given its inherent extensibility.