\documentclass[editMode]{ufdissertation}\sloppy
\usepackage[colorlinks=true]{hyperref}
%%%%%%%%%%%%%%%%%%%%%%%%%%%%%%%%%%%%%%%%%%%%%%%%%%%%%%%%%%%%%%%%%%%%%%%%%%%%%%%%
%%%                 User Package and Style File loading.
%%%%%%%%%%%%%%%%%%%%%%%%%%%%%%%%%%%%%%%%%%%%%%%%%%%%%%%%%%%%%%%%%%%%%%%%%%%%%%%%

%\usepackage{CustomMacros}%  This is a user macro/style file.

%\usepackage{tikz}%       tikz is used by almost everyone, but certainly by me for this.
%\usepackage{pgfplots}%   pgfplots is tikz but better.
%\usepackage{amsrefs}%   amsrefs contains the .bibtex style content for mathematician papers.
%\usepackage{algpseudocode}


\usepackage{scrextend}
\renewcommand{\thefootnote}{\arabic{footnote}}
\renewcommand{\thefootnotemark}{\arabic{footnote}}
\deffootnote{1.5em}{0em}{\thefootnotemark\quad}
\usepackage[all]{nowidow} %avoids single lines of text at the top or bottom of page.  If any of your packages cause this package not to work, you will have to manually remove any orphans/widows or add a line penalty

% Sam's custom commands
\usepackage{natbib}
\usepackage{amssymb}
\usepackage{amsmath,amsfonts}
\usepackage{algorithmic}
\usepackage{graphicx}
\usepackage{float}
\usepackage{textcomp}
\usepackage[symbol]{footmisc}
\usepackage{xcolor}
\def\BibTeX{{\rm B\kern-.05em{\sc i\kern-.025em b}\kern-.08em
    T\kern-.1667em\lower.7ex\hbox{E}\kern-.125emX}}

% math script font; extra commands to make slightly larger
\DeclareFontFamily{OT1}{pzc}{}
\DeclareFontShape{OT1}{pzc}{m}{it}%
{<-> s * [1.150] pzcmi7t}{}
\DeclareMathAlphabet{\mathscript}{OT1}{pzc}{m}{it}
\DeclareMathAlphabet{\mathsc}{OT1}{cmr}{m}{sc}
\DeclareMathAlphabet{\mathsl}{OT1}{cmr}{m}{sl}

% Formatting and common macros for crypto papers. Include this first.
\usepackage{graphics}
\usepackage[font={small}]{caption}
\usepackage{hyperref}
\usepackage[capitalise]{cleveref}
\usepackage{xspace}
\usepackage{sidecap}
\usepackage{amsthm}
%\newtheorem{lemma}{Lemma}
%\newtheorem{theorem}{Theorem}
\newtheorem{corollary}{Corollary}
\newtheorem{definition}{Definition}
\newtheorem{remark}{Remark}
\newcommand{\missingqed}{}
\usepackage{pifont}
\usepackage{parskip}
\usepackage{multirow}
\usepackage{array}
\usepackage{enumitem}
\usepackage[normalem]{ulem}
\useunder{\uline}{\ul}{}
%\usepackage{slashbox}
%\usepackage[table]{xcolor}
\usepackage{framed}
\usepackage[
	% lambda,
	landau,
	% operators,
	% probability,
	% sets,
	% logic,
	% complexity,
	% asymptotics
]{cryptocode}
\usepackage{thmtools, thm-restate}
\usepackage{adjustbox}
\newcommand\Wider[2][3em]{%
	\makebox[\linewidth][c]{%
		\begin{minipage}{\dimexpr\textwidth+#1\relax}
			\raggedright#2
		\end{minipage}%
	}%
}

% Editorial
\renewcommand{\paragraph}[1]{\smallskip\noindent\textbf{#1}.}
\newcommand{\heading}[1]{\paragraph{#1}}
\newcommand{\ala}{{a la}\xspace}
\newcommand{\etal}{{et al.}\xspace}
\newcommand{\viceversa}{{vice versa}\xspace}
\newcommand{\ignore}[1]{}

% Fonts for various types
\newcommand{\notionfont}[1]{{#1}}
\newcommand{\varfont}[1]{\textit{#1}}
\newcommand{\flagfont}[1]{\mathsf{#1}}
\newcommand{\vectorfont}[1]{\vec{#1}}
\newcommand{\oraclefont}[1]{\cryptofont{#1}}
\newcommand{\schemefont}[1]{\textnormal{\textsc{#1}}}
\newcommand{\expfont}[1]{{{\tiny\MakeLowercase{\textnormal{#1}}}}}
\newcommand{\procfont}[1]{\mathsf{#1}}
\newcommand{\algorithmfont}[1]{\mathcal{#1}}
\newcommand{\adversaryfont}[1]{\mathcal{#1}}
\newcommand{\setfont}[1]{\mathcal{#1}}
\newcommand{\cryptofont}[1]{\textup{\textbf{#1}}\hspace{0.5pt}}
\newcommand{\capgreekfont}[1]{\mathrm{#1}}
\newcommand{\pcgraycomment}[1]{\textcolor{gray}{\pclinecomment{\text{#1}}}}

% Crypto functions
\newcommand{\Exp}[1]{\cryptofont{Exp}^{\expfont{#1}}}
\newcommand{\Adv}[1]{\cryptofont{Adv}^{\expfont{#1}}}
\newcommand{\Atk}[1]{\cryptofont{Atk}^{\expfont{#1}}}

% Math
\DeclareMathAlphabet\mathbfcal{OMS}{cmsy}{b}{n}
\newcommand{\dqed}{\hfill$\Diamond$}
\def\ceil(#1){\left\lceil #1 \right\rceil}
\def\floor(#1){\left\lfloor #1 \right\rfloor}
\newcommand{\goesto}{{\rightarrow}}
\newcommand{\Func}{\mathrm{Func}}
\newcommand{\cmark}{\ding{51}}
\newcommand{\xmark}{\ding{55}}

% - Sets
\newcommand{\setify}[1]{\procfont{set}\left(#1\right)}
\newcommand{\setlen}[1]{|#1|}
\newcommand{\multisetlen}[1]{\|#1\|}
\newcommand{\Z}{\mathbb{Z}}
\newcommand{\N}{\mathbb{N}}
\newcommand{\R}{\mathbb{R}}
\newcommand{\bits}{\{0,1\}}
\newcommand*\bigunion{\bigcup}
\newcommand*\bigintersection{\bigcap}
\newcommand*\union{\cup}
\newcommand{\multiunion}{\uplus}
\newcommand*\intersection{\cap}
\newcommand*\cross{\times}
\newcommand*\by{\cross}
\newcommand{\getsr}{\mathrel{\leftarrow\mkern-14mu\leftarrow}}
%\newcommand{\getsr}{\xleftarrow{\text{\tiny{\$}}}}
%\newcommand{\getsr}{{\:{\leftarrow{\hspace*{-3pt}\raisebox{.75pt}{$\scriptscriptstyle\$$}}}\:}}
\newcommand{\setop}[1]{\mathsf{set}(#1)} %^ \procfont
\def\str(#1){\procfont{set}\left(#1\right)}
\def\bydef{\stackrel{\rm def}{=}}
%\newcommand{\undefn}{\mathtt{undefined}}
\newcommand{\undefn}{\bot}
\newcommand{\univ}{\setfont{U}}
\newcommand{\queries}{\setfont{Q}}
\newcommand{\results}{\setfont{R}}
\newcommand{\mutants}{\setfont{M}}
\newcommand{\keys}{\setfont{K}}
\newcommand{\col}{\setfont{S}}
\newcommand{\zeros}{\mathrm{zeros}}
\newcommand{\setA}{\setfont{A}}
\newcommand{\setB}{\setfont{B}}
\newcommand{\setC}{\setfont{C}}
\newcommand{\setD}{\setfont{D}}
\newcommand{\setI}{\setfont{I}}
\newcommand{\setK}{\setfont{K}}
\newcommand{\setM}{\setfont{M}}
\newcommand{\setN}{\setfont{N}}
\newcommand{\setP}{\setfont{P}}
\newcommand{\setQ}{\setfont{Q}}
\newcommand{\setR}{\setfont{R}}
\newcommand{\setS}{\setfont{S}}
\newcommand{\setT}{\setfont{T}}
\newcommand{\setU}{\setfont{U}}
\newcommand{\setV}{\setfont{V}}
\newcommand{\setX}{\setfont{X}}
\newcommand{\setY}{\setfont{Y}}
\newcommand{\setW}{\setfont{W}}
\newcommand{\setZ}{\setfont{Z}}

% Vectors
\newcommand{\xx}{\vectorfont{x}}
%\newcommand{\vv}{\vectorfont{v}}
\newcommand{\REVO}{\mathbf{Reveal}}
\newcommand{\diffplus}[1]{\fbox{#1}}
\newcommand{\diffplusbox}[1]{\fbox{\parbox{\dimexpr\textwidth-2\fboxsep-2\fboxrule\relax}{#1}}}
\newcommand{\diffminus}[1]{\colorbox{lightgray}{#1}}
\newcommand{\diffminusbox}[1]{\colorbox{lightgray}{\parbox{\dimexpr\textwidth-2\fboxsep-2\fboxrule\relax}{#1}}}
\newcommand{\hh}{\vectorfont{h}}
\newcommand{\fff}{\schemefont{Fn}}
\newcommand{\Rnd}{\schemefont{Rand}}
\newcommand{\Repx}{\Rep1}
\newcommand{\Qryx}{\Qry1}
\newcommand{\Upx}{\Up1}
\newcommand{\Ans}{\varfont{Ans}}
\newcommand{\setE}{\mathcal{E}}
\newcommand{\Resp}{\varfont{Resp}}
%\def\ticks(#1,#2){\procfont{T}_{\hspace*{-1.5pt}#1}({#2})}
\newcommand{\ticks}{\mathsf{T}}
\newcommand{\highlighto}[1]{\colorbox{lightgray}{$\scriptstyle #1$}}
\newcommand{\highlight}[1]{\colorbox{lightgray}{$\displaystyle #1$}}

% - String operations
\newcommand{\emptystr}{\varepsilon}
\newcommand{\cat}{\, \| \,}
\newcommand{\concat}{\cat}
\def\str(#1){\langle #1 \rangle}
\def\substr(#1,#2,#3){#1[#2\mbox{\,:\,}#3]}
\def\toint{\procfont{int}}
\def\tostr{\procfont{str}}
\def\byte(#1){[#1]}
\newcommand*{\st}{~|~}
\newcommand{\size}[1]{\left|#1\right|}
\newcommand{\mb}{\ell}
\newcommand{\cb}{\ell}
\newcommand{\m}{\{1, 2, \dots, m\}}
\newcommand{\trunc}{\text{trunc}}


% - Boolean operators
\newcommand*\AND{\wedge}
\newcommand*\OR{\vee}
\newcommand*\NOT{\neg}
\newcommand*\IMPLIES{\implies}
\newcommand*\XOR{\mathbin{\oplus}}
\newcommand*\xor{\XOR}
\newcommand*{\bigor}{\bigvee}

% - Asymptotics
\newcommand{\negl}{\procfont{negl}}
\newcommand{\poly}{\procfont{poly}}

% - Probablity
\newcommand{\E}{\mathrm{E}}
\newcommand{\Prob}[1]{\Pr\hspace{-1pt}\left[\,#1\,\right]}
\newcommand{\prob}[1]{\mathrm{Pr}\left[#1\right]}
\newcommand{\given}{\mid}

% Games
\newcommand{\halt}{\bot}
\newcommand{\game}{\cryptofont{G}}
%\newcommand{\G}{\game}
% \newcommand{\foreach}[3]{$\text{for }#1 \gets #2\text{ to }#3\text{ do}$}
\newcommand{\tab}{\hspace*{10pt}}
\newcommand{\outputs}{=}
\newcommand{\sets}{\,\cryptofont{sets}\,}
\newcommand{\bad}{\varfont{bad}}
\newcommand{\true}{1}
\newcommand{\false}{0}
\newcommand{\invalid}{\bot}
\newcommand{\exception}{\invalid}
\newcommand{\experimentv}[1]{\underline{#1}}
\newcommand{\oraclev}[1]{\underline{{oracle} #1}:}
\newcommand{\adversaryv}[1]{\underline{{adv.} #1}:}
\newcommand{\algorithmv}[1]{\underline{{alg.} #1}:}

% Notions
\newcommand{\errep}{\notionfont{ERR\mbox{-}Pub}}
\newcommand{\erreps}{\notionfont{ERR\mbox{-}Priv}}
\newcommand{\indrep}{\notionfont{IND\mbox{-}UP}}
\newcommand{\indrepr}{\notionfont{IND\mbox{-}UPR}}
\newcommand{\prf}{\notionfont{PRF}}
\newcommand{\ssrep}{\notionfont{SS\mbox{-}REP}}
\def\ssrepX#1{\mbox{\ssrep-#1}}
\newcommand{\owrep}{\notionfont{OW\mbox{-}REP}}
\newcommand{\errepone}{\notionfont{ER\mbox{-}REP1}}
\def\indrepX#1{\indrep\mbox{-}#1}
\def\indreprX#1{\indrepr\mbox{-}#1}
\def\prfX#1{\prf\mbox{-}#1}

% Adversaries
\newcommand{\advA}{\adversaryfont{A}}
\newcommand{\advB}{\adversaryfont{B}}
\newcommand{\advC}{\adversaryfont{C}}
\newcommand{\dist}{\adversaryfont{D}}

% Variables
\newcommand{\err}{\varfont{err}}
\newcommand{\ct}{\varfont{ct}}
\newcommand{\salt}{Z}
\newcommand{\cnt}{\textup{cnt}}
\newcommand{\fp}{\textup{fp}}

% Oracles
\newcommand{\REPO}{\oraclefont{Rep}}
\newcommand{\UPO}{\oraclefont{Up}}
\newcommand{\QRYO}{\oraclefont{Qry}}
\newcommand{\PRFO}{\oraclefont{F}}
\newcommand{\INITO}{\oraclefont{Init}}
\newcommand{\HASHO}{\oraclefont{Hash}}
%\newcommand{\INTO}{\oraclefont{Int}}

% Structures
\newcommand{\ed}{(\epsilon,\delta)}
\newcommand{\struct}{\capgreekfont{\Pi}}
\newcommand{\Init}{\schemefont{Init}}
\newcommand{\Up}{\schemefont{Up}}
\newcommand{\Qry}{\schemefont{Qry}}
\newcommand{\Rep}{\schemefont{Rep}}
\newcommand{\qry}{\procfont{qry}}
\newcommand{\qryin}{\procfont{qry}_{\text{in}}}
\newcommand{\qryout}{\procfont{qry}_{\text{out}}}
\newcommand{\lk}{\procfont{lk}}
\newcommand{\up}{\procfont{up}}
%\newcommand{\pub}{\procfont{pub}}
\newcommand{\pub}{\procfont{repr}}
\newcommand{\repr}{\procfont{repr}}
\newcommand{\pubhash}{\procfont{hashFuncs}}
\newcommand{\param}{\procfont{par}}
\newcommand{\key}{K}
\newcommand{\pp}{\mathit{pp}}
\renewcommand{\sp}{\mathit{sp}}
\newcommand{\ky}{K}
\newcommand{\res}{a}
\newcommand{\elts}{\setfont{X}} % TODO What does "elts" mean?

% Constructions
\newcommand{\BF}{\schemefont{BF}}
\newcommand{\SBF}{\schemefont{SBF}}
\newcommand{\KBF}{\schemefont{KBF}}
\newcommand{\SKBF}{\schemefont{KBF}} % FIXME deprecate
\newcommand{\PRLBF}{\schemefont{PPRL}}
\newcommand{\DICT}{\schemefont{DICT}}
%\newcommand{\bloom}{\schemefont{Bloom}}
\newcommand{\bff}{\mathrm{bf}}
\newcommand{\saltybloom}{\mathrm{sbf}}
\newcommand{\prfbloom}{\mathrm{kbf}}
\newcommand{\sketch}{\schemefont{Sketch}}
\newcommand{\CMS}{\schemefont{CMS}}
\newcommand{\HK}{\schemefont{HK}}
\newcommand{\CK}{\schemefont{CK}}
\newcommand{\SCMS}{\schemefont{SCMS}}
\newcommand{\KCMS}{\schemefont{KCMS}}
\newcommand{\countbloom}{\schemefont{Count}}
\newcommand{\dict}{\mathrm{dict}}
\newcommand{\hashbf}{\schemefont{2Hash}}
\newcommand{\hash}{\schemefont{Hash}}
\newcommand{\hashlin}{\schemefont{2Hash}}
\newcommand{\tinyhash}{\schemefont{Tiny}}
\newcommand{\kbf}{\schemefont{KBF}}

% Misc.
\def\v.#1{\boldsymbol{#1}}
\newcommand{\bmap}{{B}}
\newcommand{\cmap}{{B}}
\newcommand{\hw}{{w}}
\newcommand{\id}{\schemefont{id}}

\newcommand{\emptystring}{\varepsilon}

\newcommand{\streamvar}[1]{\vec{#1}}
\newcommand{\set}[1]{\mathcal{#1}}

\newcommand{\myenddef}{\hfill$\blacklozenge$}
\newcommand{\ExpErrFe}[3]{\Atk{\text{err-fe}[#3]}_{#1}(#2)}
\newcommand{\rvcover}{\mathsf{Cover}}
\newcommand{\rvcoverx}[2]{\mathsf{Cover}^{#1}_{#2}}
\newcommand{\rverr}{\mathsf{Err}}
\newcommand{\figref}[1]{Figure~\ref{#1}}
\newcommand{\encode}[1]{\langle #1 \rangle}

% Redis things
\newcommand{\ef}{\mathit{ef}}
\newcommand{\subf}{\mathit{subf}}
\newcommand{\pref}{\mathit{pref}}
\newcommand{\unique}{\mathit{unique}}
\newcommand{\murmur}{\mathit{MurmurHash64A}}
\newcommand{\murmurtwo}{\mathit{MurmurHash2}}
\newcommand{\slot}{\mathit{slot}}
\newcommand{\fmax}{\mathit{fmax}}
\newcommand{\bs}{\mathit{bs}}
\renewcommand{\pp}{\mathit{pp}}


%% syntax
\newcommand{\setupS}{\textsf{setup}}
\newcommand{\upS}{\textsf{up}}
\newcommand{\insS}{\textsf{ins}}
\newcommand{\qryS}{\textsf{qry}}
\newcommand{\delS}{\textsf{del}}
\newcommand{\infoS}{\textsf{info}}
\newcommand{\listS}{\textsf{list}}
\newcommand{\patchS}{\textsf{patch}}
\newcommand{\diffS}{\textsf{diff}}
\newcommand{\somealgS}{\textsf{alg}}
\newcommand{\nai}{\text{NAI}}
\newcommand{\naigen}{\text{\nai-gen}}
\newcommand{\statdist}{SD}
\newcommand{\image}{\text{Im}}
\newcommand{\upO}{\textbf{Up}}
\newcommand{\insO}{\textbf{Ins}}
\newcommand{\qryO}{\textbf{Qry}}


\newcommand{\oCuckoo}{\textsf{originalCF}}
\newcommand{\rCuckoo}{\textsf{CF}}
\newcommand{\oBloom}{\textsf{originalBF}}
\newcommand{\rBloom}{\textsf{BF}}
\newcommand{\rCMS}{\textsf{CMS}}
\newcommand{\rTK}{\textsf{TK}}
\newcommand{\seed}{\mathit{seed}} 
\newcommand{\unset}{\mathit{unset}} 
\newcommand{\unused}{\mathit{unused}} 
\newcommand{\bpe}{\mathit{bpe}} 
\newcommand{\const}{\mathit{const}} 

\newcommand{\load}{{\textsc{load}}}
\newcommand{\save}{{\textsc{save}}}
%%correctness error vector.
\newcommand{\levels}{{\textsf{levels}}}
\newcommand{\correrrs}{\textsc{corr.errors}}
\newcommand{\trans}{\textsf{trans}}
\newcommand{\fplist}{{\textsf{fp.list}}}
\newcommand{\lost}{\mathsf{lost}} 
\newcommand{\win}{\mathsf{win}} 
%\newcommand{\lid}{{\textsf{l}_{id}}}
%\newcommand{\fp}{{\textit{Fp}_{m,k}}}
\newcommand{\pmk}{{\textit{P}_{m,k}}}
\newcommand{\pfpsym}{P_{\Pi,pp}}
\newcommand{\pfp}[1]{\pfpsym(FP\,|\,{#1})}
\newcommand{\opfp}[1]{\overline{\pfpsym}(FP \mid {#1})}
%\newcommand{\pufp}[1]{\pfpsym(UFP\,|\,{#1})}
\newcommand{\pufp}[1]{\pfpsym^*(FP\,|\,{#1})}
\newcommand{\opufp}[1]{\overline{\pfpsym}(UFP \mid {#1})}
\newcommand{\pifp}[1]{\pfpsym(IF\,|\,{#1})} %insertion failure
\newcommand{\oifp}[1]{\overline{\pfpsym}(IF \mid {#1})} %insertion failure 
%\newcommand{\puifp}[1]{\pfpsym(UIF\,|\,{#1})} %insertion failure universal 
\newcommand{\puifp}[1]{\pfpsym^*(IF\,|\,{#1})} %insertion failure universal 
\newcommand{\ouifp}[1]{\overline{\pfpsym}(UIF \mid {#1})} %insertion failure universal 
\newcommand{\pfn}{P^{FN}_{pp}}
\newcommand{\pmkreal}{\overline{\textit{P}_{m,k}}}
\newcommand{\fpreal}{\overline{\fp}}
\newcommand{\Load}[1]{\text{load}(\ensuremath{#1})}
\newcommand{\Keys}[1]{\text{keys}(\ensuremath{#1})}
\newcommand{\Values}[1]{\text{vals}(\ensuremath{#1})}

% skipping ds

\newcommand{\heads}{\mathit{heads}}
\newcommand{\tails}{\mathit{tails}}
\newcommand{\pwin}{p_{\mathrm{win}}}
\newcommand{\plose}{p_{\mathrm{lose}}}
\newcommand{\ploss}{\plose}

\newcommand{\ins}{\procfont{ins}}
\newcommand{\kwnew}{\procfont{new}\xspace}
\newcommand{\del}{\procfont{del}}
\newcommand{\llst}{\mathsf{L}}
\newcommand{\ttree}{\mathsf{T}}
% \newcommand{\nxt}{\mathsf{next}}
\newcommand{\node}{\mathsf{node}}
\newcommand{\nlll}{\mathsf{null}}
\newcommand{\hdr}{\mathsf{header}}
\newcommand{\rt}{\mathsf{root}}
\newcommand{\tmp}{\mathsf{tmp}}
\newcommand{\lvl}{\mathsf{level}}
%\newcommand{\pub}{\procfont{pub}}
\newcommand{\answ}{\procfont{answ}}

\newcommand{\com}{\procfont{com}}
\newcommand{\Hash}{\procfont{Hash}}
\newcommand{\paren}{\procfont{par}}
\newcommand{\nparen}{\procfont{npar}}
\newcommand{\keyacc}{\procfont{key}}
\newcommand{\valueacc}{\procfont{value}}
\newcommand{\delacc}{\procfont{del}}
%\newcommand{\pp}{\mathit{pp}}
\newcommand{\vp}{\mathit{v}}
\renewcommand{\sp}{\mathit{sp}}

% Constructions
\newcommand{\SL}{\schemefont{SL}}
\newcommand{\TR}{\schemefont{TR}}
\newcommand{\DSL}{\schemefont{DSL}}
\newcommand{\KDSL}{\schemefont{KDSL}}
\newcommand{\DSSL}{\schemefont{DSSL}}

%Variables
\newcommand{\dobj}{\varfont{D}} %data object
\newcommand{\prop}{\ensuremath{\phi}}
\newcommand{\elem}{\varfont{d}} %data element
\newcommand{\dataobjects}{\setfont{D}}

%%%%%%%%%%%%%%%%%%%%%%%%%%%%%%%%%%%%%%%%%%%%%%%%%%%%%%%%%%%%%%%%%%%
%%%                     User Configuration commands
%%%%%%%%%%%%%%%%%%%%%%%%%%%%%%%%%%%%%%%%%%%%%%%%%%%%%%%%%%%%%%%%%%%%%%%%%%%%%%%%

%% Uncomment the relevant line below if you have tables or figures.
\haveTablestrue%        Uncomment this if you have tables in your thesis.
\haveFigurestrue%       Uncomment this if you have figures in your thesis.
%\haveObjectstrue%       Uncomment this if you have Objects in your thesis. This is almost certainly not the case however.

%%%%%%%%%%%%%%%%%%%%%%%%%%%%%%%%%%%%%%%%%%%%%%%%%%%%%%%%%%%%%%%%%%%%%%%%%%%%%%%%
%%% Below are the commands to set the degree type, department, graduation time, and chair. 
%       Most of these are self explanatory. 
%       Note: The \chair command takes an optional argument for a cochair. 
%           So if John was your chair and Jacob was a cochair, you would use \chair[Jacob]{John}.
%           If John was your chair and you had no cochair, you can simply use \chair{John}.
%%%%%%%%%%%%%%%%%%%%%%%%%%%%%%%%%%%%%%%%%%%%%%%%%%%%%%%%%%%%%%%%%%%%%%%%%%%%%%%%

\title{Data Structures in Adversarial Environments}%  Put your title here.

\degreeType{Doctor of Philosophy}%   Official name of your degree; eg "Doctor of Philosophy".
\major{Computer Science}%                    Your official Department
\author{Sam A. Markelon}%                  Your Name
\thesisType{Dissertation}%              Dissertation (PhD) or Thesis (Masters)
\degreeYear{2025}%                      Intended graduation year (not the year you submit the thesis)
\degreeMonth{August}%                   Month of graduation should be May, August, or December.
\chair{Vincent Bindschaedler}%                   Chair and Cochair (see comment block above).

%%%%%%%%%%%%%%%%%%%%%%%%%%%%%%%%%%%%%%%%%%%%%%%%%%%%%%%%%%%%%%%%%%%%%%%%%%%%%%%%
%%% For each of the following, type in the name of the file that contains each section. 
% They are assumed to be tex files, but if they aren't the command takes an optional argument for the extension.
%So, you could load dedication.tex as your dedication file using \setDedicationFile{dedication}
% You could load dedication.txt instead with \setDedicationFile[txt]{dedication}.

% NOTE: For some compilers they may or may not add a .tex to the end of the file automatically.
% If you get a "couldn't find dedication.tex.tex" type error, try the command with an empty optional argument,
%e.g. \setDedicationFile[]{dedication}
%%%
%%%%%%%%%%%%%%%%%%%%%%%%%%%%%%%%%%%%%%%%%%%%%%%%%%%%%%%%%%%%%%%%%%%%%%%%%%%%%%%%

%%% These are REQUIRED sections; easiest to do via these commands.

\setDedicationFile{required/dedicationFile}%                 Dedication Page
\setAcknowledgementsFile{required/acknowledgementsFile}%     Acknowledgements Page
\setAbstractFile{required/abstractFile}%                     Abstract Page (This should only include the abstract itself)
\setReferenceFile{required/refs}{ieeetr}%         References. First argument is your bibtex source file
%                                                       the second argument is your bibtex style file.

\setBiographicalFile{required/biographyFile}%                Biography file of the Author (you).

%%% These are NOT required, so only use them if you actually need/have them.

%\setAbbreviationsFile{abbreviations}%           Abbreviations Page
%\setAppendixFile{appendix}%                     Appendix Content; hyperlinking might be weird.
%\multipleAppendixtrue%                          Uncomment this if you have more than one appendix, 
%                                                   comment it if you have only one appendix.


%%%%%%%                     End of File Assignment
%%%%%%%%%%%%%%%%%%%%%%%%%%%%%%%%%%%%%%%%%%%%%%%%%%%%%%%%%%%%%%%%%%%%%%%%%%%%%%%%

\begin{document}

%%%% Here you just need to include/input your actual work. 
%       The above files (dedication, acknowledgement, titlepage, etc etc) will all be added for you 
%       using the files you assigned above. 
%       If you want to input the above files manually you can comment out the \setFILE command above 
%       and use \input or \include here. Generally you want to use \include to get your pagebreak.
%       NOTE: If you input manually you will have to do some/all the formatting manually.

\chapter{Introduction}

\authorRemark{The following needs to be re-organized and re-written.}

Data structures define representations of possibly dynamic (multi)sets, along with the operations that can be performed on this representation of the underlying data. Efficient data structures are crucial for designing efficient algorithms~\cite{clrs}. The development and analysis of data structures has largely been driven by operational concerns, e.g., efficiency, ease of deployment, support for broad application. Security concerns, on the other hand, have traditionally been afterthoughts (at best). However, recent research has highlighted that many widely-used data structures do not behave as expected when in the presence of adversaries that have the ability to control the data they represent. Further, complex protocols that have sophisticated security goals are increasingly using a variety of bespoke data structures as fundamental components of their design. Therefore, it is wise to begin applying the provable security paradigm to data structures themselves. 

For instance, consider probabilistic data structures (PDS). They provide compact (sublinear) representations of potentially large collections of data and support a small set of queries that can be answered efficiently. Prime examples of such structures include the Bloom filter~\cite{bloom1970space}, the HyperLogLog~\cite{flajolet2007hyperloglog}, and the Count-min Sketch~\cite{cormode2005improved}. These space and (by extension) performance gains come at the expense of correctness.  Specifically, PDS query responses are computed over the compact representation of the data, as opposed to the complete data.  As a result, PDS query responses are only guaranteed to be \emph{close} to the true answer with \emph{large} probability, where \emph{close} and \emph{large} are typically functions of structure parameters (e.g., the representation size) and properties of the data. These guarantees are stated under the assumption that the data and the internal randomness of the PDS are independent.  Informally, this is tantamount to assuming  that the entire collection of data is (or can be) determined \emph{before} any random choices are made by the PDS.  For many PDS, this means before some number of hash functions are sampled, as the PDS operates deterministically after that. Recent works have begun to explore the impact on correctness guarantees for data that \emph{may} depend upon the internal randomness of the structure, and the initial findings are negative. 

Moreover, consider the class of data structures we refer to as \emph{skipping data structures}. Unlike the probabilistic data structures we discussed earlier, this class of structure are not space-efficient (compact) and, in turn, give exact answers to queries. These data structures (e.g., hash tables, skip lists, and treaps) offer fast average-case runtime of their operations, but have worst-case runtime that is  poor. They achieve this by using some form of randomness to determine the representation of the underlying data collection.  Recent research shows that adaptive adversaries are able to force worst-case runtime for these structure, often demonstrated by attacks on real-world systems. Therefore, instead of focusing on adversarial correctness as in the PDS section, we focus on preserving the expected run time of these structures with large probability in the presence of an adversary.
\chapter{Background}\label{chap:background}

\section{Notation}

\paragraph{Bitstring and Set Operations}

Let $\bits^*$ denote the set of bitstrings and let~$\emptystr$ denote the empty
string. Let $X \cat Y$ denote the concatenation of bitstrings~$X$ and~$Y$.  When~$\col$ is an abstract data-object (e.g., a (multi)set, a list) and~$e$ is an object that can be appended (in some understood fashion) to~$\col$, we overload the $\cat$ operator and write $\col \cat e$.

Let $x \getsr \setX$ denote sampling~$x$ from a set~$\setX$ according to the distribution associated with~$\setX$; if~$\setX$ is finite and the distribution is unspecified, then it is uniform. Moreover, we denote by $\mathsf{U}(S)$ the uniform distribution on the (finite or uncountable) set $S\neq\emptyset$, and by $\mathsf{G}(p)$ be the geometric distribution for success probability $p$.

Let $[i..j]$ denote the set of integers $\{i, \ldots, j\}$; if $i > j$, then define $[i..j] = \emptyset$. For all $m \geq 2$, let $[m] = \{1,2,\ldots,m\}$.

Let $\mathcal{A}$ and $\mathcal{B}$ be sets. We take $\mathcal{A} \cup \mathcal{B}$ to be the union of the sets, $\mathcal{A} \cap \mathcal{B}$ to be the intersection of the sets, and $\mathcal{A} \setminus \mathcal{B}$ to be set-theoretic difference of $\mathcal{A}$ and $\mathcal{B}$.

\paragraph{Functions}

Let $\Func(\setX,\setY)$ denote the set of functions $f:\setX\to\setY$. For every function~$f: \setX \to \setY$, define $\id^f: \{\emptystr\} \times \setX \to \setY$ so that $\id^f(\emptystr, x) = f(x)$ for all $x$ in the domain of~$f$. This allows us to use unkeyed hash functions $H$ in situations where, syntactically, a function is required to take a key along with its input. 

\paragraph{Arrays and Tuples}

We use the distinguished symbol~$\star$ to mean that a variable is uninitialized. By $[\text{item}] \times \ell$ for~$\ell \,{\in}\, N$ we mean a vector of $\ell$ replicas of $\text{item}$. We use $\zeros(m)$ denote a function that returns an $m$-length array of 0s and, likewise, $\zeros(k,m)$ to denote a function that returns an $k \times m$ array of 0s.  We index into arrays (and tuples) using $[\cdot]$ notation; in particular, if $R$ is a function returning a $k$-tuple, we write $R(x)[i]$ to mean the $i$-th element/coordinate of $R(x)$.  If~$X{=}\,(x_1,x_2,\ldots,x_t)$ is a tuple and $\set{S}$ is a set, we overload standard set operators (e.g., $X \,{\subseteq}\, \set{S}$) treating the tuple as a set; if we write $X \setminus \set{S}$, we mean to remove all instances of the elements of~$\set{S}$ from the tuple~$X$, returning a tuple~$X'$ that is ``collapsed'' by removing any now-empty positions.

\section{A Syntax for Data Structures}\label{subsec:syntax}

We present a syntax for data structures first provided by~\cite{clayton2019}. While originally used to describe a variety of probabilistic data structures, the syntax is appropriately general. A syntactic formalization of data structures in this way not only allows us to elegantly describe numerous data structures, but also craft security definitions that are directly related to the operations the data structure allows. We will do exactly this throughout the rest of this work.

We start by fixing three non-empty sets~$\set{D},\set{R},\set{K}$ of \emph{data objects}, \emph{responses} and \emph{keys}, respectively.  Let $\mathcal{Q}\subseteq \Func(\mathcal{D},\mathcal{R})$ be a set of allowed \emph{queries}, and let $\mathcal{U} \subseteq \Func(\mathcal{D},\mathcal{D})$ be a set of allowed data-object \emph{updates}.  A {\em data structure} is a tuple $\Pi =
(\Rep,\Qry,\Up)$, where:

\begin{itemize}[leftmargin=.2in]
  \item $\Rep\colon \keys \times \mathcal{D} \to \{0,1\}^* \cup \{\bot\}$ is a
  (possibly) randomized {\em representation algorithm}, taking as input a key $\key \in
  \keys$ and data object $\col \in \mathcal{D}$, and outputting the
  representation $\pub \in \{0,1\}^*$ of $D$, or $\bot$ in the case of a
  failure. We write this as $\pub \gets \Rep_\key(\col)$.
%
  \item $\Qry\colon \keys \times \{0,1\}^* \times \mathcal{Q} \to \mathcal{R} \cup \{\bot\}$
  is a deterministic {\em query-evaluation algorithm}, taking as input $\key \in
  \keys$, $\pub \in \{0,1\}^*$, and $\qry \in \mathcal{Q}$, and outputting an
  answer $a \in \mathcal{R}$, or $\bot$ in the case of a failure. We write this as $a \gets \Qry_\key(\pub,\qry)$.
%
  \item $\Up\colon \keys \times \{0,1\}^* \times \mathcal{U} \to \{0,1\}^* \cup
  \{\bot\}$ is a (possibly) randomized {\em update algorithm}, taking as input $\key \in
  \keys$, $\pub \in \{0,1\}^*$, and $\up \in \mathcal{U}$, and outputting an
  updated representation $\pub'$, or $\bot$ in the case of a failure. We write
  this as $\pub' \gets \Up_\key(\pub,\up)$.
\end{itemize}

Allowing each of the algorithms to take a key~$K$ permits one to separate (for some
security notion) any secret randomness used across data structure operations,
from per-operation randomness (e.g., generation of a salt). Note that this syntax admits the
common case of \emph{unkeyed} data structures, by setting
$\keys=\{\emptystring\}$. Moreover, we can set $\keys=\mathsf{priv}$ to be a private key and allow the corresponding public key $\mathsf{pub}$ to be a public parameter in the case the data structure relies on asymmetric cryptographic primitives.  

Both~$\Rep$ and the~$\Up$ algorithm can be viewed (informally) as mapping data
objects to representations ---~explicitly so in the case of~$\Rep$, and
implicitly in the case of~$\Up$~--- so we allow~$\Up$ to make per-call random
choices, too. 

Note that $\Up$ takes a function operating on data objects as an argument, even
though $\Up$ itself operates on \emph{representations} of data objects. This is
intentional, to match the way these data structures generally operate.
In a data structure representing a set or multiset, we often think of performing
operations such as `insert $x$' or `delete $y$'. When the set or multiset is not
being stored, but instead modeled via a representation, the representation must
transform these operations into operations on the actual data structure it is
using for storage. This is common for operation on probabilistic data structures. 

We also note that the query algorithm $\Qry$ is deterministic, which reflects the overwhelming majority of data structures in practice. Allowing~$\Qry$ to be randomized would allow for a greater degree of syntactic expressiveness, particularly for some data structures that provide privacy guarantees. However, it can make it more difficult to craft correctness properties in that it may be difficult to discern the errors caused by an adaptive adversary versus ``intended'' error arising from the randomized query algorithm. Care must be taken when both designing structures and defining security properties to ensure issues do not arise from this.  

\section{Streaming Data}

A \emph{stream} data-object~$\streamvar{S} = e_1,e_2,\ldots$ is a finite sequence of elements $e_i \in \set{U}$ for some universe~$\set{U}$.  
The elements of a stream are not necessarily distinct, and the (stream) frequency of some $x \in \set{U}$ is $|\{i: e_i=x \}|$.  
From the perspective of the PDS, the stream is presented one element at a time, with no buffering or ``look ahead".  
That is, processing of a stream is performed in order, and the processing of $e_i$ is completed before the processing of $e_{i+1}$ may begin; once~$e_i$ has been processed, it cannot be revisited.

\section{Related Works}

We now provide a summary of related work for the classes of data structures that we focus on in this dissertation.

\subsection{Compact Probabilistic Data Structures and Compact Frequency Estimators}

\subsubsection{Approximate Set Membership Data Structures}

The first works to explore CPDS in a provable security style focused on the Bloom filter~\cite{bloom1970space}. The Bloom filter admits approximate set-membership queries. The structure is widely used in many computing contexts, such as databases~\cite{chang2008bigtable}, networking~\cite{broder2004network}, distributed systems~\cite{tarkoma2011theory}, and search~\cite{goodwin2017bitfunnel}. 

Naor and Yogev were the first to consider settings in which inputs and queries may be chosen by an adaptive adversary and formally investigate attacks that can occur in such a setting~\cite{naor2015bloom}. Their results show that adversaries can find queries that are guaranteed to be false positive for a given instantiation of a filter and data collection. They formalized a notion of adversarial correctness for a modified Bloom filter structure of their own construction and provide a correctness bound for it.  Clayton et al.~\cite{clayton2019} extend this work by considering stronger adversaries. They allow for the adversary to insert elements into the structure after the adversary has started to issue queries -- that is, they consider a fully mutable setting.  They find that the basic Bloom filter is vulnerable to adversarial manipulation, which can increase false positives to nearly $100\%$. To secure it, they recommend adding a unique salt in an immutable setup, or using a private representation, keyed hash functions, and insertion thresholds in a mutable setting. Further, they formalize a notion of adversarial correctness that extends past only Bloom filters, also concretely analyzing the counting filter~\cite{fan2000summary} and the Count-min sketch~\cite{cormode2005improved}. Filić et al.~\cite{FPUV22,filic2025deletions} further analyze the adversarial correctness of Bloom filters and Cuckoo filters (another approximate membership data structure) in a simulation style security notion. They reach similar conclusions to~\cite{clayton2019}. 

\subsubsection{HyperLogLog}

The HyperLogLog (HLL)~\cite{flajolet2007hyperloglog} is a CPDS that provides a compact representation of a set and can accurately approximate the number of distinct elements in the set (i.e., the set's cardinality). Patterson and Raynal~\cite{PatersonR22} provide a provable security treatment of the HLL. They first present attacks which exploit the use of fixed and publicly computable hash function in the HLL to cause large cardinality estimate errors. They then show that by switching these hash functions for a secretly keyed primitive that (even in the setting where an adversary has complete access to the internal state of the structure) the structure remains secure in terms of conserving the non-adversarial correctness guarantees of the structure. Prior to this, Revirigeo and Ting provide attacks against the HLL in a model where the adversary has access to a ``shadow'' device that mirrors the structure that is being attacked~\cite{reviriego2020security}. Patterson and Raynal point out this setting in unrealistic, but nonetheless improve the attack in this model. 


\subsubsection{Compact Frequency Estimators}

Recall that compact frequency estimators are a class of CPDS that compactly represent a collection of streaming data (usually modeled as a multiset), and provide approximately correct frequency estimates (that is, the number of times any particular element has appeared in the stream). Alternately, compact frequency estimators can be viewed as providing a compact representation of the frequency distribution of a particular data stream. 

As previously stated, Clayton et al. were the first to examine compact frequency estimators from a provable security perspective~\cite{clayton2019}. They specifically examined the Count-min sketch and presented attacks that could cause large frequency estimation error when the internals state of the structure or the hash functions used by the structure were made available to the adversary. They were able to prove security of the structure when the internal state of the structure is kept private and a secretly keyed primitive was used in place of the usual hash functions. However, their defined adversarial goal was very conservative. Any fixed amount of frequency estimation error was considered a win for the adversary, rather than an accumulated error that surpassed that of the non-adversarial correctness guarantee. Further, their construction relied on a thresholding technique, in which the structure would not accept any more updates after a bounded number of insertions -- something that in practice is unrealistic.  

\subsection{Probabilistic Skipping-Based Data Structures}

\subsubsection{Self-Balancing and Self-Organizing Data Structures}

Although PSDS share conceptual similarities with self-balancing and self-organizing data structures, they differ fundamentally in their guarantees and methodological approach. 
Notably, self-organizing data structures have been extensively analyzed under adversarial models where input sequences are deliberately constructed to degrade performance, whereas the corresponding analysis for PSDS against adaptive adversaries remains a significant open problem. Similarly, self-balancing data structures have been studied extensively under worst-case analyses that inherently account for adversarial strategies.

\emph{Self-organizing data structures}~\cite{albers2005self}, whether randomized or deterministic, dynamically adjust their internal ordering of elements to optimize performance based on a given (potentially adversarial) sequence of input requests. For instance, self-organizing lists may employ the move-to-front heuristic, where accessed elements are relocated to the front of the list, or the transpose method, where elements swap positions with their predecessors when accessed. Similarly, splay trees~\cite{sleator1985self} rotate frequently accessed nodes closer to the root to reduce future access times. This approach has been shown to be challenging in adaptive adversarial settings, with (randomized) self-organizing lists incurring a cost at least three times that of the optimal reordering strategy \cite{reingold1994randomized}. 

\emph{Self-balancing} data structures, such as Red-Black trees~\cite{bayer1972symmetric} and AVL trees~\cite{adel1962algorithm}, \emph{deterministically} ensure an upper-bound on node depth, thereby providing worst-case performance guarantees for search operations. This deterministic approach is also exemplified by the deterministic skip list~\cite{munro1992deterministic}, which enforces an optimal structure by carefully promoting inserted nodes and their neighborhoods to appropriate levels. While these structures guarantee bounded search path lengths (even in adversarial settings), they require complex re-balancing mechanisms. In steep contrast, PSDS, such as the treap~\cite{seidel1996randomized} and the original skip list~\cite{pugh}, offer comparable expected performance, achieved through simple, probabilistic updating mechanisms. This presents a clear trade-off: deterministic structures provide absolute performance guarantees at the cost of implementation complexity, while probabilistic alternatives offer simplicity, albeit, with only probabilistic guarantees. In this work, we investigate whether we can maintain the implementation simplicity of probabilistic data structures while preserving their performance guarantees even in adversarial settings.

\subsubsection{Complexity Attacks Against Probabilistic Skipping-Based Data Structures}

This section provides a concise overview of so-called \emph{complexity attacks} targeting PSDS. Previous research has identified clear vulnerabilities in hash tables and skip lists, but these works lack formal security analysis and rigorous proofs of security when potential mitigations are put forth. Hash tables have received the most attention, while skip lists have been addressed (to our knowledge) in only a single paper in this context. Further, to our knowledge, no prior work has examined complexity attacks against treaps. This absence is consistent with our finding that treaps possess inherent resistance to such attacks.

\paragraph{Hash Tables} Assuming a hash table's internal hash function has ``good'' collision-resistance properties, the amortized average-case complexity of insertions, deletions, and look-ups is~$O(1)$. For these efficiency reasons, hash tables are widely used in many applications such as implementing associative arrays~\cite{mehlhorn2008hash} and sets~\cite{blandy2021programming} in many programming languages, in cache systems~\cite{istvan2015hash}, as well as for database indexing~\cite{zobel2001memory}.

However, this average-case performance relies on a critical assumption: that the data inserted into a hash table is independent of the (potentially randomly selected) hash function used to map key-value pairs to buckets. This assumption fundamentally breaks down in adversarial scenarios where an attacker can deliberately craft insertions that exploit knowledge of the hash function or its outputs. Given the ubiquity of hash tables in modern computing systems, numerous researchers \cite{paxson1999bro, CrosbyW03, bar2007remote, eckhoff2009hash, klink2011efficient, aumasson2012hash,bottinelli2025hash} have investigated techniques to compromise the data structure, forcing operations to degrade from expected $O(1)$ to worst-case $O(n)$ time complexity, where $n$ represents the total number of elements in the structure. These adversarial approaches typically constitute complexity attacks that strategically engineer inputs causing multi-collisions -- deliberately exploiting hash function properties to force numerous distinct keys into identical buckets.

Crosby and Wallach~\cite{CrosbyW03} demonstrated denial-of-service attacks via complexity attacks in applications using hash tables, such as the Bro intrusion detection system~\cite{paxson1999bro}, by forcing collisions with weak, fixed hash functions. They suggested universal hashing~\cite{carter1977universal} as a mitigation, though without any formal guarantees. Klink and Walde~\cite{klink2011efficient} showed similar CPU exhaustion attacks on web servers (e.g., PHP, ASP.NET, Java), only using a single carefully crafted HTTP request. Aumasson et al.\cite{aumasson2012hash} further revealed vulnerabilities in hash tables using non-cryptographic hash functions (like MurmurHash and CityHash\cite{appleby2016smhasher}), proposing SipHash~\cite{aumasson2012hash} as a secure alternative -- which is widely adopted but lacks a holistic formal analysis as it comes to security of hash tables in adversarial settings. Complexity attacks have also been shown effective in causing denial-of-service against flow-monitoring systems~\cite{eckhoff2009hash}. Further, the use of salting was undermined by remote timing attacks~\cite{bar2007remote}. Recently, Bottinelli et al.~\cite{bottinelli2025hash} found nearly a third of QUIC implementations vulnerable to similar attacks. Despite these works and many proposed defenses, no formal framework exists for the provable security of (keyed) hash tables against adaptive adversaries. We address this gap by introducing the first rigorous security model for this setting, along with formal proofs establishing bounds on adversarial runtime degradation.

\paragraph{Skip Lists}
In the original skip list paper~\cite{pugh}, it is noted that it is imperative to keep the internal structure of the skip list hidden. Otherwise, adversarial users could observe the levels of individual elements and delete any element at a level greater than zero (the bottom layer). This would degenerate the structure to a simple linked list and force worst-case run time ($O(n)$) on subsequent operations after these deletions occur. 

Nussbaum and Segal~\cite{nussbaum2019skiplist} demonstrate that private internal structure alone fails to protect skip lists against this style of attack. They present a (remote) timing attack that correlates query response times with element heights, ultimately allowing adversaries to force all elements in the structure to the lowest level. Their adversarial model is notably limited: the adversary cannot access the internal skip list structure, the initial data collection is non-adversarially selected, and the original data collection must be preserved during the attack. While they propose a structure called the \emph{splay skip list} as a countermeasure, their solution lacks formal security analysis. Our work presents a significantly stronger adversarial model and provides a construction with formal security guarantees. We give an extensive commentary on~\cite{nussbaum2019skiplist} and vulnerabilities below. 

Nussbaum and Segal~\cite{nussbaum2019skiplist} show that keeping the internal structure of the skip list private is insufficient to protect against complexity attacks. We discuss their attack in more detail because it is instructive in light of how to model attacks and prove the properties of robust alternatives.
Nussbaum and Segal present a timing attack that allows an adversary to discover the levels at which specific elements reside through a series of queries and, in turn, correlate the time it takes to answer a query on a given element with the height of that element. After the heights of the elements are discovered, the simple deletion attack can be mounted. 

The specific attack they present includes several assumptions.

\begin{itemize}
    \item The size of the collection represented by the structure,~$n$, is known to the adversary,~$\advA$.
    \item Each node in the structure holds a unique value.
    \item The well-ordered universe~$\univ$ is known and is of size~$O(n)$.
    \item The runtime of the search algorithm in the structure is consistent. That is, a search for the same value will yield the same runtime each time the search is executed.
\end{itemize}

Further, their adversarial model is the following. 

\begin{itemize}
    \item $\advA$ is given a skip list containing some collection of data,~$D$ that was selected by some (non-adversarial) process. 
    \item The adversary, $\advA$ does not have access to the internal structure of the skip list at any point. $\advA$ can only interact with the structure through oracles that provide search, insertion, and deletion functionality to the structure that is under attack.
    \item After the completion of the attack,~$\advA$ is required to have altered the skip list it interacts with such that it contains the original~$D$ represented by the structure (before any adversarial interaction occurs) and the level that all (or nearly all) the elements reside at is the first.  
\end{itemize}

The attack in this setting works by first running the timing attack to discover the level at which the elements in the structure exist (and, on the first iteration, which elements from~$\univ$ are present in the structure). Then all elements with a level greater than zero (exist at high level than the initial later) are removed. This set of removed elements are reinserted. These steps are repeated until (nearly) all the elements in the structure reside at level zero and the original collection represented by the structure is conserved -- thereby, degrading the representation of this collection to (nearly) a flat singly-linked list. 

As a countermeasure, the splay skip list structure is presented~\cite{nussbaum2019skiplist}.  The approach is to swap the levels of certain elements during a search query, thereby preventing the adversary from discovering information about the level where any particular element resides (as they are not fixed). The structure is believed to prevent the timing attack from being effective, but no formal analysis of the security of the structure is given. 

We again note that the adversarial setting that is given in~\cite{nussbaum2019skiplist} is rather limited. It assumes the adversary does not have access to the internal structure of the skip list, nor the ability to control the initial collection of data the skip represents. Further, it requires the adversary to conserve the initial data collection~$D$ that the skip list represents before any adversarial interaction occurs. We present a much stronger adversarial model in our work and a construction that satisfies this definition.

The authors propose a new structure that is believed to prevent the timing attack they present; however, as previously stated, no formal security analysis is given. 
Indeed, the splay skip list is still vulnerable to attacks, as demonstrated by the following scenario. Consider a collection~$D$ of elements represented by a splay skip list, where a total order is defined on the universe in which~$D$ resides. Suppose there exists an element~\( d \) such that~\( x_1 \leq d \leq x_2 \) for every pair of elements~\( x_1, x_2 \in D \), where~$x_1 \neq x_2$. For a specific order, $x_1 \leq d_1 \leq x_2 \leq d_2 \leq \ldots$ for~$x_i \in D$ and~$d_i \notin D$, an adversary can exploit this by conducting search queries for the intermediary elements $d_i$. 

Unlike searches for elements~$x_i \in D$, which would trigger the splay mechanism, searches for these intermediary elements~$d_i \not\in D$ bypass the splay security mechanism. The runtimes required to (not) find these intermediate nodes, however, still uniquely determine the height of elements contained in~$D$.\footnote{Compared to searching for elements~$x_1,x_2,\ldots$ as described in the original attack, the runtimes for searching~$d_1,d_2,\ldots$ only change by a constant factor (one extra step to find that the~$d_i \not\in S$) .} After the discovery of the heights of the elements contained in~$D$, the trivial deletion attack could be carried out as before.
\chapter{Compact Frequency Estimators in Adversarial Environments}

TODO!
\chapter{Compact Probabilistic Data Structures in the Wild: A Security Analysis of Redis}\label{chap:redis}

%-------------------------------------------------------------------------------
As we have seen, compact probabilistic data structures are becoming ubiquitous in modern computing applications that deal with large amounts of data, especially when the data is presented as a stream. Many modern data warehousing and processing systems provide access to CPDS as part of their functionality.  A prominent example of such a system is Redis, a general purpose, in-memory database. Redis is integrated into general data analytics and computing platforms offered by AWS, Google Cloud, IBM Cloud, and Microsoft Azure, amongst others. Redis supports a variety of CPDS: HyperLogLog (HLL), Bloom filter, Cuckoo filter, t-digest, Top-K, and count-min sketch~\cite{redisPDS}. 
While Redis was mostly used as a cache in the past, it is now a fully general system, used by a companies like Adobe~\cite{Web:RedisForAdobe}, Microsoft~\cite{Web:RedisForMicrosoft}, Facebook~\cite{Web:RedisForFacebook} and Verizon~\cite{Web:RedisForVerizon} for a variety of purposes. These include security-related applications, such as traffic analysis and intrusion detection systems~\cite{Web:RedisForSiemens}.

As the functionality of Redis has broadened, so has its maturity with respect to security. Initially, the Redis developers stated that no security should be expected from Redis:
The Redis security model is: “it’s totally insecure to let untrusted clients access the system, please protect it from the outside world yourself”~\cite{antirez15}. In reality, users failed to comply with this~\cite{FiebigFP16}. Today, Redis has a number of security features, and has adopted a different model, with a protected mode as default, user authentication, use of TLS, and command block-listing amongst other features~\cite{redisSec}. Redis now also recognize security and performance in the face of adversarially-chosen inputs as being a valid concern, stating that ``an attacker might insert data into Redis that triggers pathological (worst case) algorithm complexity on data structures implemented inside Redis internals'' and then going on to discuss two potential issues, namely hash table exhaustion and worst-case sorting behavior triggered by crafted inputs~\cite{redisSec}. The first issue is prevented in Redis by using hash function seeding; the second issue is not currently addressed. However, Redis' consideration of malicious inputs does not seem to extend to their CPDS implementations.

Given its prominence in the marketplace and the many other systems that rely on it, we contend that the CPDS used in Redis are deserving of detailed analysis. Moreover, in view of the broad set of use cases for these CPDS, including those where adversarial interference is anticipated and would be damaging if successful, this analysis should be done in an adversarial setting. This approach follows a line of recent work on CPDS analysis~\cite{GerbetKL15,clayton2019,cardestprivacy,hllvuln,PatersonR22,markelon23}. In this paper, we make a comprehensive security analysis of the suite of CPDS provided by Redis, with a view to understanding how its constituent CPDS perform in adversarial settings. As argued in~\cite{cryptoeprint:2024/532}, we regard the observation, documentation, and analysis of such security phenomena ``in the wild'' as constituting scientific contributions in their own right.

Following prior work, we assume only that the adversary has access to the functionality provided by the CPDS (eg. via the presented API). The adversary's aim is then to subvert the main goal of the specific CPDS under study. 
We deliberately remain agnostic about precisely which application is running on top of Redis, since the relevant applications will change over time and are anyway largely proprietary. The real-world effects of a successful attack will vary across applications, but might include, for example, false statistical information being presented to users (in the case of frequency estimation), wrongly reporting the presence of certain data items in a cache (in the case of Bloom filters or Cuckoo filters) leading to performance degradation, or the evasion of network attack detection (in the case of cardinality estimation being used in network applications). 
Instead of making application-specific analyses, we focus on the core CPDS functionalities in Redis and how their goals can be subverted in general. Naturally, our analyses are specific to each of the different CPDS supported in Redis, and depend on various low-level implementation choices made by Redis. These choices lead us to develop novel attacks that are more powerful than the known generic attacks against the different CPDS in Redis.

Since HLL in Redis was already comprehensively studied in~\cite{PatersonR22}, we do not consider it further here. We note only that~\cite{PatersonR22} showed how to manipulate data input to Redis HLL to distort cardinality estimates in severe ways, in a variety of adversarial settings. The t-digest is a data structure first introduced in~\cite{dunning2021t}; it uses a k-means clustering technique~\cite{kodinariya2013review} to estimate percentiles over a collection of measurements. The structure is an outlier in the Redis CPDS suite as it does not work in the streaming setting, but necessitates the batching of data in memory, and it is not really probabilistic in the same sense as the other CPDS in Redis (in particular it does not employ the ``hash functions mapping to array positions" paradigm that the other CPDS in Redis use). For these reasons, we omit a security evaluation of t-digest (both in general and in the case of the Redis implementation).

This leads us to focus on the remaining four CPDS in Redis: Bloom filter, Cuckoo filter, Top-K, and count-min sketch. For each  CPDS, we discuss how the CPDS was originally described in the literature and lay out how the Redis implementation differs from this ``theoretical'' description. We then develop attacks for each of these four PDS, with the attacks in most cases exploiting specific features of the Redis implementations and being more efficient for this reason (simultaneously, we have to deal with the many oddities of the Redis codebase in our attacks). In total, we present 10 different attacks across the four CPDS. We compare our attacks with  known attacks for these CPDS from the literature. We also look at how the CPDS in Redis can be protected against attacks, drawing on existing literature that considers this question for CPDS more generally~\cite{clayton2019,FPUV22,PatersonR22,markelon23}. For the purposes of this work, we provide the structural descriptions and attacks for the compact frequency estimators implemented in Redis (Top-K and count-min sketch). Recall, that in~\Cref{chap:cfe} we saw that count-min sketch and HeavyKeeper (called Top-K in Redis) are susceptible to devastating attacks in even limited adversarial settings. In this chapter we demonstrate how these attacks are augmented by the specific implementation choices that Redis makes.Full details on the Bloom filter and Cuckoo filter are available in the full paper~\cite{cryptoeprint:2024/1312}.

Further, we notified Redis of our findings on 29.04.2024. The full version of our paper~\cite{cryptoeprint:2024/1312} is identical to the document we sent to Redis on 29.04.2024 aside from changes made in this subsection. We offered to engage in a coordinated approach to vulnerability disclosure and suggested a 90-day period before any public distribution of our research paper. Redis acknowledged our findings immediately and then gave a detailed response on 16.05.2024. In this response, Redis disputed the validity of analyzing Redis' CPDS in adversarial settings; naturally we disagree with their viewpoint. However, they also committed to consider changes to their implementation in future versions, including using random seeds instead of fixed seeds, considering alternative hash functions, and adding disclaimers to their documentation. They did not commit to a timeline for this consideration. They decided not to handle our disclosure as a ``Redis vulnerability".

%-------------------------------------------------------------------------------

%-------------------------------------------------------------------------------
\section{PDS in Redis}
We start by (re)introducing theC PDS that we consider in this chapter -- the count-min sketch and the Top-K (HeavyKeeper). We will describe their original specification, the probabilistic guarantees they provide, and give a detailed description of their Redis implementation. We depart from our use of the generic data structure's syntax of~\cite{clayton2019}, instead using an ad-hoc syntax that better matches how the structures are defined and implemented in Redis. 

\subsection{Count-min Sketches}
\label{sec:cms-intro}

\begin{figure*}[h]
    \centering
\begin{pcvstack}[boxed,space=1em]
\begin{pchstack}
    \begin{pcvstack}[space=0.5em]
        \procedure[linenumbering, headlinecmd={\vspace{.1em}\hrule\vspace{.2em}}]{\rCMS.$\setupS(pp)$}{%
        \varepsilon, \delta \gets pp \\ 
        m \gets \left\lceil \frac{e}{\varepsilon} \right\rceil\\
        k \gets  \left\lceil \ln(\frac{1}{\delta}) \right\rceil\\
        h(\circ) \gets \murmurtwo(\circ) \bmod m\\
        \sigma \gets \zeros(k,m)\\
        \pcreturn \top
        }
       \end{pcvstack}
    \begin{pcvstack}[space=0.5em]
        \procedure[linenumbering, headlinecmd={\vspace{.1em}\hrule\vspace{.2em}}]{\rCMS.$\insS(x, \sigma, v)$}{%
            (p_1,\ldots,p_k) \gets h(x,1),\ldots,h(x,k)\\
            \pcfor i \in [k]\\
            \t \sigma[i][p_i] += v\\
            \pcreturn  \mathrm{min}_{i \in [k]}\{\sigma[i][p_i]\}
        }
        \procedure[linenumbering, headlinecmd={\vspace{.1em}\hrule\vspace{.2em}}]{\rCMS.$\qryS(x, \sigma)$}{%
            (p_1,\ldots,p_k) \gets h(x,1),\ldots,h(x,k)\\
            \pcreturn  \mathrm{min}_{i \in [k]}\{\sigma[i][p_i]\}
        }
    \end{pcvstack}	
\end{pchstack}		
\end{pcvstack}
\caption[The Redis CMS Structure.]{Redis count-min sketch algorithms. 
The analogous functions in the Redis API are: $\rCMS.\setupS$ is \textsf{CMS.INITBYPROB}, $\rCMS.\insS$ is \textsf{CMS.INCRBY}, and $\rCMS.\qryS$ is \textsf{CMS.QUERY}.
We refer to a Redis count-min sketch initialized with $\varepsilon, \delta \in (0,1)$ as CMS[$\varepsilon, \delta$].} 
\label{fig:redis-cms}
\end{figure*}

A count-min sketch supports frequency estimates, i.e. estimates of the number of times a particular element occurs in a data set. Originally introduced in~\cite{cormode2005improved}, a count-min sketch consists of a $k \times m$ array $\sigma$ of (initially zero) counters, and $k$ pairwise independent hash functions $h_1, ..., h_k$ that map between the universe~$\mathcal{U}$ of data items and $[m]$.

An element $x$ is added to a count-min sketch by computing 
$(p_1,p_2,\ldots,p_k) \gets h(x,1),\ldots,h(x,k)$, 
 then adding $1$ to each of the counters at $\sigma[i][p_i]$ for $i \in [k]$. This extends in the obvious way to insertions of $v$ instances of an element at a time. 
A frequency estimate for $x$ is computed as $\hat{n}_x = \ \mathrm{min}_{i \in [k]} \{\sigma[i][p_i]\}$. A count-min sketch may produce overestimates of the true frequency, but never underestimates.

For any $\varepsilon,\delta \, {\geq}\, 0$, any $x {\in}\, \mathcal{U}$, and any collection of data $\mathcal{C}$ stored by the count-min sketch (over $\mathcal{U}$) of length $N$, it can be guaranteed by appropriate setting of parameters that $\Pr[\hat{n}_x - n_x \,{>}\, \varepsilon N] \,{\leq}\, \delta$, where $n_x$ is the true frequency of $x$. Specifically, we can take $m \gets \lceil{ e/\varepsilon \rceil}$, $k \gets \lceil{ \ln{(1/\delta)} \rceil}$. This correctness bound holds when the individual hash functions are sampled from a pairwise-independent hash family $H$ (see~\cite{cormode2005improved} for a proof). It further assumes that insertions are done in the honest setting. That is, $\mathcal{C}$ and the queried element $x$ are independent of the internal randomness of the structure (the random choice of the hash functions). 


In Redis, a count-min sketch is initialized by the user calling $\rCMS.\setupS(\varepsilon, \delta)$. We will refer to the resulting sketch as CMS[$\varepsilon, \delta$]. The dimensions $m, k$ of the count-min sketch are then calculated as above, and a $k \times m$ array of zeros is initialized. We note that it is also possible to initialize the structure from the dimensional parameters $m, k$, rather than deriving them from $\varepsilon, \delta$. Insertions and membership queries on any element $x$ are carried out in the same way as in the original structure, using the commands $\rCMS.\insS(x, \sigma, v)$ and $\rCMS.\qryS(x, \sigma)$; both return the frequency estimate of $x$.
The analogous functions in Redis are called \textsf{CMS.INITBYPROB}, \textsf{CMS.INCRBY} and \textsf{CMS.QUERY}, respectively. 

To instantiate the $k$ pairwise independent hash functions, Redis uses $\murmurtwo$ with a per row seed equal to the row index, i.e. $h_1(x) \gets h(x, 1), ..., h_k(x) \gets h(x, k)$, where the syntax~$h(x,i)$ means $\murmurtwo$ evaluated on input~$x$ with seed~$i$. For full details of count-min sketches in Redis, see~\Cref{fig:redis-cms}.

 We point out that using fixed hash functions violates the honest setting assumptions that are required for the guarantees on frequency estimation errors in~\cite{cormode2005improved}. We will leverage this and the properties of~$\murmurtwo$ in our attacks to cause large frequency overestimates. 

\subsection{Top-K}

\begin{figure*}[h]
    \centering
	\begin{pcvstack}[boxed,space=1em]
		\begin{pchstack}
			\begin{pcvstack}[space=0.5em]
        %
        \procedure[linenumbering, headlinecmd={\vspace{.1em}\hrule\vspace{.2em}}]{\rTK.$\setupS(pp)$}{
        %$\mathsf{INIT}(w,d,\mathrm{decay},K)$}{%
        	   m, k, \mathrm{decay}, K \gets pp \\
	   \seed \gets 1919 \\ 
        	   h(\circ) \gets \murmurtwo(\circ) \bmod m\\
			   h_\mathit{fp} \gets \murmurtwo(\circ)\\
            \pcfor i \in [k] \\
            \t \sigma[i] \gets [(\star,0)]\times m\\
            H \gets \mathsf{initminheap}(K)\\
            \pcreturn \top
        }
        %
        %
        \procedure[linenumbering, headlinecmd={\vspace{.1em}\hrule\vspace{.2em}}]{\rTK.$\qryS(x, \sigma)$}{
        %$\mathsf{QUERY}(A,x)$}{%
        (p_1,\ldots,p_k) \gets h(x,1),\ldots,h(x,k)\\
        \fp_{x} \gets h_\mathit{fp}(x,\seed)\\
        \cnt_x \gets 0\\
        \pcfor i \in [k] \\
        \t \pcif  \sigma[i][p_i].\fp = \fp_x \\
        \t \t \cnt {\gets} {\sigma[i][p_i]}.\cnt\\
        \t \t  \cnt_x{\gets}{\max}\left\{\cnt_x, \cnt\right\}\\
        \pcreturn \cnt_x
        }
        %
        \procedure[linenumbering, headlinecmd={\vspace{.1em}\hrule\vspace{.2em}}]{\rTK.$\listS(\sigma)$}{
        %$\mathsf{KLIST}(\sigma)$}{%
        T \gets H.\mathsf{list}()\\
        \pcreturn T
        }
        %
    \end{pcvstack}
			\begin{pcvstack}[space=0.5em]
            \procedure[linenumbering, headlinecmd={\vspace{.1em}\hrule\vspace{.2em}}]{\rTK.$\insS(x, \sigma)$}{
        %$\mathsf{ADD}(A,H,\mathrm{decay})$}{%
			r \gets \mathrm{nil}\\
            (p_1,\ldots,p_k) \gets h(x,1),\ldots,h(x,k)\\
            \fp_{x} \gets h_\mathit{fp}(x,\seed)\\
            \cnt_x \gets 0\\
            \pcfor i \in [k]\\
            \t{\pcif} \sigma[i][p_i].\fp \, {\not\in} \{\fp_x, {\star}\}\\
            \t \t r \getsr \left[0,1\right)\\
            \t \t \pcif r \leq \mathrm{decay}^{\sigma[i][p_i].\cnt}\\
            \t\t\t \sigma[i][p_i].\cnt \,{-}{=}\, 1\\
            \t\pcif  \sigma[i][p_i].\cnt = 0 \\
            \t\t \sigma[i][p_i].\fp  \gets \fp_x \\
            \t\pcif \sigma[i][p_i].\fp = \fp_x \\
            \t\t \sigma[i][p_i].\cnt \,{+}{=}\, 1 \\
            \t\t \pcif  \sigma[i][p_i].\cnt > \cnt_x \\
            \t\t\t \cnt_x \gets \sigma[i][p_i].\cnt \\
            \pcif \cnt_x \in H \\
            \t H.\mathsf{update}(x,\cnt_x)\\
            \pcelseif \cnt_x > H.\mathsf{getmin}() \\
			\t r \gets H.\mathsf{getmin}()\\
            \t H.\mathsf{poppush}(x,\cnt_x)\\
            \pcreturn r
        }
        \end{pcvstack}
        \end{pchstack}
        \end{pcvstack}
	\caption[The Redis Top-K Structure.]{Redis Top-K structure algorithms. The analogous functions in the Redis API are: $\rTK.\setupS$ is \textsf{TOPK.RESERVE}, $\rTK.\insS$ is \textsf{TOPK.ADD}, $\rTK.\qryS$ is \textsf{TOPK.COUNT}, and $\rTK.\listS$ is \textsf{TOPK.LIST}.
		  We refer to a Redis Top-K structure initialized with $pp=m,k,\mathrm{decay}, K$ as TK[$m,k,\mathrm{decay}, K$]. 
	} 
	\label{fig:redis-topk}	
\end{figure*}

A Top-K structure, originally introduced as the HeavyKeeper in~\cite{yang2019heavykeeper}, solves the approximate top-$K$ problem. 

The exact version of the problem is defined as follows: given elements of a data collection~$\mathcal{C} \subseteq  \{ e_1,e_2,...,e_m \}$ with associated frequencies~$(n_{e_1},n_{e_2},...,n_{e_m})$, we can order the elements~$\{ e^{*}_{1},e^{*}_{2},...,e^{*}_{M} \}$ such that~$(n^{*}_{e_1} \geq n^{*}_{e_2} \geq ... \geq n^{*}_{e_M})$. Then, for some~$K \in \mathbb{Z}^{+}$, we output the set of elements~$\{ e^{*}_{1},e^{*}_{2},...,e^{*}_{K} \}$ with the~$K$ largest frequencies~$(n^{*}_{e_1} \geq n^{*}_{e_2} \geq ... \geq n^{*}_{e_K})$. Given space linear in the stream this is trivial to solve exactly. However, by the pigeonhole principle, it is not possible to find an exact solution with space less than linear (see~\cite{Roughgarden_Valiant} for a formal impossibility argument).
A common technique is to place a small data structure of size~$O(K)$, like a heap or list, on top of a compact frequency estimator. By updating this small structure at most once upon an insertion of each element, we can approximate this top-$K$ set~\cite{mandal2018topkapi,metwally2006}. Using this technique we will obtain the~$K$ elements with the largest estimated frequencies. 

The Top-K structure is represented by a $k{ \times} m$ matrix~$\sigma$. Each entry in~$\sigma$ is an $(\fp,\cnt)$ pair, where~$\fp$ is a fingerprint of the element that ``owns'' the counter, and~$\cnt$ is said element's recorded count. These entry pairs are initialized to the distinguished symbol~$\star$ and zero, respectively. Associated with each row is a hash function that maps elements in~$\mathcal{U}$ to $[m]$, i.e. $k$ hash functions $h_1,...,h_k$. The fingerprint hash function $h_\mathit{fp}$ maps elements in~$\mathcal{U}$ to $\{ 0,1 \}^{\lambda_{\mathit{fp}}}$, for some desired fingerprint length~$\lambda_{\mathit{fp}}$. Further, we initialize a min-heap~$H$ of maximal size~$K$ to store the elements with the~$K$ largest estimated frequencies. Lastly, a~$\mathrm{decay}$ value is set, which is used to decrement a counter when a specific condition is hit. 

To insert an element~$x$, we start by computing $(p_1,...,p_k) \gets (h_1(x),\ldots,h_k(x))$. We then compute the fingerprint $\fp_x$ associated with the element~$x$ as $h_\mathit{fp}(x)$. We also set a variable~$\cnt_x \gets 0$. We then go row by row (indexed by~$i \in [k]$), with the following cases:
\begin{enumerate}
    \item \textbf{if} $\fp^{*} = \star$, 
	where~$\fp^{*}$ is the current fingerprint value at matrix position~$(i,p_i)$, \textbf{then} we set the counter value to~$1$, the fingerprint to~$\fp_x$, and if~$\cnt_x < 1 : \cnt_x \gets 1$.
    \item \textbf{else if} $\fp_x = \fp^{*}$,
    we add~$1$ to the counter value, and if~$\cnt_x < c : \cnt_x \gets c$, where $c$ is the current counter value at matrix position~$(i,p_i)$. 
    \item \textbf{else}
    we select a random value~$r \getsr [0,1)$. If~$r < \mathrm{decay}^{c}$, where~$c$ is the current counter value at matrix position~$(i,p_i)$, we decrement the counter value stored at this position. If, after decrementing, this value is~$0$, we then set the counter value to~$1$, the fingerprint to~$\fp_x$, and if~$\cnt_x < 1 : \cnt_x \gets 1$. This is the so-called probabilistic decay process. 
\end{enumerate}

If, after this procedure, it is such that $x \in H$, we update the entry in the heap based on the current value of~$\cnt_x$. Else, we check that~$\cnt_x > H.\mathsf{min}$, and if so we remove the min entry in~$H$ and replace it with~$(x,\cnt_x)$. This ensures that we are keeping an accurate account of the~$K$ highest estimated frequencies in~$H$. 

Top-K provides approximate answers to frequency queries for any element $x$, by 
computing $(p_1,\ldots,p_k) \gets (h_1(x),\ldots,h_k(x))$ and~$\fp_x \gets h_\mathit{fp}(x)$, and returning $\hat{n}_x = \max_{i \in [k]} \{\sigma[i][p_i]\}$ where $\sigma[i][p_i].\fp = \fp_x$. % removed ~ for formatting reasons
If none of the fingerprints in this set of buckets equals~$\fp_x$, then~$0$ is returned. 
Top-K returns the estimated top-$K$ elements by returning all the pairs of items and estimated counts stored in~$H$. 


In~\cite{yang2019heavykeeper}, a probabilistic guarantee for estimation error magnitude is presented, assuming that each $\sigma[i][j]$ has a sole owner throughout the processing of the entire stream. However, the statement lacks precision, and its proof is flawed, thus we will not restate it (see instead~\cite{markelon23} for a meaningful result). Moreover, the results in~\cite{yang2019heavykeeper} rely on a no-fingerprint collision (NFC) assumption, ensuring that all frequency estimates satisfy $\hat{n}_x \leq n_{x}$, where $n_x$ is the true frequency of $x$, i.e. Top-K strictly underestimates frequencies. While not formally defined in the original paper, a rigorous definition is given in~\cite{markelon23}, characterizing NFC as the assumption that elements hashing to the same row position in any row do not share a fingerprint. This assumption is reasonable for practical sizes of $\mathcal{U}$ and a sufficiently large fingerprint space.

To initialize a Top-K structure in Redis, the user specifies $k, m$, $\mathrm{decay}$, and $K$, by calling \rTK.$\setupS(k, m, \mathrm{decay}, K)$. (The analogous function in Redis is called \textsf{TOPK.RESERVE}.) We refer to the resulting structure as TK[$k, m, \mathrm{decay}, K$].
The hash functions for each row are again computed as $h_1(x) \gets h(x, 1), ..., h_k(x) \gets h(x, k)$, with $h$ set to $\murmurtwo \bmod m$. The fingerprint hash function is computed as $h_\mathit{fp} \gets h(x, \seed)$, with $h_\mathit{fp}$ set to $\murmurtwo$ ($\lambda_{\mathit{fp}} = 32$) with a fixed $\seed = 1919$. The $\mathrm{decay}$ value is by default set to~$0.9$. 

Insertions and frequency queries on an element $x$ then proceed as described above, through the $\rTK.\insS(x, \sigma)$ and $\rTK.\qryS(x, \sigma)$ functionalities. Similar to the count-min sketch, multiple instances of an element can be added to the Top-K, however this is implemented through repeated invocations of the insert algorithm described above. 
To return the top-$K$ elements, one invokes $\rTK.\listS(\sigma)$. (The analogous functions in Redis are called \textsf{TOPK.ADD}, \textsf{TOPK.COUNT} and \textsf{TOPK.LIST}, respectively.) For full details of the Redis Top-K structure, see~\Cref{fig:redis-topk}. 

We will show that the specific implementation choices that Redis makes leads to security issues. Specifically, we give attacks that block the true~$K$ most frequent elements from being reported in the top-$K$ estimation (with overwhelming probability) whether or not these elements are known to the attacker before the attack. Further, we show that one is able to trivially violate the NFC assumption and cause the Redis Top-K structure to allow for frequency overestimates.
%-------------------------------------------------------------------------------

%-------------------------------------------------------------------------------
\section{Attacks Against PDS in Redis}
In this section, we construct attacks against the Redis implementations of  count-min sketches and Top-K structures (for attacks against Bloom filters and Cuckoo filters see~\cite{cryptoeprint:2024/1312}). While our attacks vary in their goals and complexity, at their core, they all exploit Redis' choice of weak hash functions (from the $\mathit{MurmurHash}$ family) and their invertibility.
By implementing our attacks and giving experimental results, we demonstrate that malicious Redis users can severely disrupt the performance of each PDS. Code for our attacks can be found at~\cite{gitrepo}.

\subsection{MurmurHash Inversion Attacks}\label{sec:MurmurHash}
The Redis PDS suite relies heavily on two different $\mathit{MurmurHash}$ hash functions: $\murmur$ and $\murmurtwo$. Both functions accept an element, a length parameter and a $\seed$ as input. The functions have, respectively, 64-bit and 32-bit outputs. In Redis, all inputs must have valid ASCII encoding, as the length field is set to the character length of the string representation of the input. Seeds are usually set to fixed values.

The $\mathit{MurmurHash}$ family of hash functions are designed to be fast but are not cryptographically secure. 
Indeed, starting with a target hash value $h$ and a given seed, it is easy to find one or many elements that hash to $h$ under either $\murmur$ or $\murmurtwo$, so these functions are not even one-way. 
We refer to these resulting elements as pre-images of $h$, and the algorithms that compute them as \emph{inversion} algorithms. % If explanation of MurmurHash2Inverse, put it here
Our inversion algorithms for $\murmur$ and $\murmurtwo$ are about as fast as computing the hash functions in the forward direction. They are based on the deterministic approach in \cite{murmurhash64Ainv}. However, we adapt this method to make our algorithms randomized and to be able to produce many pre-images for the same target hash value $h$. 
For $\murmur$, our inversion algorithm outputs strings consisting of two $64$-bit blocks $B_1, B_2$ in which $B_2$ is chosen arbitrarily and $B_1$ is then determined by $B_2$ and the seed. 
Similarly, for $\murmurtwo$, but with 32-bit blocks. 
In both cases, the algorithms can be modified to produce inversions that are $t$-block messages for any $t$; then any $t-1$ of the blocks can be freely chosen (with the remaining one then being determined). 
However, pre-images that comprise two $64$-bit or $32$-bit blocks suffice for our attacks.

For attacks on Redis, we must also further modify our algorithms to ensure the pre-images are valid ASCII-encoded strings. Meeting this additional requirement incurs extra cost. 
For $\murmur$, given a valid ASCII-encoded $B_2$, ensuring that $B_1$ has the correct format requires on average $2^8$ trial inversions, hence costing roughly the same as 256 forward hash function computations. 
Here, the factor of $2^8$ comes from a 64-bit string representing 8 ASCII characters, each of which must have a single bit set to zero. 
For $\murmurtwo$, an average of 16 trial inversions is needed to obtain a 2-block pre-image respecting the ASCII constraint. 
Additionally, we enforce the leading byte of $B_1$ to be non-zero to ensure that the length of the pre-image, when viewed as a string, is exactly 16 or 8 bytes. This is important as $\murmur$ and $\murmurtwo$ outputs depend on the input length. 
%and our inversion process depends on it
Overall, this results in an average number of an equivalent of $256\cdot \frac{128}{127}\approx258$ and $16\cdot \frac{128}{127}\approx16$ hash function calls to compute a correctly formatted 2-block pre-image for $\murmur$ and $\murmurtwo$.
% Additionally, we need to assume the length field of the hash function when computing a pre-image. We assume $16$ bytes for $\murmur$ and $8$ bytes for $\murmurtwo$. However, there is a chance that the constructed pre-image has a leading zero byte, in which case we rerun the algorithm. Overall, this results in an average number of $256\cdot \frac{128}{127}\approx258$ and $32\cdot \frac{128}{127}\approx32$ trials to get a correctly formatted pre-image for $\murmur$ and $\murmurtwo$.

It is also possible to construct so-called universal multi-collisions for certain hash functions in the $\mathit{MurmurHash}$ family~\cite{MurmurUMC}. These are large sets of input values that all hash to the same output, irrespective of the seed. For $\murmur$, such inputs could be useful in our targeted false positive attack on Redis' Bloom filter below; however, they seem to be difficult to construct while respecting the ASCII encoding requirement. We leave the construction and exploitation of such collisions to future work.

\subsection{Count-Min Sketch Attack}\label{attacks:cms}

We give an attack against Count-Min sketches in Redis that causes large frequency overestimates for any target element.

\subsubsection{Overestimation attack} 
\label{sec:cms-overestimation}

Consider a Count-Min sketch with parameters $\varepsilon, \delta$. 
After initializing CMS[$\varepsilon, \delta$] $\sigma$, an adversary $\advA$ is given access to insertion and query oracles: $\insO(\cdot) := \rCMS.\insS(\cdot, \sigma)$ and $\qryO(\cdot) := \rCMS.\qryS(\cdot, \sigma)$. In a frequency overestimation attack, the adversary is given a target element $x$ as input and is challenged with causing the frequency of $x$ to be overestimated. A metric for the adversary's success is the value $\rCMS.\qryS(x, \sigma) - n_x$, where $n_x$ is the number of times $x$ was actually inserted into the Count-Min sketch. 

We begin by recalling that, for a frequency estimation query on an element $x \in \mathcal{U}$, the response given by a Count-Min sketch has one-sided error, i.e. it only overestimates. In the honest setting, this error can be bounded according to the number of items inserted into the structure and the parameters of the structure (see Section~\ref{sec:cms-intro}). We will show that in an adversarial setting, we can exploit knowledge of the internal randomness of the structure to cause the sketch to make massive overestimates of the frequency of a target element~$x$. 

In~\Cref{sec:cms-attacks} we present attacks against the general CMS structure. 
We could directly apply their ``public hash'' attack to the Redis implementation of the Count-Min sketch, as the seeds used for each row hash function are hard-coded. However, Redis' choice to use~$\murmurtwo$ for row position hash functions allows us to exploit the invertibility of the function to speed up the attack. As $\murmurtwo$ is invertible, we can generate an arbitrary number of multicollisions for a fixed hash output and seed. This allows us to carry out the attack more efficiently.%for reasonable sized parameters. 

\begin{figure}[htp] 
    \centering
    \begin{pcvstack}[boxed,center]
        \procedure[linenumbering, headlinecmd={\vspace{.1em}\hrule\vspace{.2em}}]{overestimation\_attack$^{\insO}(x,pp,I)$}{%
        \textrm{cover} \gets \textrm{find\_cover}(x,pp)\\
        \pcuntil I \text{ insertions are made}\\
        \t \pcfor e \in \textrm{cover}{:} \ \insO(e)\\
        \pcreturn \textrm{done}
        }
        \procedure[linenumbering, headlinecmd={\vspace{.1em}\hrule\vspace{.2em}}]{find\_cover$(x,pp)$}{%
            \varepsilon, \delta \gets pp \\
            k \gets  \left\lceil\ ln(\frac{1}{\delta}) \right\rceil\\
            \textrm{cover} \gets \emptyset\\
            (p_1,\ldots,p_k) \gets h(x,1),\ldots,h(x,k)\\
            \pcfor i \in [k]\\
            \t y \gets \mathit{MurmurHash2Inverse}(p_i,i)\\
            \t \textrm{cover} \gets \textrm{cover} \cup \{ y \} \\
            \pcreturn \textrm{cover}
        }
        %
    \end{pcvstack}
    \caption[Redis CMS Overestimation Attack.]{The count-min sketch overestimation attack. We use the invertibility of $\murmurtwo$ to find a cover. We then repeatedly insert the cover to create error. Note that we abuse notation and assume that~$\mathit{MurmurHash2Inverse}$ is run until a validly encoded pre-image is found.} 
	\label{fig:cms-attack}
\end{figure}

To create an overestimation error on~$x$, one must find a cover for~$x$, which (with respect to the parameters of a given Count-Min sketch) is a set of elements~$\{ y_1{,}...{,}y_k\}$ such that ${\forall} i {\in} [k] {:} h(x,i) {=} h(y_i,i)$ and $\forall i {\in} [k] {:} y_i {\neq} x$. We use the fact that $\murmurtwo$ is invertible to find our cover. Let $p_i$ denote $h_i(x)$ for $i \in [k]$, where $h_i(\cdot)$ is instantiated using $\murmurtwo(\cdot,i)$ as in Redis. We then set $y_i$ by inverting $\murmurtwo(\cdot,i)$ at $x$ for $i \in [k]$. Respecting Redis' ASCII encoding constraint, we expect this to cost an equivalent of about 16 hash function evaluations for each $i$ (as per Section~\ref{sec:MurmurHash}). Therefore, we expect a total cost of about~$16 k$ $\murmurtwo$ computations. Once the cover is found, we simply repeatedly insert it, using $\insO$ calls on $y_i$ for $i \in [k]$. Since we never insert $x$ and our covers are always of size $k$, after $I$ insertions we observe an error on $x$ equal to $\lfloor \frac{I}{k}  \rfloor$, i.e. $\rCMS.\qryS(x, \sigma) {-} n_x \geq \lfloor \frac{I}{k} \rfloor$. For a full description of our attack, see~\Cref{fig:cms-attack}.

We remark that the attack also works against structures that already have elements stored in them., as the Count-Min sketch is a linear structure.


\begin{table}[htp]
	\centering
	\setlength\tabcolsep{4pt}%
	\begin{tabular}{|c|c|c|}
		\hline
		\begin{tabular}[c]{@{}c@{}}$\epsilon,\delta$ $(m,k)$\end{tabular}                          & \begin{tabular}[c]{@{}c@{}} Ours \end{tabular} & \cite{markelon23} \\ \hline
		\begin{tabular}[c]{@{}c@{}}$2.7 \times 10^{-3},1.8 \times 10^{-2}$\\ $(1024,4)$\end{tabular} & 66.85                                                                          &  8533.32  \\ \hline
		\begin{tabular}[c]{@{}c@{}}$6.6 \times 10^{-4},1.8 \times 10^{-2}$\\ $(4096,4)$\end{tabular} & 61.11                                                                          & 34133.36   \\ \hline
		\begin{tabular}[c]{@{}c@{}}$2.7 \times 10^{-3},3.4 \times 10^{-4}$\\ $(1024,8)$\end{tabular} & 124.22                                                                         & 22264.72   \\ \hline
		\begin{tabular}[c]{@{}c@{}}$6.6 \times 10^{-4},3.4 \times 10^{-4}$\\ $(4096,8)$\end{tabular} & 128.8                                                                          & 89058.72 \\ \hline
	\end{tabular}
	\caption[Comparison of Redis CMS Attack Versus Generic Attack.]{Experimental number (average over 100 trials) of equivalent~$\murmurtwo$ calls needed to find a cover for a random target~$x$. We compare the average to the expected number of~$\murmurtwo$ calls needed in the attack of~\cite{markelon23} given in~\Cref{sec:cms-attacks}, namely $kmH_{k}$. 
	}
	\label{tab:cms-oe-atk}
\end{table}

We implemented the attack and measured the computation needed for a variety of~$\varepsilon, \delta$. We compare the error to the forward hash computation based attack of~\cite{markelon23} given in~\Cref{sec:cms-attacks} with the one we present here. The results are summarized in Table~\ref{tab:cms-oe-atk}. As we can see our experimental results tightly match our analysis, and our attack is at least an order of magnitude less expensive than previous best attack in~\cite{markelon23}. Further, to verify the correctness of our attack we mounted it against the Redis Count-Min sketch and selected a random target element. We found a cover for said element and verified that for a fixed number of insertions~$I$ we obtained the expected error on the target, i.e. achieved error~$\lfloor \frac{I}{k}  \rfloor$ in all trials. 

\subsection{Top-K}\label{attacks:tk}

We present three attacks on the Top-K structure in Redis. 
The first two attacks suppress the reporting of the true top-$K$ elements, while the third attack causes frequency overestimates by violating the no-fingerprint collision assumption.

\subsubsection{Known top-$K$ hiding attack}
\label{sec:known-top-k-hiding}

Consider a Top-K structure with parameters $m, k, \mathrm{decay}, K$. After initializing TK[$m, k, \mathrm{decay}, K$] $\sigma$, a collection of data $\mathcal{C}$ with true top-$K$ elements $F$ is generated from some honest distribution (that is, a distribution that does not depend on the internal randomness of the structure). In practice, we can take this to be some collection of network traffic or a collection of items in a large database. 

Then, an adversary $\advA$ is given access to insertion and query oracles $\insO(\cdot) := \rTK.\insS(\cdot, \sigma)$ and $\qryO(\cdot) := \rTK.\qryS(\cdot, \sigma)$. In a known top-$K$ hiding attack, the adversary receives $F$ as input and wins if it suppresses the reporting of the true top-$K$ elements $F$. The adversary's success can be checked by inserting $\mathcal{C}$ and checking whether $[f {\notin} \rTK.\listS(\sigma)]$ for all $f \in F$. Due to the probabilistic decay mechanism, we need the adversary to be able to insert elements into the structure before the honest collection is processed. In practice this is reasonable, as adversaries can time their attacks to ensure  they have early access to the structure.

To carry out this attack, we adapt the strategy from~\cite{markelon23} given in~\Cref{sec:hk-attacks}. We begin by computing a cover using the inversion strategy for every element in~$F$. We then insert every element in the cover $t$ times through $\insO(\cdot)$ calls, where 
$t$ is computed such that there exists negligible probability that, after the cover is inserted, any element from~$F$ will ever own any of their counters. The algorithm to compute~$t$ takes inputs~$p,n$, where $p$ is the probability that a cover element will relinquish ownership of its counters and~$n$ is the number of colliding insertions we expect. We set~$p=2^{-128}$ and~$n$ to the frequency of the maximum~$f \in F$ for this attack. Once~$\mathcal{C}$ is inserted after the attack phase, all elements in~$F$ will have estimated frequency equal to zero, and will in turn not be reported in the top-$K$ list as they should.

In practice, $t$ will be quite small compared to the frequencies of the elements in~$F$ for a real-world data collection~$C$. The frequency of all~$f \in F$ is often of the order of~$10^5$ or greater, yielding~$t$ of the order of~$10^3$ for~$p=2^{-128}$.  Thus, the true top-K of $\mathcal{C}$ equals the top-K of the new stream consisting of our attack elements concatenated with $\mathcal{C}$. For more details on this attack (including the calculation of $t$), see ~\Cref{fig:tk-known-attack}.

We expect an equivalent of $16k|F|$ calls to~$\murmurtwo$ to find a cover for known true top-K list~$F$. To test our attack, we initialized a TK[$4096, 20, 0.9, 20$], selected our data collection~$\mathcal{C}$ as the individual words in the English language version of \emph{War and Peace}, and computed~$F$ for~$K{=}20$ for $\mathcal{C}$. Our choice of $\mathcal{C}$ was inspired by Redis' blog post introducing the structure~\cite{redisblogtopK}. 
We then computed a cover on~$F$ using our technique described above. Averaged over 100 trials, we made an equivalent of~$2580$ calls to~$\murmurtwo$, matching our analysis. We then inserted each element in the cover~$t$ times for~$t{=}206$ based on input parameters~$p{=}2^{-128}, n{=}34577$ (the frequency of the most frequent element). After this, the entirety of~$\mathcal{C}$ was inserted. In every trial, the reported top-$K$ and $F$ were disjoint as desired. 

\begin{figure}[htp]
    \centering
    \begin{pcvstack}[boxed,center]
        \procedure[linenumbering, headlinecmd={\vspace{.1em}\hrule\vspace{.2em}}]{known\_F\_attack$^{\insO}(F,n,p,pp)$}{%
        t \gets \textrm{get\_t}(n,p,pp)\\
        \textrm{F\_cover} \gets \textrm{find\_F\_cover}(F,pp)\\
        \pcfor e \in \textrm{F\_cover}\\
        \t \pcfor i \in [t]\\
        \t\t \insO(e)\\
        \pcreturn \textrm{done}
        }
        \procedure[linenumbering, headlinecmd={\vspace{.1em}\hrule\vspace{.2em}}]{get\_t$(n,p,pp)$}{%
            m, k, \mathrm{decay}, K \gets pp \\
            g(t) \gets \log_2(k \cdot n^t \cdot \mathrm{decay}^{t(t+1)/2} ) - \log_2(p)\\
			t_1,t_2 \gets \text{FindRootsOf}(g)\\
			\pcif t_1 > 1 \textbf{ or } t_2 < 1: t \gets 1\\
			\pcif t_2 > 1: t \gets \left\lceil t_2 \right\rceil\\
			\pcif t_2 = 1: t \gets 2\\
			\pcreturn t
        }
        \procedure[linenumbering, headlinecmd={\vspace{.1em}\hrule\vspace{.2em}}]{find\_F\_cover$(F,pp)$}{%
            m, k, \mathrm{decay}, K \gets pp \\
            \textrm{F\_cover} \gets \emptyset\\
            \pcfor f \in F\\
            \t (p_1,\ldots,p_k) \gets h(f,1),\ldots,h(f,k)\\
            \t \pcfor i \in [k]\\
            \t\t y \gets \mathit{MurmurHash2Inverse}(p_i,i)\\
            \t\t \textrm{F\_cover} \gets \textrm{F\_cover} \cup \{ y \} \\
            \pcreturn \textrm{F\_cover}
        }
        %
    \end{pcvstack}	
	\caption[Redis TK Known Top-$K$ Hiding Attack.]{The Top-K known top-$K$ hiding attack.}\label{fig:tk-known-attack}
\end{figure}

\subsubsection{Hidden top-$K$ hiding attack}

We consider a similar attack model to~\cref{sec:known-top-k-hiding} with the modification that the adversary $\advA$ receives no input.
Since $\advA$ does not know~$F$, it must compute a cover for the entire structure, i.e. all~$k {\times} m$ counters. 
We go counter-by-counter and use hash inversion to compute a cover element for each counter. Note, however, that when computing a cover element for a particular counter, we collect additional positions in other rows that the element touches (if we have not yet covered said positions). In this way, we actually do less work than the expected equivalent of~$16mk$ calls to~$\murmurtwo$. 

After computing this cover for the entire structure, $\advA$ then inserts each element in the cover~$t$ times through $\insO(\cdot)$ calls, with $t{=}500$ (corresponding to~$p{=}2^{-128},n{=}10^{11}$ from the previous method of computing~$t$). In practice, setting~$t{=}500$ means that with overwhelming probability no true top-K element will ever own its counters for any realistic data collection. Then, for any subsequent items inserted that are not part of the cover, their estimated frequency will be zero. In practice, this blocks any~$F$ from any realistic data collection~$\mathcal{C}$ from being reported in the top-$K$ list. This attack can be seen as a denial-of-service attack, as after the attack phase the structure is prevented from making accurate frequency estimates for any elements that are subsequently inserted into the Top-K. Our full attack is given in~\Cref{fig:tk-hidden-attack}.

We verified the correctness of the attack as in~\cref{sec:known-top-k-hiding}, except again now setting~$t{=}500$. 

\begin{figure}[htp]
    \centering
    \begin{pcvstack}[boxed,center]
        \procedure[linenumbering, headlinecmd={\vspace{.1em}\hrule\vspace{.2em}}]{hidden\_F\_attack$^{\insO}(n,p,pp)$}{%
        t \gets 500\\%\textrm{get\_t}(n,p,pp)\\
        \textrm{S\_cover} \gets \textrm{find\_S\_cover}(pp)\\
        \pcfor e \in \textrm{S\_cover}\\
        \t \pcfor i \in [t]\\
        \t\t \insO(e)\\
        \pcreturn \textrm{done}
        }
        \procedure[linenumbering, headlinecmd={\vspace{.1em}\hrule\vspace{.2em}}]{find\_S\_cover$(pp)$}{%
            m, k, \mathrm{decay}, K \gets pp \\
            \eta \gets \zeros(k,m)\\
            \textrm{S\_cover} \gets \emptyset\\
            \pcfor i \in [k]\\
            \t \pcfor j \in [m]\\
            \t\t \pcif \eta[i][j] = 0\\
            \t\t\t y \gets \mathit{MurmurHash2Inverse}(j,i)\\
            \t\t\t \textrm{S\_cover} \gets \textrm{S\_cover} \cup \{ y \}\\
            \t\t\t (p_1,\ldots,p_k) \gets h(y,1),\ldots,h(y,k)\\
            \t\t\t \pcfor r \in [k]\\
            \t\t\t\t \eta[r][p_r] \gets 1\\
            \pcreturn \textrm{S\_cover}
        }
        %
    \end{pcvstack}
	\caption[Redis TK Hidden Top-$K$ Hiding Attack.]{The Top-K hidden top-$K$ attack.}
    \label{fig:tk-hidden-attack} 
\end{figure}


\subsubsection{NFC assumption violation attack}

Consider a Top-K structure with parameters $m, k, \mathrm{decay}, K$. After initializing TK[$m, k, \mathrm{decay}, K$] $\sigma$, an adversary $\advA$ is given access to insertion and query oracles: $\insO(\cdot) := \rTK.\insS(\cdot, \sigma)$ and $\qryO(\cdot) := \rTK.\qryS(\cdot, \sigma)$. The adversary's goal in an NFC assumption violation attack equates to the same goal as of that in Section~\ref{sec:cms-overestimation}. That is, $\advA$ receives $x$ as input and is challenged with causing the frequency of $x$ to be overestimated. Again we can use $\rTK.\qryS(x, \sigma) - n_x$ as a metric of success, where $n_x$ is the number of times $x$ was actually inserted into the Top-K structure.


Recall that under the no-fingerprint collision assumption, the Top-K structure only underestimates frequencies of elements. 
We will show that, with the Redis implementation of Top-K, it is trivial to violate this assumption, and thus create large frequency overestimation errors. 

\begin{table}[htp]
	\centering
	\begin{tabular}{|c|c|c|}
	\hline
	$(m,k)$    & \begin{tabular}[c]{@{}c@{}}$\murmurtwo$ inversions\end{tabular} & \begin{tabular}[c]{@{}c@{}}$\murmurtwo$ calls\end{tabular} \\ \hline
	$(1024,4)$ & 4296.69                                                                        & 1072.52                                                    \\ \hline
	$(4096,4)$ & 18489.68                                                                       & 4602.56                                                    \\ \hline
	$(1024,8)$ & 1849.71                                                                        & 905.44                                                     \\ \hline
	$(4096,8)$ & 10058.16                                                                       & 5031.52                                                    \\ \hline
	\end{tabular}
	\caption[Cost of Redis TK NFC Violation Attack.]{Experimental number (averaged over 100 trials) of~$\murmurtwo$ inversion trials  and~$\murmurtwo$ calls needed to find a cover element for a randomly selected target~$x$. Recall that the cost of each is about the same.
	}
	\label{tab:tk-nfcv-atk}
\end{table}

To create large error on a given target~$x$, we compute multicollisions on the fingerprint of~$x$, stopping when we find a collision such that it shares one row position with~$x$. Unlike the attack against the Count-Min sketch, we only need to find such a collision in one row, as the Top-K takes the maximum count over all owned counters. Therefore, we are now finding a single cover element $y$. Then, by inserting the cover element $I$ times using $\insO(y)$, $\advA$ can expect to create error $I$ on the frequency estimation of $x$, i.e. $\rTK.\qryS(x, \sigma) - n_x \geq I$. Experimental results measuring the cost for this attack are given in Table~\ref{tab:tk-nfcv-atk}. We need~$\murmurtwo$ computations ($k$ per successful inversion) to check if the collision element we found matches any of the row positions to which our target maps. 

We verified the correctness of the attack in the same way as in~\cref{sec:cms-overestimation}, obtaining the expected error~$I$  on the randomly select target~$x$ over all trials. For more details of our attack, see ~\Cref{fig:tk-nfc-attack}.

\begin{figure}[htp]
    \centering
    \begin{pcvstack}[boxed,center]
        \procedure[linenumbering, headlinecmd={\vspace{.1em}\hrule\vspace{.2em}}]{nfc\_violation\_attack$^{\insO}(x,pp,I)$}{%
        y \gets \textrm{find\_cover\_element}(x,pp)\\
        \pcuntil I \text{ insertions are made}\\
        \t \insO(y)\\
        \pcreturn \textrm{done}
        }
        \procedure[linenumbering, headlinecmd={\vspace{.1em}\hrule\vspace{.2em}}]{find\_cover\_element$(x,pp)$}{%
            m, k, \mathrm{decay}, K \gets pp \\
            \seed \gets 1919 \\ 
            \textrm{done} \gets \bot \\
            P \gets (h(x,1),\ldots,h(x,k))\\
            \fp_x \gets h_\mathit{fp}(x)\\
            \pcwhile \textrm{done} = \bot \\
            \t y \gets \mathit{MurmurHash2Inverse}(\fp_x,\seed)\\
            \t C \gets (h(y,1),\ldots,h(y,k))\\
            \t \pcfor i \in [k]\\
			\t\t \pcif P[i] = C[i] \\
            \t\t\t \textrm{done} \gets \top \\
            \pcreturn y
        }
        %
    \end{pcvstack}	
\caption[Redis TK NFC Violation Attack.]{The Top-K no-fingerprint collision violation attack. We use the invertibility of $\murmurtwo$ to find a single fingerprint collision and row pair element for the target~$x$. We then repeatedly insert the element to create error.}\label{fig:tk-nfc-attack}
\end{figure}
%-------------------------------------------------------------------------------

%-------------------------------------------------------------------------------
\section{Potential Countermeasures}
In this section, we outline some countermeasures that limit the effectiveness of our attacks. For remarks on the Bloom filter and Cuckoo filter we again refer the reader to~\cite{cryptoeprint:2024/1312}.

Protecting the count-min sketch and Top-K against frequency estimation attacks is challenging. Recall, that both the Count-Min sketch and the Top-K are a class of probabilistic data structures called \emph{compact frequency estimators} (CFE). In Chapter 3 we explore both of these structures in detail, and show that even when switching the hash functions to a keyed primitive (e.g. a PRF) and keeping the internal state of the structure efficient attacks that cause massive frequency estimation errors are still possible~\cite{markelon23}. That is the leakage from insertions and queries to a black-boxed structure is sufficient to carry out the style of attacks we present in this paper. The choices of Redis make these attacks easier to carry out, but findings are negative in any case. 

It is of great interest to explore secure PDS for frequency estimate queries that are tenable for real world applications. One could of course disallow queries to the structure, or use some public-key infrastructure to only allow insertions from authenticated parties. However, this clearly limits both the usability and performance. Another possibility is to explore new ways of constructing frequency estimation PDS, such as the Count-Keeper introduced in~\cite{markelon23}. While this structure remains susceptible to the types of attacks we present here, they are less effective, and the Count-Keeper has a native ability to flag suspicious frequency estimates. 

\subsection{Concluding Remarks}

We made a comprehensive security analysis of the Redis PDS suite, developing 10 different attacks across four PDS. 
Our attacks can be used to cause severe disruptions to the performance of systems relying on these PDS, ranging from mis-estimation of data statistics to triggering denial-of-service attacks. 
Our work illustrates the importance of low-level algorithmic choices and the dangers of using weak hash functions in PDS. 

Our work opens up interesting directions for future work. Various other PDS suites exist in the wild, such as in Google BigQuery and Apache Spark, and could also be subjected to detailed security analysis as we have done for Redis here. 
Methods to provably protect PDS against attacks have been proposed in~\cite{NaorY15,clayton2019,FPUV22,PatersonR22}. However, these analyses tend to focus on textbook versions of the PDS. 
Adapting these analyses to cater to the specifics of different implementations would help improve confidence in the deployed variants.

At a higher level, there still seems to be a lack of understanding  in the broader developer community about the risks of using PDS in potentially adversarial settings. 
Work is needed to educate developers about these risks; we hope this paper can play a part in this effort. 
As an alternative, in an effort to shield developers from these risks, one could develop new PDS implementations that are secure by default and package them in the form of easily consumed libraries with safe APIs. Such an effort could leverage the experience that the research community has gained from developing ``safe by default'' cryptographic libraries.

%-------------------------------------------------------------------------------     
\chapter{Provably Robust Skipping-based Probabilistic Data Structures}

Probabilistic data structures (PDS), such as hash tables, Bloom filters, and skip lists, are widely implemented due to their memory efficiency and favorable operational time complexity, making them essential tools for cost-effective, scalable data processing in resource-constrained environments. Industry adoption of these structures is extensive and growing: Redis, an open-source in-memory data structure store, leverages HyperLogLog for cardinality estimation and Bloom filters for membership testing, achieving superior performance in high-volume deployments at Twitter, Pinterest, and other large internet companies~\cite{redis}. Similarly, Nowack et al. demonstrated enhanced scalability in Discord's server member management through skip list-based implementations~\cite{discord}, Prout et al. utilized skip lists for relational database indexing~\cite{singlestore}, while Schanck et al. reported significant cost reductions in Mozilla Firefox's certificate revocation checking using the novel Clubcard PDS~\cite{clubcard}. Despite their widespread application, analyses of these structures' behavior under adversarial conditions remain notably sparse in the literature. This gap is significant because these data structures frequently operate in contexts where malicious actors might deliberately manipulate inputs to induce erroneous outputs or degrade the performance of these structures.

Compressing probabilistic data structures (CPDS), such as Bloom filters, HyperLogLog, count-min sketch, etc., provide compact (sublinear) representations of potentially large collections of data and support a small set of queries that can be answered efficiently. These space and (by extension) performance gains come at the expense of correctness.  Specifically, CPDS query responses are computed over the compact representation of the data, as opposed to the complete data.  As a result, CPDS query responses are only guaranteed to be \emph{close} to the true answer with \emph{large} probability, where \emph{close} and \emph{large} are typically functions of structure parameters (e.g., the representation size) and properties of the data. These guarantees are stated under the assumption that the data and the internal randomness of the PDS are independent. Informally, this is tantamount to assuming that the entire collection of data is (or can be) determined \emph{before} the PDS makes any random choices.  For many PDS, this amounts to the sampling of hash functions, as the PDS operates deterministically afterward.

Recent research \cite{naor2015bloom, clayton2019, FPUV22, PatersonR22, markelon23, cryptoeprint:2024/1312, filic2025deletions} shows that, under adversarial conditions, these errors for a variety of CPDS can be exacerbated significantly, potentially undermining the reliability of systems that rely on these structures in critical or adversarial contexts. Some of these works also explore applying provable security techniques to these structures, in turn providing robust versions of these structures with respect to correctness in adversarial environments. 

A distinct subset of probabilistic data structures, which ensures correctness (and hence are not compressing) while offering fast probabilistic runtime guarantees, have received considerably less attention in the literature. Existing security analyses, such as those addressing the robustness of hash tables \cite{CrosbyW03, aumasson2012hash, bar2007remote, eckhoff2009hash,klink2011efficient,bottinelli2025hash} and skip lists \cite{nussbaum2019skiplist}, provide valuable insights but lack formal adversarial models and rigorous security analyses. Due to their runtime properties, we refer to these as \emph {probabilistic skipping-based data structures} (PSDS), as they inherently ``skip'' over parts of their internal structure to accelerate lookup operations.
The lack of research in this area is particularly concerning given that the studies on hash tables have already uncovered practical attacks, including methods to mount denial-of-service attacks attacks against intrusion detection systems~\cite{bar2007remote}, web application servers \cite{klink2011efficient}, and the QUIC protocol~\cite{bottinelli2025hash}.


%-------------------------------------------------------------------------------
\section{Relation to Previous Work}
\subsection{Self-Balancing and Self-Organizing Data Structures}

Although PSDS share conceptual similarities with self-balancing and self-organizing data structures, they differ fundamentally in their guarantees and methodological approach. 
Notably, self-organizing data structures have been extensively analyzed under adversarial models where input sequences are deliberately constructed to degrade performance, whereas the corresponding analysis for PSDS against adaptive adversaries remains a significant open problem. Similarly, self-balancing data structures have been studied extensively under worst-case analyses that inherently account for adversarial strategies.

\emph{Self-organizing data structures}~\cite{albers2005self}, whether randomized or deterministic, dynamically adjust their internal ordering of elements to optimize performance based on a given (potentially adversarial) sequence of input requests. For instance, self-organizing lists may employ the move-to-front heuristic, where accessed elements are relocated to the front of the list, or the transpose method, where elements swap positions with their predecessors when accessed. Similarly, splay trees~\cite{sleator1985self} rotate frequently accessed nodes closer to the root to reduce future access times. This approach has been shown to be challenging in adaptive adversarial settings, with (randomized) self-organizing lists incurring a cost at least three times that of the optimal reordering strategy \cite{reingold1994randomized}. 

\emph{Self-balancing} data structures, such as Red-Black trees~\cite{bayer1972symmetric} and AVL trees~\cite{adel1962algorithm}, \emph{deterministically} ensure an upper-bound on node depth, thereby providing worst-case performance guarantees for search operations. This deterministic approach is also exemplified by the deterministic skip list~\cite{munro1992deterministic}, which enforces an optimal structure by carefully promoting inserted nodes and their neighborhoods to appropriate levels. While these structures guarantee bounded search path lengths (even in adversarial settings), they require complex re-balancing mechanisms. In steep contrast, PSDS, such as the treap~\cite{seidel1996randomized} and the original skip list~\cite{pugh}, offer comparable expected performance, achieved through simple, probabilistic updating mechanisms. This presents a clear trade-off: deterministic structures provide absolute performance guarantees at the cost of implementation complexity, while probabilistic alternatives offer simplicity, albeit, with only probabilistic guarantees. In this work, we investigate whether we can maintain the implementation simplicity of probabilistic data structures while preserving their performance guarantees even in adversarial settings.

\subsection{Complexity Attacks Against Probabilistic Skipping-Based Data Structures}

This section provides a concise overview of so-called \emph{complexity attacks} targeting PSDS. Previous research has identified clear vulnerabilities in hash tables and skip lists, but these works lack formal security analysis and rigorous proofs of security when potential mitigations are put forth. Hash tables have received the most attention, while skip lists have been addressed (to our knowledge) in only a single paper in this context. Further, to our knowledge, no prior work has examined complexity attacks against treaps. This absence is consistent with our finding that treaps possess inherent resistance to such attacks.

\subsubsection{Hash Tables} Assuming a hash table's internal hash function has ``good'' collision-resistance properties, the amortized average-case complexity of insertions, deletions, and look-ups is~$O(1)$. For these efficiency reasons, hash tables are widely used in many applications such as implementing associative arrays~\cite{mehlhorn2008hash} and sets~\cite{blandy2021programming} in many programming languages, in cache systems~\cite{istvan2015hash}, as well as for database indexing~\cite{zobel2001memory}.

However, this average-case performance relies on a critical assumption: that the data inserted into a hash table is independent of the (potentially randomly selected) hash function used to map key-value pairs to buckets. This assumption fundamentally breaks down in adversarial scenarios where an attacker can deliberately craft insertions that exploit knowledge of the hash function or its outputs. Given the ubiquity of hash tables in modern computing systems, numerous researchers \cite{paxson1999bro, CrosbyW03, bar2007remote, eckhoff2009hash, klink2011efficient, aumasson2012hash,bottinelli2025hash} have investigated techniques to compromise the data structure, forcing operations to degrade from expected $O(1)$ to worst-case $O(n)$ time complexity, where $n$ represents the total number of elements in the structure. These adversarial approaches typically constitute complexity attacks that strategically engineer inputs causing multi-collisions -- deliberately exploiting hash function properties to force numerous distinct keys into identical buckets.


Crosby and Wallach~\cite{CrosbyW03} demonstrated denial-of-service attacks via complexity attacks in applications using hash tables, such as the Bro intrusion detection system~\cite{paxson1999bro}, by forcing collisions with weak, fixed hash functions. They suggested universal hashing~\cite{carter1977universal} as a mitigation, though without any formal guarantees. Klink and Walde~\cite{klink2011efficient} showed similar CPU exhaustion attacks on web servers (e.g., PHP, ASP.NET, Java), only using a single carefully crafted HTTP request. Aumasson et al.\cite{aumasson2012hash} further revealed vulnerabilities in hash tables using non-cryptographic hash functions (like MurmurHash and CityHash\cite{appleby2016smhasher}), proposing SipHash~\cite{aumasson2012hash} as a secure alternative -- which is widely adopted but lacks a holistic formal analysis as it comes to security of hash tables in adversarial settings. Complexity attacks have also been shown effective in causing denial-of-service against flow-monitoring systems~\cite{eckhoff2009hash}. Further, the use of salting was undermined by remote timing attacks~\cite{bar2007remote}. Recently, Bottinelli et al.~\cite{bottinelli2025hash} found nearly a third of QUIC implementations vulnerable to similar attacks. Despite these works and many proposed defenses, no formal framework exists for the provable security of (keyed) hash tables against adaptive adversaries. We address this gap by introducing the first rigorous security model for this setting, along with formal proofs establishing bounds on adversarial runtime degradation.

\subsubsection{Skip Lists}
In the original skip list paper~\cite{pugh}, it is noted that it is imperative to keep the internal structure of the skip list hidden. Otherwise, adversarial users could observe the levels of individual elements and delete any element at a level greater than zero (the bottom layer). This would degenerate the structure to a simple linked list and force worst-case run time ($O(n)$) on subsequent operations after these deletions occur. 

Nussbaum and Segal~\cite{nussbaum2019skiplist} demonstrate that private internal structure alone fails to protect skip lists against this style of attack. They present a (remote) timing attack that correlates query response times with element heights, ultimately allowing adversaries to force all elements in the structure to the lowest level. Their adversarial model is notably limited: the adversary cannot access the internal skip list structure, the initial data collection is non-adversarially selected, and the original data collection must be preserved during the attack. While they propose a structure called the \emph{splay skip list} as a countermeasure, their solution lacks formal security analysis. Our work presents a significantly stronger adversarial model and provides a construction with formal security guarantees. We give an extensive commentary on~\cite{nussbaum2019skiplist} and vulnerabilities below. 

Nussbaum and Segal~\cite{nussbaum2019skiplist} show that keeping the internal structure of the skip list private is insufficient to protect against complexity attacks. We discuss their attack in more detail because it is instructive in light of how to model attacks and prove the properties of robust alternatives.
Nussbaum and Segal present a timing attack that allows an adversary to discover the levels at which specific elements reside through a series of queries and, in turn, correlate the time it takes to answer a query on a given element with the height of that element. After the heights of the elements are discovered, the simple deletion attack can be mounted. 

The specific attack they present includes several assumptions.

\begin{itemize}
    \item The size of the collection represented by the structure,~$n$, is known to the adversary,~$\advA$.
    \item Each node in the structure holds a unique value.
    \item The well-ordered universe~$\univ$ is known and is of size~$O(n)$.
    \item The runtime of the search algorithm in the structure is consistent. That is, a search for the same value will yield the same runtime each time the search is executed.
\end{itemize}

Further, their adversarial model is the following. 

\begin{itemize}
    \item $\advA$ is given a skip list containing some collection of data,~$D$ that was selected by some (non-adversarial) process. 
    \item The adversary, $\advA$ does not have access to the internal structure of the skip list at any point. $\advA$ can only interact with the structure through oracles that provide search, insertion, and deletion functionality to the structure that is under attack.
    \item After the completion of the attack,~$\advA$ is required to have altered the skip list it interacts with such that it contains the original~$D$ represented by the structure (before any adversarial interaction occurs) and the level that all (or nearly all) the elements reside at is the first.  
\end{itemize}

The attack in this setting works by first running the timing attack to discover the level at which the elements in the structure exist (and, on the first iteration, which elements from~$\univ$ are present in the structure). Then all elements with a level greater than zero (exist at high level than the initial later) are removed. This set of removed elements are reinserted. These steps are repeated until (nearly) all the elements in the structure reside at level zero and the original collection represented by the structure is conserved -- thereby, degrading the representation of this collection to (nearly) a flat singly-linked list. 

As a countermeasure, the splay skip list structure is presented~\cite{nussbaum2019skiplist}.  The approach is to swap the levels of certain elements during a search query, thereby preventing the adversary from discovering information about the level where any particular element resides (as they are not fixed). The structure is believed to prevent the timing attack from being effective, but no formal analysis of the security of the structure is given. 

We again note that the adversarial setting that is given in~\cite{nussbaum2019skiplist} is rather limited. It assumes the adversary does not have access to the internal structure of the skip list, nor the ability to control the initial collection of data the skip represents. Further, it requires the adversary to conserve the initial data collection~$D$ that the skip list represents before any adversarial interaction occurs. We present a much stronger adversarial model in our work and a construction that satisfies this definition.

The authors propose a new structure that is believed to prevent the timing attack they present; however, as previously stated, no formal security analysis is given. 
Indeed, the splay skip list is still vulnerable to attacks, as demonstrated by the following scenario. Consider a collection~$D$ of elements represented by a splay skip list, where a total order is defined on the universe in which~$D$ resides. Suppose there exists an element~\( d \) such that~\( x_1 \leq d \leq x_2 \) for every pair of elements~\( x_1, x_2 \in D \), where~$x_1 \neq x_2$. For a specific order, $x_1 \leq d_1 \leq x_2 \leq d_2 \leq \ldots$ for~$x_i \in D$ and~$d_i \notin D$, an adversary can exploit this by conducting search queries for the intermediary elements $d_i$. 

Unlike searches for elements~$x_i \in D$, which would trigger the splay mechanism, searches for these intermediary elements~$d_i \not\in D$ bypass the splay security mechanism. The runtimes required to (not) find these intermediate nodes, however, still uniquely determine the height of elements contained in~$D$.\footnote{Compared to searching for elements~$x_1,x_2,\ldots$ as described in the original attack, the runtimes for searching~$d_1,d_2,\ldots$ only change by a constant factor (one extra step to find that the~$d_i \not\in S$) .} After the discovery of the heights of the elements contained in~$D$, the trivial deletion attack could be carried out as before.
%-------------------------------------------------------------------------------

%-------------------------------------------------------------------------------
\section{Structures we Analyze}
We give pseudocode and textual description of the probabilistic skipping-based data structures we consider in this work: hash tables, skip lists and treaps.

\subsection{Hash Tables}
\label{prelim:ht}

\begin{figure*}[!htbp]
    %	\Wider[4em]{
            \centering
            \begin{pchstack}[boxed,center,space=0.5em]
                \begin{pcvstack}[space=1ex]
    \procedure[linenumbering, headlinecmd={\vspace{.1em}\hrule\vspace{.1em}}]{$\Rep_{\key}(\setS)$}{%
                              \pcfor i \gets 1 \ \mathbf{to} \ m \pcdo \\
                                \t T[i] \gets \kwnew\; \llst\\
                            \pcfor (x,v) \in \setS \\
                            \t T \gets \Up_{\key}(T,\ins_{(x,v)})\\
                            \pcreturn T \pcskipln\\ 
                        }
                        
                       \procedure[linenumbering, headlinecmd={\vspace{.1em}\hrule\vspace{.1em}}]{$\Up_{\key}(T,\ins_{(x,v)})$}{%
                              v' \gets \Qry_{\key}(T,\qry_{x})\\
                            \pcif v' \neq \star\\
                            \t \Up_{\key}(T,\del_{x})\\
                            i \gets \hash(\key,x)\\
                            T[i].\mathsf{insert}((x,v))\\
                            \pcreturn T
                        }
                \end{pcvstack}	
                \begin{pcvstack}[space=0.45em]
                        \procedure[linenumbering, headlinecmd={\vspace{.1em}\hrule\vspace{.1em}}]{$\Up_{\key}(T,\del_{x})$}{%
                            i \gets \hash(\key,x)\\
                            T[i].\mathsf{remove}(x)\\
                          \pcreturn T\pcskipln\\ 
                        }
                        \procedure[linenumbering, headlinecmd={\vspace{.1em}\hrule\vspace{.1em}}]{$\Qry_{\key}(T,\qry_{x})$}{%
                            v \gets \star \\
                            i \gets \hash(\key,x)\\
                           v' \gets T[i].\mathsf{find}(x)\\
                            \pcif v' \neq \nlll\\
                            \t v \gets v'\\
                            \pcreturn v
                        }
                \end{pcvstack}	
            \end{pchstack}
    %	}
      \caption[Hash Table Structure.]{A possibly keyed hash-table structure $\mathrm{HT}[\hash_\key,b]$ admitting insertions, deletions, and queries for any~$k \in \univ_{\kappa}$ and its associated value~$v$. The parameters are an integer $b \geq 1$, and a keyed function $\hash: \keys\times\univ_{\kappa} \to [b]$ that maps the key part of key-value pair data-object elements (encoded as strings) to a position in the one of the table buckets~$v.T$. A particular choice of parameters gives a concrete scheme. Each bucket contains a simple linked list~$\llst$ equipped with its usual operations $\mathsf{insert}$, $\mathsf{find}$, and $\mathsf{remove}$ for insertion, searching, and deletion. If an item is not contained in the map, the distinguished symbol~$\star$ is returned.} 
      \label{fig:ht}
    \end{figure*}

We give a pseudocode description of a hash table (HT) in~\Cref{fig:ht}. Elements consist of a pair of entries $(x,v)$ of a unique index value $x$ and the value $v$. An instance of HT consists of~$b$ buckets, each containing an (initially empty) linked list~$\llst$, and a mapping~$\hash(\key,\cdot)$ from the index value $x$ to the bucket number in $[b]$. 

An index-value pair~$(x,v)$ is inserted into the HT representation by computing $\hash(\key,x){=} i$ and traversing it to the $i$-th bucket. We then check if the pair is already in the linked list~$\llst[i]$ stored and delete the prior mapping if this is the case. This is necessary since we insert elements according to the index key $x$, and the value entry $v$ may have changed in the new request. Finally, we insert the new pair into~$\llst[i]$. Likewise, a key is deleted by searching in the bucket to which it is assigned and removing the key and its associated value from the linked list~$\llst[i]$ in the bucket if this pair exists there. Traditionally, it is assumed that $i = \hash(\key,x)$, where~$\hash$ is a fast-to-compute hash function with good (enough) collision resistance properties. However, we generalize here to make the exposition cleaner and allow the mapping to depend upon secret randomness (i.e., a key~$\key$). To query a key for its associated value, the algorithm $\Qry(\qry_x)$ searches the bucket~$x$ maps to and returns the index-value pair if it exists there; otherwise, we return the distinguished null symbol~$\star$.

Hash tables are widely adopted for their $O(1)$ amortized average-case complexity for insertions, deletions, and look-ups, assuming "good" collision-resistance properties in the internal hash function. This efficiency has led to their extensive use across various applications, including implementations of associative arrays \cite{mehlhorn2008hash} and sets \cite{blandy2021programming} in programming languages, cache systems \cite{istvan2015hash}, and database indexing \cite{zobel2001memory}. However, despite their performance advantages, hash tables have inherent functional limitations—they cannot efficiently support operations that depend on order relationships between keys, such as range queries, predecessor/successor lookups, or sorted traversals, restricting their applicability in scenarios where such operations are essential.

\subsection{Skip Lists}
\label{prelim:sl}

\begin{figure*}[thp]
    %	\Wider[4em]{
            \centering
            \begin{pchstack}[boxed,center,space=0.5em]
                \begin{pcvstack}[space=0.45em]
                        \procedure[linenumbering, headlinecmd={\vspace{.1em}\hrule\vspace{.1em}}]{$\Rep_{\key}(\setS)$}{%
                            \mathsf{h} \gets \schemefont{NewNode}(m,\star)\\
                            \llst.\hdr \gets \mathsf{h}, \;
                            \llst.\lvl \gets 1\\
                            \pcfor (x,v) \in \setS \\
                            \t \llst \gets \Up_{\key}(\llst,\ins_{(x,v)})\\						
                            \pcreturn \llst
                        }
                        \procedure[linenumbering, headlinecmd={\vspace{.1em}\hrule\vspace{.1em}}]{$\schemefont{NewNode}(\ell,(x,v))$}{%
                            \pccomment{array position $0$ is reserved for a key, value pair $(x,v)$}\\
                            \pccomment{accessible via $n.\keyacc$ and $n.\valueacc$}\\
                            \pccomment{array positions $1 \ldots \ell$ are forward pointers}\\
                            \pccomment{level is accessible via $n.\lvl$}\\
                            \node \gets \kwnew\; [0,..,\ell]\\
                            \node[0] \gets (x,v)\\
                            \pcfor i \gets \ell \ \mathbf{downto} \ 1 \pcdo\\
                            \t \node[i] \gets \nlll\\
                            \pcreturn \node
                        }
                        \procedure[linenumbering, headlinecmd={\vspace{.1em}\hrule\vspace{.1em}}]{$\schemefont{RandomLevel}_{\key}(\boxed{x})$}{%
                            \boxed{\ell \gets R(\key,x,m,p)}\\
                            \boxed{\pcreturn \ell}\\
                            \ell \gets 1, \;
                            r \getsr [0,1)\\
                            \pcwhile r < p \ \mathbf{and} \ \ell < m \pcdo\\
                            \t \ell \gets \ell + 1, \; r \getsr [0,1)\\
                            \pcreturn \ell
                        }
                       \procedure[linenumbering, headlinecmd={\vspace{.1em}\hrule\vspace{.1em}}]{$\Qry(\llst,\qry_{x})$}{%
                          c \gets \llst.\hdr\\
                            \pcfor i \gets \llst.\lvl \ \mathbf{downto} \ 1 \pcdo\\
                            \t \pcwhile c[i] \neq \nlll \ \mathbf{and} \ c[i][0].\keyacc < x \pcdo \\
                            \t \t c \gets c[i]\\
                            c \gets c[1]\\
                            \pcif c \neq \nlll \ \mathbf{and} \  c[0].\keyacc = x \pcthen\\
                            \t \pcreturn c[0].\valueacc \\
                            \pcelse \\
                            \t \pcreturn \star
                        }
                \end{pcvstack}	
                \begin{pcvstack}[space=0.45em]
                        \procedure[linenumbering, headlinecmd={\vspace{.1em}\hrule\vspace{.1em}}]{$\Up_{\key}(\llst,\ins_{(x,v)})$}{%
                            \mathsf{u} \gets \kwnew \ [1,..,m] \pccomment{local array of pointers}\\
                            c \gets \llst.\hdr\\
                            \pcfor i \gets \llst.\lvl \ \mathbf{downto} \ 1 \pcdo\\
                            \t \pcwhile c[i] \neq \nlll \ \mathbf{and} \ c[i][0].\keyacc < x \pcdo \\
                            \t \t c \gets c[i]\\
                            \t u[i] \gets c\\
                            c \gets c[1]\\
                            \pcif c \neq \nlll \ \mathbf{and} \  c[0].\keyacc = x \pcthen\\
                            \t c[0].\valueacc \gets v \\
                            \t \pcreturn \llst \\
                            \pcelse\\
                            \t \ell \gets \schemefont{RandomLevel}_{\key}(\boxed{x})\\
                            \t \pcif \ell > \llst.\lvl \pcthen \\
                            \t\t \pcfor i \gets \llst.\lvl + 1 \ \mathbf{upto} \ \ell \pcdo\\
                            \t\t\t \mathsf{u}[i] \gets \llst.\hdr\\
                            \t\t \llst.\lvl \gets \ell\\
                            \mathsf{n} \gets \schemefont{NewNode}(\ell,(x,v))\\
                            \pcfor i \gets 1 \ \mathbf{upto} \ \ell \pcdo\\
                            \t \mathsf{n}[i]\gets \mathsf{u}[i][i] ,\;
                            u[i][i]  \gets \mathsf{n}\\
                            \pcreturn \llst
                        }
                        \procedure[linenumbering, headlinecmd={\vspace{.1em}\hrule\vspace{.1em}}]{$\Up(\llst,\del_{x})$}{%
                            \mathsf{u} \gets \kwnew \ [1,..,m] \pccomment{local array of pointers}\\
                            c \gets \llst.\hdr\\
                            \pcfor i \gets \llst.\lvl \ \mathbf{downto} \ 1 \pcdo\\
                            \t \pcwhile c[i] \neq \nlll \ \mathbf{and} \ c[i][0].\keyacc < x \pcdo \\
                            \t \t c \gets c[i]\\
                            \t u[i] \gets c\\
                            c \gets c[1]\\
                            \pcif c \neq \nlll \ \mathbf{and} \  c[0].\keyacc = x \pcthen\\
                            \t \pcfor i \gets 1 \ \mathbf{upto} \ c.\lvl \pcdo \\
                            \t\t \mathsf{u}[i][i] \gets c[i] \pccomment{free c}\\
                            \t \pcwhile \llst.\lvl > 1 \ \textbf{and} \ \llst.\hdr[\llst.\lvl] = \nlll \pcdo\\
                           \t \t \llst.\lvl \gets \llst.\lvl - 1\\
                            \pcreturn \llst
                        }
                \end{pcvstack}	
            \end{pchstack}
    %	}
      \caption[Skip List Structure.]{A possibly ``deterministic'' (and keyed) skip list structure $\SL[\boxed{R},m,p]$ admitting insertions, deletions, and queries for any~$x \in \univ$ for some well-ordered universe~$\univ$. The parameters are an integer $m \geq 0$ representing the maximum level of the structure, a fraction~$p \in (0,1)$ used for determining an element's random level, and, if using the deterministic version of the structure, a keyed function $R: \keys \by \univ \by \mathbb{Z}^{+} \by (0,1) \to [m]$ that maps an element to a level in accordance with the distribution imposed by~$m$ and~$p$. A concrete scheme is given by a particular choice of parameters. Subroutines used by the deterministic version of the structure appear in the boxed environment. 
      } 
      \label{fig:sl}
    \end{figure*}

In~\Cref{fig:sl}, we give a pseudocode description of the skip list (\SL). \SL \ maintains an ordered collection of data that allows for average-case runtime~$O(\log n)$ for search, insertions, and deletions (where the size of the represented collection is~$n$). The structure is maintained as a hierarchy of linked lists, with the first level containing all the elements of the collection and each higher level in the structure skipping over an increasing number of elements. Searching (as well as insertions and deletions) starts at the highest level, only moving down to lower levels as necessary. The specific elements that are skipped at each level are determined either probabilistically or deterministically (using (say) a PRF) at insertion time -- we focus on the probabilistic version of this structure in this paper. For a full structure description, we point to the original paper~\cite{pugh}. 

Skip lists provide an elegant probabilistic alternative to balanced binary search trees. They are widely deployed in industry applications -- managing millions of Discord server members~\cite{discord}, storing data in Apache Web Servers~\cite{apache}, and indexing SingleStore databases~\cite{singlestore}. Unlike hash tables, skip lists efficiently support range queries, ordered traversals, and predecessor/successor operations, making them valuable for various applications~\cite{quantumwalk, skabnet, InPlaceKV}.

\subsection{Treaps}
\label{prelim:tr}

\begin{figure*}[thp]
    %	\Wider[4em]{
            \centering
            \begin{pchstack}[boxed,center,space=0.5em]
                \begin{pcvstack}[space=0.45em]
                        \procedure[linenumbering, headlinecmd={\vspace{.1em}\hrule\vspace{.1em}}]{$\Rep_{K}(\setS)$}{%
                            \ttree.\rt \gets \nlll \\
                            \pcfor (x,v) \in \setS \pcdo \\
                            \t \ttree \gets \Up_{K}(\ttree,\ins_{(x,v)})\\						
                            \pcreturn \ttree
                        }
                        \procedure[linenumbering, headlinecmd={\vspace{.1em}\hrule\vspace{.1em}}]{$\schemefont{RandomPriority}_{K}(\boxed{x})$}{%
                            \boxed{p \gets R(K,x)}\\
                            \boxed{\pcreturn p}\\
                            p \getsr (0,1)\\
                          \pcreturn p
                        }
                        \procedure[linenumbering, headlinecmd={\vspace{.1em}\hrule\vspace{.1em}}]{$\schemefont{NewNode}((x,v),p)$}{%
                            \pccomment{array position $0$ is reserved for a key, value pair $(x,v)$}\\
                            \pccomment{accessible via $n.\keyacc$ and $n.\valueacc$}\\
                            \pccomment{array positions $2, 3$ are child pointers and $1$ is priority}\\
                            \node \gets [(x,v),p,\nlll,\nlll]\\
                          \pcreturn \node
                        }
                       \procedure[linenumbering, headlinecmd={\vspace{.1em}\hrule\vspace{.1em}}]{$\Qry(\ttree,\qry_{x})$}{%
                          \ttree.\rt \gets \Qry^{\text{rec}}(\ttree.\rt,\qry_{x}) \\
                          \pcreturn \ttree
                        }
                       \procedure[linenumbering, headlinecmd={\vspace{.1em}\hrule\vspace{.1em}}]{$\Qry^{\text{rec}}(c,\qry_{x})$}{%
                          \pcif c = \nlll \pcthen \\
                            \t \pcreturn \star \\
                          \pcif c[0].\keyacc = x \pcthen \\
                            \t \pcreturn c[0].\keyacc \\
                            b \gets (x > c[0].\keyacc)\\
                            \pcreturn \Qry^{\text{rec}}(c[2+b],\qry_{x})
                        }
                        \procedure[linenumbering, headlinecmd={\vspace{.1em}\hrule\vspace{.1em}}]{$\schemefont{Rotate}(c,b)$}{%
                             \tmp \gets c[2+b][3-b] \\
                             c[2+b][3-b] \gets c \\
                             c[2+b] \gets \tmp \\
                            \pcreturn \tmp
                        }
                \end{pcvstack}	
                \begin{pcvstack}[space=0.45em]
                        \procedure[linenumbering, headlinecmd={\vspace{.1em}\hrule\vspace{.1em}}]{$\Up_{K}(\ttree,\ins_{(x,v)})$}{%
                            \ttree.\rt \gets \Up^{\text{rec}}_{K}(\ttree.\rt,\ins_{(x,v)}) \\
                            \pcreturn \ttree
                        }
                        \procedure[linenumbering, headlinecmd={\vspace{.1em}\hrule\vspace{.1em}}]{$\Up^{\text{rec}}_{K}(c,\ins_{(x,v)})$}{%
                            \pcif c = \nlll \pcthen \\
                             \t p \gets \schemefont{RandomPriority}_{K}(\boxed{x})\\
                            \t \pcreturn \schemefont{NewNode}((x,v),p) \\
                            \pcif c[0].\keyacc = x \pcthen \\
                            \t c[0].\valueacc \gets v \\
                            \t \pcreturn c \\
                            b \gets (x > c[0].\keyacc) \\
                            c[2+b] \gets \Up^{\text{rec}}_{K}(c[2+b],\ins_{(x,v)}) \\ 
                            \pccomment{maintain MIN Heap property}\\
                            \pcif c[1] > c[2+b][1] \pcthen \\
                            \t c \gets \schemefont{Rotate}(c,b) \\
                            \pcreturn c
                        }
                        \procedure[linenumbering, headlinecmd={\vspace{.1em}\hrule\vspace{.1em}}]{$\Up_{K}(\ttree,\del_{x})$}{%
                            \ttree.\rt \gets \Up^{\text{rec}}_{K}(\ttree.\rt,\del_{x}) \\
                            \pcreturn \ttree
                        }
                        \procedure[linenumbering, headlinecmd={\vspace{.1em}\hrule\vspace{.1em}}]{$\Up^{\text{rec}}(c,\del_{x})$}{%
                            \pcif c = \nlll \pcthen \\
                            \t \pcreturn \nlll \\
                            \pcif c[0].\keyacc = x \pcthen \\
                            \pccomment{Remove node} \\
                            \t \pcif c[2]=\nlll \ \mathbf{and} \ c[3]=\nlll \pcthen \\
                            \t \t \pcreturn \nlll \\
                            \t \pcif c[2]=\nlll \pcthen \\
                            \t \t \pcreturn c[3] \\
                            \t \pcif c[3]=\nlll \pcthen \\
                            \t \t \pcreturn c[2] \\
                            \pccomment{Rotate node down before removing} \\
                            \t b \gets c[3][1] > c[2][1] \pcthen \\
                            \t c \gets \schemefont{Rotate}(c,b) \\
                            \t c[3-b] \gets \Up^{\text{rec}}_{K}(c[3-b],\del_{x})) \\
                            \pcelse \\
                            \t b \gets (x > c[0].\keyacc) \\
                            \t c[2+b] \gets \Up^{\text{rec}}_{K}(c[2+b],\del_{x})) \\ 
                            \pcreturn c
                        }
                \end{pcvstack}	
            \end{pchstack}
    %	}
      \caption[Treap Structure.]{A possibly ``deterministic'' (and keyed) MIN treap structure $\TR[\boxed{R}]$ admitting insertions, deletions, and queries for any~$x \in \univ$ for some well-ordered universe~$\univ$. The parameter is a keyed function $R: \keys \by \univ \to (0,1)$ that assigns an element a random priority. Subroutines used by the deterministic version of the structure appear in the boxed environment. Let $\schemefont{MinPrioChild}(c)$ denote the function that returns the child index (0 or 1) of node $c$ with the minimum priority, or null if $c$ has no children.} 
      \label{fig:treap}
\end{figure*}

In~\Cref{fig:treap} , we give a pseudocode description of the treap (\TR).
A treap~\cite{aragon1989randomized} combines the algorithms of a binary search tree (BST) and a heap and achieves an expected height of $O(\log n)$ \cite{aragon1989randomized}.
Inserting a node into a treap works analogously to a BST, but the node gets assigned an additional random priority value.
Subsequently, the algorithm rotates the tree to maintain a heap order amongst the priority values without affecting the key ordering.
For instance, in a MIN heap, the parent nodes are guaranteed lower priority values than their children.
Intuitively, when interpreting the priority values as timestamps, the resulting treap will correspond to a binary search tree in which all nodes have been inserted in random order (i.e., a randomized binary search tree). 
Deletion first rotates a node down the heap without affecting the key ordering and then removes it once it reaches a leaf position. 

Treaps efficiently support the full spectrum of binary tree operations, including range queries, predecessor/successor lookups, in-order traversals, and advanced tree operations like join, split, and union. This versatility has made treaps valuable in applications where search efficiency and ordered operations are critical requirements, such as implementing retroactive data structures~\cite{demaine2007retroactive}.
%-------------------------------------------------------------------------------

%-------------------------------------------------------------------------------
\section{Unifying Probabilistic Skipping-Based Data Structures}
Informally, one can think of a probabilistic skipping-based data structure as a data structure that uses some form of randomness (either fixed at initialization time or freshly sampled per operation) to distribute the underlying collection within its representation. This randomized representation is to (generally) allow for efficient search by ``skipping'' over some elements, such that the resulting expected runtime is sublinear with high probability. 

For instance, hash tables employ a hash function to ``randomly'' map elements to buckets, and therefore, one only has to search in this bucket for a desired element. Likewise, skip lists randomly assign heights to elements to facilitate ``skipping'' over a sequence of elements while performing a search. While the treap randomly assigns priority values to maintain an (approximately) balanced tree representation.  In turn, the hash table achieves non-adaptive adversarial expected runtime~$O(1)$ for insertions, deletions, and search; similarly, the skip list and treap achieve non-adversarial expected runtime~$O(\log n)$ for these operations on an ordered collection (but has other advantages such as supporting range queries). 

In contrast to compressing probabilistic data structures (e.g., Bloom filters, count-min sketches, HyperLogLogs, etc.), PSDS always return a correct~$\Qry$ response. Further, unlike self-balancing data structures (e.g., splay trees, red-black trees, sorted arrays, etc.), skipping data structures do not require complex update mechanisms to maintain favorable representations. That is, under non-adversarial conditions, using randomness is sufficient to facilitate efficient operational runtimes (with high probability) without the overhead of complex and potentially expensive rebalancing algorithms. 

While this provides an intuitive notion of a skipping-based data structure, it fails to provide a formal or constructive definition. Therefore, let us consider the following. Take a hash table, whose representations are built over a size~$n$ set of elements (index keys) from the domain~$\{0,1\}^\secpar$ by running them each through a hash function and putting them into a bucket depending on the output of this hash function. Under the assumption that the hash function is uniform and the non-adversarial assumption that the set of elements is selected uniformly at random from the universe of all elements, then the elements in the table can be viewed as an (unordered) sequence of i.i.d. random variables. That is, we can decompose a hash table's representation as~$B_1,B_2,\ldots,B_N$ where~$\forall i \in [n]: B_{i} \sim \mathsf{U}(\{1,2,\ldots,b\})$, where~$b$ is the number of buckets for the particular structure.

For a skip list, we can take a similar view. Here, we again assume that a skip list represents a size~$n$ set of elements from the domain~$\{0,1\}^\secpar$. Additionally, we assumed that the set is well-ordered. Under the non-adversarial assumption that all updates are made uniformly at random from the universe of all elements, the representation can be viewed as a sequence of ordered i.i.d.~random variables (again, in the adaptive adversarial setting independence of these random variables does not necessarily hold). We can decompose the skip list representation as~$H_1,H_2,\ldots,H_N$ where~$\forall i \in [n] : H_{i} \sim \mathsf{G}(p)$ for the geometric distribution, where~$p$ is the probability parameter of the structure. That is, a skip list can be viewed as the ordered sequence of its elements heights. The sequence of random variables~$H_1,H_2,\ldots,H_N$ (heights) is sorted according to the order of the keys in the representation. Similarly, one can decompose the treap representation as~$P_1,P_2,\ldots,P_N$ where~$\forall i \in [n] : P_{i} \sim \mathsf{U}([0,1])$.  This sequence of random variables represents the priority of elements in the treap, and the sequence is again ordered by the keys in the representation. That is, treaps can be viewed as a binary search tree where the order of insertion is determined by the randomly sampled priorities~\cite{seidel1996randomized}. 

With this intuition built, we arrive at our definition for probabilistic skipping-based data structures. 

\begin{definition}[Probabilistic Skipping-Based Data Structure]
\label{def:sbds}
A probabilistic skipping-based data structure that represents a size~$n$ collection of elements from the domain~$\{0,1\}^\lambda$ is a data structure whose representation can be decomposed as a sequence of identically distributed random variables from some distributions~$\mathcal{X}$. This sequence is either unordered (for data structures representing unordered data, like hash tables) or implicitly ordered by some well-defined ordering over the domain of the underlying collection (as is the case for ordered data structures, like skip lists and treaps). 
\end{definition}

This definition offers a few key advantages. First, from an attack perspective, it helps us formally specify the necessary conditions for an adversary to succeed in our security game. For hash table attacks, this means forcing a large portion of the discrete uniform random variables $B_1, B_2, \ldots, B_n$ to be equal -- a condition any successful attack strategy must achieve to degenerate the data structure. Additionally, it allows us to precisely differentiate between adaptive and non-adaptive adversarial capabilities. When decomposing a skip list into geometric random variables $H_1, H_2, \ldots, H_N$ (sorted according to key order), an adaptive adversary can observe previous outcomes and strategically insert a new $H_i$ at any position in the sequence, thereby creating dependencies among the variables. In contrast, a non-adaptive adversary cannot observe previous geometric random variable outcomes, resulting in a final sequence $H_1, H_2, \ldots, H_N$ that maintains independence among the sequence of random variables. 

Second, this stochastic formalization enables the application of well-established probabilistic techniques to derive tight bounds on adversarial success probabilities: balls-and-bins analysis for hash tables and martingale-based arguments for skip lists and treaps. Finally, for researchers looking at different PSDS from the ones we consider, it allows for generalization of our robust data structures: proving security for one structure characterized by a particular sequence of identically distributed random variables allows us to transfer robustness techniques to other structures of the same type. Though specific structural details may prevent exact technique transfer, this approach should inform effective general strategies.

\subsection{Timing Side Channels}
\label{sec:side}

PSDS share a critical vulnerability: their runtime variation for distinct queries directly reveals information about their internal structure. This inherent timing side-channel has been successfully exploited in attacks against hash tables with (secret) salts~\cite{bar2007remote} and skip lists~\cite{nussbaum2019skiplist}. For treaps, this vulnerability also manifests, as runtime correlates with node depth, potentially exposing the complete internal structure when combined with the ordering of the inserted elements. While remote attackers might face challenges like network latency in precisely measuring timing differences, recent research demonstrates that timing side-channels can be exploited with remarkable precision -- as shown in \cite{ji2025your}, where researchers recovered an AES key from a Bluetooth chip's hardware accelerator.

Implementing enforced constant-time operations fails as a solution, as this would require the data structure to always operate at worst-case (linear) time, defeating the purpose of using these efficient structures. Similarly, making the data structure oblivious to prevent information leakage has significant limitations. Such approaches are inherently fragile -- once an adversary learns anything about the internal structure, the security guarantees collapse entirely. As aforementioned, previous attempts to prevent information leakage in skip lists~\cite{nussbaum2019skiplist} by randomly swapping elements have proven unsuccessful.

Given these considerations, we adopt a more realistic approach by considering a very strong adversarial model. We grant the adversary full access to the internal structure of the PSDS, then prove that even with this knowledge, they cannot successfully degenerate the structure. This robust security model acknowledges that side channels inevitably exist in practical implementations and builds defenses that remain effective despite full information leakage. While this represents a strong adversarial capability, we argue it better reflects real-world threat scenarios than a model that assume perfect or partial information hiding.

\subsection{Towards Robust PSDS}

We observe that two abilities allow an adaptive adversary to shape the distribution of data in a PSDS such that subsequent operations on the structures are degraded with high probability. The first is the ability to delete elements. This allows an adversary to degenerate a structure after a series of insertions by deleting unfavorable (w.r.t. to the adversary's goal) elements. The second is the ability of the adversary to influence where a particular element gets placed in the structure upon insertion. This is akin to knowing in advance which bucket an element will be inserted into in a hash table, at what position and height an element will be inserted in a skip list, or the priority an element will receive upon insertion to a treap. Therefore, we propose two inexpensive and general modifications to the base PSDS to make them robust in an adversarial setting. We will later prove these modified structures secure. 

\subsubsection{Lazy deletion} The first modification prevents the adversary from deleting (unfavorable) elements from the structure. This stultifies the ability of an adversary to perform a skip list degeneration-style attack, even with full access to the data structure's internal state. 

Removing the deletion functionality entirely from our data structure would be undesirable. Instead, we use a simple scheme that allows for removing elements without modifying the underlying structure of a PSDS that previous insertions have imposed. We achieve this by simply labeling an element as ``deleted''. For the hash table, we replace the element's label (e.g., the key-value data) with a distinguished symbol~$\diamond$ but do not modify the linked list in a hash table bucket by removing the node. For operational reasons, in the skip list and treap, we store a bit along with each node that indicates whether an element has been removed, but do not overwrite the originally inserted key with a distinguished symbol. 

This change prevents the adversary from eliminating desired skip connections in a skip list, obtaining trivial wins in our security model against a hash table (when taking the represented set to be the collection of all empty and non-empty elements), or only allowing elements to persist solely on the longest path in a treap. However, this modified deletion functionality affects the space efficiency of the structures. In later sections, we discuss approaches to ameliorating such concerns and analyze the trade-offs of these approaches. Lastly, since ``all bets are off'' when deletions are allowed, we implicitly provide security bounds that would compare to insertion-only versions of these structures in the non-adaptive case. That is, since an adaptive adversary can pathologically degrade the base structures, we enforce that using our modified deletion mechanisms never helps the adversary achieve their goal (in fact, this point is the first step of all our security proofs). 

\subsubsection{Adversarial robustness}
The second modification eliminates (to the greatest extent possible) an adversary's ability to predict element placement within data structures. All analyzed data structures require distinct security approaches for adversarial robustness, which heavily depends on how randomness is used internally. Skip lists and treaps use per-insertion randomness while preserving key-based ordering. During queries, the element's key guides traversal, although specific paths vary based on insertion-time randomization. Hash tables function fundamentally differently -- they determine bucket placement solely based on random experiment outcomes rather than element keys. This approach necessitates reproducing identical outcomes during search queries. Conventional implementations rely on public hash functions, creating a critical security vulnerability: adversaries can precalculate outcomes for elements and execute complexity attacks.

To provide adversarial robustness for hash tables, we replace public hash functions with secretly keyed primitives that effectively behave like truly random functions, preventing adversarial precalculation. Note that this approach necessitates secret key management, which presents potential implementation challenges. For skip lists, we develop an unkeyed, algorithmic approach to secure against adversarial manipulation. Despite an adversary's inability to have a priori knowledge of coin flip outcomes that determine the height of an element, skip lists remain vulnerable -- an adversary can strategically shift unfavorable random outcomes to one side of the structure, effectively placing elements with specific heights at chosen positions. We counter this by enforcing a local balance in the internal representation through a constant overhead swap operation, making such attacks exponentially more difficult.

Note that simply pre-applying a (secretly-keyed) random function to the items a skip list stores, as in the hash table mechanism, would alter the structure's ordering, rendering these data structures incapable of performing range queries, join operations, and other order-dependent functions. We therefore developed security mechanisms that maintain fundamental ordering properties while enhancing attack resilience.

Treaps, by contrast,inherently rebalance their entire structure based on the priorities of all previously inserted elements. As we will demonstrate, this property already substantially reduces an adversary's ability to place elements at positions of their choosing.
%-------------------------------------------------------------------------------

%-------------------------------------------------------------------------------
\section{A Security Model for Probabilistic Skipping-Based Structures}
Our goal is to capture the average-case run time of operations PSDS being conserved in the face of an adaptive adversary that can control the data represented by the structure. Loosely, the average-case run time of PSDS relates to how data is ``distributed'' in the representation.  For instance, an ideal hash table would distribute the elements it represents equally among the buckets. Analogously, ideal ordered structures (e.g., a skip list or a treap) would be isomorphic to a balanced tree. If a data collection was fixed, and we ignored a desire for efficiency, one could always craft an ideal representation with respect to the runtime of queries. For a hash table, one could find a hash function that equally distributes the fixed collection to its buckets. For a fixed-ordered structure, one could simply assign the heights (depths) of elements such that the shortest possible search paths are guaranteed, as with a perfectly balanced tree structure. 

However, PSDS are used in mutable settings. For this reason (and for efficiency), PSDS use some form of randomness to process updates dynamically and update their representation. Hash tables select a random hash function to map elements to buckets, and ordered PSDS employ per-operation randomization during insertion to determine an element's position in the structure --- typically through coin flips for skip lists or random priority assignments for treaps. These processes have been shown (with high probability) to yield representations of a dynamic data collection that are ``close'' to the ideal representations. Hash tables are analyzed using standard ball-and-bin arguments. Assuming a collision-resistant hash function and a load factor such that~$n \approx b$ (i.e., the size $n$ of the data collection stored is about equal to the number $b$ of buckets), it is known \cite{chawla09} that with probability~$p = 1 - \frac{1}{b}$ that at any point in time no bucket has more than~$3\frac{\log b}{\log \log b}$ entries. This maximum bucket population bounds directly corresponds with a subsequent operation's maximum insertion, deletion, or query time. Likewise, the maximum search cost path of any element queried to a skip list or treap has been shown to not exceed~$O (\log n)$ with high probability (where the exact constants are functions of the parameters of the structure).  

The above analyses are done under a strictly non-adaptive adversarial assumption. That is, these probabilistic bounds on the ``distribution'' of elements are done under the assumption that the updates and queries made to the structure do not depend on the internal randomness of the structure, the results of past operations, or the state of the representation. In the adaptive adversarial setting, this cannot be assumed. This is seen in both the hash flooding attack and the skip list degeneration attack~\cite{CrosbyW03,bar2007remote,klink2011efficient,nussbaum2019skiplist}. 
Therefore, intuitively, a robust PSDS would conserve the desired element distribution property of the structure with high probability, even in the face of an adaptive adversary. This is what we aim to capture with our formal security model below.

\begin{figure}[h]
 \centering
    

	\begin{pchstack}[boxed,center,space=0.5em]
	
	\begin{pcvstack}[space=0.45em]
		
    \procedure[linenumbering, headlinecmd={\vspace{.1em}\hrule\vspace{.8em}}]{$\Exp{\text{aapc}}_{\struct, \phi, \beta, \epsilon}(\advA)$}{%
                r \gets 0; K \getsr \mathcal{K}\\
                %\mathsf{kv} \gets \top; \mathsf{rv} \gets \top\\
                %\pcif u = 1 : \mathsf{kv} \gets K\\
                %\pcif v = 1 : \mathsf{rv} \gets \repr\\
				\mathrm{done} \getsr \advA^{\REPO,\UPO,\QRYO}\\%(\mathsf{kv},\mathsf{rv})\\
				\pcreturn \big[\frac{\phi(D,\repr)}{\beta(\mathcal{P},|D|)} \geq \epsilon\big]
		}

  \procedure[linenumbering, headlinecmd={\vspace{.1em}\hrule\vspace{.8em}}]{$\HASHO(X)$}{%
        \pcif X \not\in \mathcal{X} : \pcreturn \bot\\
        \pcif H[X] = \bot\\
        \t X[X] \getsr \mathcal{Y}\\
		\pcreturn H[X]
	}

	\end{pcvstack}

	\begin{pcvstack}[space=0.45em]

 		\procedure[linenumbering, headlinecmd={\vspace{.1em}\hrule\vspace{.8em}}]{$\REPO(C)$}{%
        \pcif r = 1 : \pcreturn \bot \\
        r \gets 1\\
		\pub \getsr \Rep_{K}(C)\\
        D \gets C\\
        %\pcif v = 0 : \pcreturn \top\\ 
		\pcreturn \pub
	}
 
		\procedure[linenumbering, headlinecmd={\vspace{.1em}\hrule\vspace{.8em}}]{$\UPO(\up)$}{%
		\pub \getsr \Up_{K}(\pub, \up)\\
		D \gets \up(D)\\
        %\pcif v = 0 : \pcreturn \top\\ 
		\pcreturn \pub
	}

	\procedure[linenumbering, headlinecmd={\vspace{.1em}\hrule\vspace{.8em}}]{$\QRYO(\qry)$}{%
		\pcreturn \Qry_{K}(\pub, \qry)
	}
	\end{pcvstack}
 \end{pchstack}


  \caption[The AAPC Security Model.]{The Adaptive Adversary Property Conservation (AAPC) security game. The experiment enforces that the adversary is only able to call~$\REPO$ once. The experiment returns the output of a predicate that returns~$1$ iff the property function~$\phi(D,\repr)$ computed over the representation the adversary interacts with is greater than~$\epsilon$-times (for some~$\epsilon > 0$) larger than some target bound~$\beta$ (that only depends on the parameters of the structure~$\mathcal{P}$ and the size of the represented data object~$|D|$). The $\HASHO$ oracle computes a random mapping $\set{X}\to\set{Y}$ (i.e., a random oracle), and is implicitly provided to $\Rep$, $\Up$ and $\Qry$ as needed.}
  \label{fig:aapc}
\end{figure}

Let~$\Pi = (\Rep,\Up,\Qry)$ be a probabilistic skipping-based  data structure. We define a notion of adversarial property conservation involving~$\Pi$, a property function~$\prop : \dataobjects \times \{0,1\}^{*} \rightarrow \mathbb{R}$, a target bound~$\beta : \mathcal{P} \times \mathbb{Z}^{+} \rightarrow \mathbb{R}$, and a threshold $\epsilon \in \mathbb{R},\epsilon > 0$.   


\begin{figure}[h]
    %	\Wider[4em]{
            \centering
            \begin{pchstack}[boxed,center,space=0.5em]
                \procedure[linenumbering, headlinecmd={\vspace{.1em}\hrule\vspace{.8em}}]{HT Maximum Search Path: $\phi(D,\repr)$}{%
                    e \gets 0\\
                    \pcfor i \gets 1 \ \mathbf{to} \ m\\
                    \t \ell \gets \mathsf{length}(T[i])\\
                    \t \pcif \ell > e\\
                    \t \t e \gets \ell\\
                    \pcreturn e
                }
            \end{pchstack}
    %	}
      \caption[HT Maximum Search Path.]{The HT Maximum Search Path function~$\prop : \dataobjects \times \{0,1 \}^{*} \rightarrow  \mathbb{R}$. The function iterates through all $m$ buckets, returning the bucket with the greatest population, which is equivalent to the longest search path in the table.
      } 
      \label{fig:ht-pop}
\end{figure}

    \begin{figure}[h]
    %	\Wider[4em]{
            \centering
            \begin{pcvstack}[boxed,center,space=0.5em]
                \procedure[linenumbering, headlinecmd={\vspace{.1em}\hrule\vspace{.8em}}]{TR Maximum Search Path: $\phi(D,\repr)$}{%
                    \pcreturn \phi^{\text{rec}}(\ttree.\rt, 0) 
                }
                \procedure[linenumbering, headlinecmd={\vspace{.1em}\hrule\vspace{.1em}}]{$\phi^{\text{rec}}(n,e)$}{%
                    \pcif n = \nlll \pcthen \\
                     \t \pcreturn  \\
                    e_1 \gets \phi^{\text{rec}}(n[2],e+1) \\
                    e_2 \gets \phi^{\text{rec}}(n[3],e+1) \\
                    \pcreturn \max(e_1, e_2)
                }
            \end{pcvstack}
    %	}
      \caption[Treap Maximum Search Path.]{The TR Maximum Search Path function~$\prop : \dataobjects \times \{0,1 \}^{*} \rightarrow  \mathbb{R}$. The function performs an in-order traversal for all elements~$d \in D$, returning the longest search path cost among them.
      } 
      \label{fig:t-cost}
\end{figure}
    
\begin{figure}[h]
    %	\Wider[4em]{
            \centering
            \begin{pchstack}[boxed,center,space=0.5em]
                \procedure[linenumbering, headlinecmd={\vspace{.1em}\hrule\vspace{.8em}}]{SL Maximum Search Path: $\phi(D,\repr)$}{%
                    m \gets 0\\
                    \pcfor d \in D \\
                    \t \ell \gets 0, \;
                    c \gets \llst.\hdr\\
                    \t \pcfor i \gets \llst.\lvl \ \mathbf{downto} \ 1 \pcdo\\
                    \t \t \pcwhile c[i] \neq \nlll \ \mathbf{and} \ c[i][0].\keyacc < d \t \pcdo \\
                    \t \t \t c \gets c[i], \;
                    \ell \gets \ell + 1\\
                    \t c \gets c[1], \;
                    \ell \gets \ell + 1\\
                    \t \pcif c \neq \nlll \ \mathbf{and} \  c[0].\keyacc = d \pcthen\\
                    \t\t \pcif \ell > m \pcthen \\
                    \t\t\t m \gets \ell\\
                    \pcreturn m
            }
            \end{pchstack}
    %	}
      \caption[SL Maxium Search Path.]{The SL Maximum Search Path functions~$\prop : \dataobjects \times \{0,1 \}^{*} \rightarrow  \mathbb{R}$. The function iterates through all elements~$d \in D$, returning the longest search path cost among them. Our function only computes rightward pointer traversals, as downward movements equate to a simple array lookup. 
      } 
      \label{fig:sl-cost}
\end{figure}


A property function~$\prop$ takes as input the data object~$D \subseteq \dataobjects$  represented by~$\repr$ (the representation the adversary produces during its execution) and the representation~$\repr$ itself and outputs a value that indicates the concrete property for the given adversarially chosen data collection and corresponding representation. This function represents the desired property one would like to conserve. For all structures of interest, this is the maximum search path cost over all elements~$d \in D$~\footnote{This property sufficiently captures the search path cost of any~$d$ in the universe of all possible elements, as a search for an element not in the representation terminates with at most one more pointer traversal compared to any element in the representation.}. The intuition is that a complexity attack is deemed successful precisely when it significantly increases the maximum search path cost; therefore, a robust data structure must maintain nearly equivalent worst-case performance (with high probability) regardless of adversarial manipulation. We give the exact property function for a hash table in ~\Cref{fig:ht-pop},  for a skip list in ~\Cref{fig:sl-cost} , and for a treap in~\Cref{fig:t-cost}.

A target bound~$\beta$ takes as input the structure parameters~$\mathcal{P}$ (e.g., the number of buckets for a given hash table) and a size of the represented data object~$|\mathcal{D}|$ (denoted~$n$ below), and outputs the resulting bound value. We choose a target bound such that it corresponds to the known non-adaptive bound for the property we want to conserve. For the hash table maximum search path cost, this is~$\beta(\langle b \rangle, n) = 3\frac{\log b}{\log \log b}$. For the skip list and treap maximum search path, we chose~$\beta (\langle p,m \rangle, n) = c \log_{1/p}(n)$ (for a small constant~$c$), and $\beta (\langle  \rangle, n) = 2\lg(n)+1$, respectively, as these are the (blunt) non-adversarial expected search path lengths~\cite{pugh,reingold1994randomized}.

We give this notion of adversarial property conservation in~\Cref{fig:aapc}. The Adaptive Adversary Property Conservation (AAPC) experiment aims to capture an adversary's ability to adaptively craft a representation~$\repr$ of some dynamic and adversarially decided data object~$D$, such that when the property function~$\phi$ is computed, the ratio of its output to the target bound's output is large (to win the experiment this ratio needs to exceed~$\epsilon$). As the properties (and their accompanying target bounds) measure how data elements are distributed in a particular representation (and bound how they are distributed in the non-adaptive setting), this notion directly translates to an adversary's ability to disrupt the expected runtime of a data structure's operations. 

The AAPC experiment begins by setting a parameter~$r=0$ and selecting a key~$K$ from the key space~$\mathcal{K}$. For unkeyed hash tables (insecure) and non-deterministic versions of the ordered PSDS, the key space is the empty set. The adversary is then allowed to instantiate the data structure with any initial data object~$C$ (including the empty data object) via the~$\REPO$ oracle and receives back the resulting representation. We enforce that the adversary is only allowed to call~$\REPO$ once via the parameter~$r$. This is to disallow the adversary from leveraging past information from a data structure that is keyed with the same key~$K$ to trivially win the game. That is, keyed hash tables and deterministic PSDS must sample a fresh random key to guarantee security. 

The adversary is then allowed to make any sequence of~$\UPO$ and~$\QRYO$ calls. Upon each update, we also update the internal data object~$D$ kept by the experiment, as this is used for computing~$\phi$ and~$\beta$. After each update, the updated representation~$\repr$ is returned to the adversary. Thus, the notion of security we propose is quite strong in that it allows an adversary to have complete access to the structure's internals during its execution (as discussed in Section~\ref{sec:side}). The only information kept from the adversary is the secret key (in the case the structure relies on one). This further makes calls to~$\QRYO$ unnecessary, as the adversary entirely determines the underlying collection represented by the structure and has access to the internal representation at all times. 

The adversary ends its execution by announcing \textbf{done} or is implicitly done when it exhausts its~$\UPO$ budget (the number of updates they are allowed to make). The experiment concludes by outputting a bit that determines whether the adversary has successfully met the winning condition.

With this intuition built, we give our succinct formal definition of security. 

\begin{definition}[$(\phi,\beta,\epsilon,\delta,t)$-Conserved]\label{def:aapc}
    We say a skipping-based probabilistic data structure $\Pi$ is $(\phi,\beta,\epsilon,\delta,t)$-conserved if the advantage of an AAPC-adversary~$\advA$ running in time~$t$ is less-than-or-equal to~$\delta$ for some property function~$\phi$, some target bound~$\beta$, some~$\epsilon \in \mathbb{R}, \epsilon > 0$, and some~$\delta \in [0,1)$. More precisely, we say the structure is $(\phi,\epsilon,\beta,\delta,t)$-conserved iff,
    \[ \Adv{\text{aapc}}_{\struct, \phi, \beta,\epsilon}(\advA) = \Pr[\Exp{\text{aapc}}_{\struct, \phi, \beta,\epsilon}(\advA) =1] \leq \delta
    \]
    and write $\Adv{\text{aapc}[u,v]}_{\struct, \phi, \beta, \epsilon}(t,q_Q,q_U,q_H)$ as the maximum advantage of any AAPC-adversary running in~$t$ time steps and making~$q_Q$ calls to~$\QRYO$,~$q_U$ calls to~$\UPO$, and~$q_H$ calls to~$\HASHO$ in the ROM. We are interested in ensuring  $\Adv{\text{aapc}}_{\struct, \phi, \beta,\epsilon}(t,q_Q,q_U,q_H) \leq \delta$.
\end{definition}

%-------------------------------------------------------------------------------

%-------------------------------------------------------------------------------
\section{Robust Hash Tables}\label{sec:ht}
\subsection{Insecurity Of Standard Hash Tables}

\paragraph{Unkeyed Hash Tables}

Consider a standard hash table instantiated with a fixed and publicly known hash function. A simple pre-computation attack will trivially win our security experiment (with the experiment parameters the same as in Theorem~\ref{thm:rhtsr}) with probability one (assuming the ability to make sufficiently many local hash computations). An adversary can sample index keys from the universe and compute the bucket they will map to by using the public hash function (assuming the parameters of the structure are known). The adversary can select a target bucket and insert index keys (with some arbitrary value) iff they map to this target bucket. In this way, an adversary can ensure that all elements go to a single bucket, causing a linear overhead when searching for an element in this bucket.

\paragraph{Keyed Hash Tables with Deletions}

Consider a hash table where we replace a public hash function with a secretly keyed primitive, like a PRF. Our security game also yields a simple strategy for an adversary to win our game with a high probability if the hash tables support deletions in the usual way. The adversary selects a target bucket. Then it samples keys from the universe (along with arbitrary values for these keys) and inserts them into the table. Observing the state of the table after each insertion, the adversary deletes the element unless it has been inserted in the target bucket. At the end of the adversary's execution, the hash table will only have elements that reside in a single bucket. For this reason, we do not allow adversaries to make deletions that actually remove elements from the hash table and compare our adversarial results to a standard ball-in-bins result that assumes no deletions. 

While an attack of this nature may seem vacuous and an artifact of our security experiment, it is designed to capture something more complex. Consider if you could guarantee that the state of the hash table could remain hidden during the adversary's execution. Then, it seems intuitive that just keying the structure would result in robust construction per our security definition. However, as evidenced by side-channel attacks against hash tables~\cite{bar2007remote}, it is nearly impossible to guarantee that the internal structure of the hash table remains entirely hidden. Therefore, we continually leak the entire state of the structure to the adversary during its execution to emulate the best possible side channel (as detailed in \Cref{sec:side}). 

\subsection{A Robust Construction}

\begin{figure*}[!htbp]
    %	\Wider[4em]{
            \centering
            \begin{pchstack}[boxed,center,space=0.5em]
                \begin{pcvstack}[space=0.45em]
                        \procedure[linenumbering, headlinecmd={\vspace{.1em}\hrule\vspace{.1em}}]{$\Rep_{\key}(\setS)$}{%
                              \pcfor i \gets 1 \ \mathbf{to} \ m \pcdo \\
                                \t T[i] \gets \kwnew\; \llst\\
                            \pcfor (x,v) \in \setS \\
                            \t T \gets \Up_{\key}(T,\ins_{(x,v)})\\
                            \pcreturn T \pcskipln\\ 
                        }
                       \procedure[linenumbering, headlinecmd={\vspace{.1em}\hrule\vspace{.1em}}]{$\Up_{\key}(T,\up_{(x,v)})$}{%
                              v \gets \Qry_{\key}(T,\qry_{x})\\
                            \pcif v \neq \star\\
                            \t \Up_{\key}(T,\del_{x})\\
                            i \gets \hash(\key,x)\\
                            T[i].\mathsf{ireplace}((x,v),(\diamond,\diamond))\\
                            \pcreturn T
                        }
                \end{pcvstack}	
                \begin{pcvstack}[space=1ex]
                        \procedure[linenumbering, headlinecmd={\vspace{.1em}\hrule\vspace{.1em}}]{$\Up_{\key}(T,\del_{x})$}{%
                            v \gets \Qry_{\key}(T,\qry_{x})\\
                            \pcif v \neq \star\\
                            \t i \gets \hash(\key,x)\\
                            \t T[i].\mathsf{replace}((x,v),(\diamond,\diamond))\\
                            \pcreturn T\pcskipln\\ 
                        }
                        \procedure[linenumbering, headlinecmd={\vspace{.1em}\hrule\vspace{.1em}}]{$\Qry_{\key}(T,\qry_{x})$}{%
                            v \gets \star \\
                            i \gets \hash(\key,x)\\
                            v' \gets T[i].\mathsf{find}(x)\\
                            \pcif v' \neq \nlll\\
                            \t v \gets v'\\
                            \pcreturn v
                        }
                \end{pcvstack}	
            \end{pchstack}
    %	}
      \caption[A Robust Hash Table.]{
      A robust hash table in the AAPC security model. It is an explicitly keyed hash-table structure $\mathrm{RHT}[\hash,b]$ admitting insertions, modified deletions, and queries for any~$k \in \univ_{\kappa}$ and its associated value~$v$. The parameters are an integer $b \geq 1$, and a keyed function $\hash: \keys\by\univ_{\kappa} \to [b]$ that maps the key part of key-value pair data-object elements (encoded as strings) to a position in the one of the table buckets~$v.T$. A particular choice of parameters gives a concrete scheme. Each bucket contains a simple linked list~$\llst$ equipped with its usual operations. We define the~$\mathsf{replace}$ operation of~$\llst$, such that if it finds an item with~$(x,v)=(\diamond,\diamond)$ during its internal search, the item to be inserted is written in this location; otherwise a regular insertion occurs. If an item is not contained in the map, the distinguished symbol~$\star$ is returned. 
      } 
      \label{fig:robust_ht}
\end{figure*}

We give a robust hash table construction in Figure~\ref{fig:robust_ht}. The robust hash table requires that a keyed mapping function~$R$ is used. Concretely, this can be instantiated as PRF that is then mapped to~$b$ (by, say, taking the output of the PRF modulo~$b$). In particular, SipHash~\cite{aumasson2012hash} provides performance that is comparable to traditionally used non-cryptographic hash functions~\cite{PatersonR22}. We also use our modified deletion scheme. The deletion functionality simply relabels the key-value pair to be deleted as~$(\diamond,\diamond)$, where~$\diamond$ is a distinguished symbol. The insertion functionality changes such that if an element to be inserted can overwrite a linked list node containing~$(\diamond,\diamond)$, it does; otherwise, a normal insertion occurs. The query functionality remains unchanged.

We will now state and prove a formal security theorem and prove the robust hash table construction secure in the AAPC model.

\begin{theorem}[Robust Hash Table AAPC Security Result]\label{thm:rhtsr}
    Let~$\Pi$ be our robust hash table from ~\Cref{fig:robust_ht}, using PRF~$F$ to map elements to buckets. For integers~$q_U,q_Q,q_H,t \geq 0$ such that $q_U = b$ (i.e., $b$ is the number of buckets in the hash table~$\Pi$), it holds that~$\Pi$ is~$(\phi,\beta,\epsilon,\delta,t)$-conserved with $\phi$ being the HT Maximum Bucket Population function (~\Cref{fig:ht-pop}), $\beta = 3 \frac{\log b}{\log \log b}$, $\epsilon = 1$, and $\delta = (\frac{1}{n} + \Adv{\text{prf}}_{F}(O(t),b+q_Q))$.
\end{theorem}

\begin{proof}
    Observe that the modified insertion and deletion procedures ensure that once an element is inserted into a bucket, it cannot actually be removed but rather only relabeled (either to $(\diamond,\diamond)$ or a newly inserted key-value pair). Observe that an optimal adversary never makes deletions for our construction, as our modified deletion procedure ensures this cannot possibly add to the maximum search path cost. Thus, we start with a game that assumes the adversary never makes deletions, and the proof follows from a simple hybrid argument.
    
    We start with a game~$\game_0$ that is that the AAPC security game instantiated with our robust hash table~$\Pi$ using PRF~$F$, property function~$\phi$ as the HT Maximum Bucket Population function (~\Cref{fig:ht-pop}), and target bound~$\beta = 3 \frac{\log b}{\log \log b}$. As indicated by the theorem statement, we assume that the adversary cannot insert more than~$q_U = b$ distinct elements into the table and, from above, never makes a deletion. In this game, the number of times~$F$ is evaluated on distinct inputs bounded by the adversary's resource budget. Calls to~$\UPO$ (also implicitly used by~$\REPO$) call~$F$ once. Calls to~$\QRYO$ also call~$F$ once. Thus, when executed with~$\advA$, game~$\game_0$ makes at most~$Q = b + q_Q$ queries to~$F$.
    
    Let~$\game_1$ be identical to~$\game_0$ except we use truly random sampling (modeled in the ROM) in place of the PRF. If~$\advA$ cannot distinguish~$F$ from a random function. Then, these games are indistinguishable from the adversary's perspective. We build a~$O(t)$-time PRF distinguishing adversary~$\advB$ making at most~$Q$ queries to its oracle such that
    \begin{equation}
        \Adv{\text{prf}}_{F}(\advB) = \Pr[\game_0(\advA) = 1] - \Pr[\game_1(\advA) = 1].
    \end{equation}
    
    Adversary~$\advB^{F}$ works by executing~$\advA$ in~$\game_1$. Whenever~$\game_1$ calls~$F$, adversary~$\advB$ computes the response using its own oracle. When~$\advA$ halts, if the winning condition of~$\game_1$ is satisfied, then~$\advB$ outputs~$1$; otherwise it outputs~$0$. Conditioning on the outcome of the coin flip~$z$ in~$\advB$'s game, we have the following: 
    
    \begin{align*}
        \Adv{\text{prf}}_{F}(\advB) &= 2\Pr[\Exp{\text{prf}}_{F}(\advB = 1)] -1\\
        &= 2(\frac{1}{2}\Pr[\Exp{\text{prf}}_{F}(\advB = 1) | z =1]   \\  &+ \frac{1}{2}\Pr[\Exp{\text{prf}}_{F}(\advB = 1) | z =0])-1\\
        &= \Pr[\Exp{\text{prf}}_{F}(\advB = 1) | z =1]  + \Pr[\Exp{\text{prf}}_{F}(\advB = 1) | z =0]-1\\
        &= \Pr[\game_0(\advA) = 1] - \Pr[\game_1(\advA) = 1].
    \end{align*}
    
    Now, with~$\game_1$, we immediately have a standard insertion-only truly random balls-and-bins problem with~$\leq q_U = b$ balls being randomly thrown into~$q_U = b$ bins. We can apply the standard bound and conclude~$\phi(\cdot) \leq \beta (\cdot)$ (that is~$\frac{\phi(\cdot)}{\beta (\cdot)} \leq \epsilon = 1$) with probability~$1-\delta$ where~$\delta = (\frac{1}{b} + \Adv{\text{prf}}_{F}(O(t),Q))$. The first term comes from the standard bound and the second results from the hybrid we showed above.
\end{proof} 

To give a concrete illustration of this bound, suppose we had~$n = b = 2^{32}$ and~$\epsilon = 1$. Leveraging our results from~\Cref{thm:rhtsr}, the probability our maximum search cost path is greater than~$M = 3 \frac{\log 2^{32}}{\log\log 2^{32}} \approx 21.47$ is less than or equal to $\delta = \frac{1}{2^{32}} + \Adv{\text{prf}}_{F}(O(t),b+q_Q) \approx 2.33 \cdot 10^{-10} + \Adv{\text{prf}}_{F}(O(t),b+q_Q)$.

\subsection{Robust Hash Tables in Real World Deployments}

When initializing a hash table, there's an implicit promise to allocate enough memory for a pre-defined number of elements. If the collection grows too large and exceeds this capacity, the structure must be resized, typically by doubling the number of buckets. For a key-value pair where keys are~$x$ bits and values are~$v$ bits, we expect to allocate up to~$\alpha \cdot b \cdot x \cdot v$ bits of memory, where~$\alpha$ is the load factor defined as~$\alpha = \frac{n}{b}$, with~$n$ being the number of elements and~$b$ the number of buckets~\cite{clrs}. If the load factor exceeds a set limit, resizing is required. In our security experiment, we implicitly specify a load factor of~$\alpha \leq 1$ by setting~$q_U = b$, and in turn, never allow this load factor to be exceeded. That is, we do not consider attacks that would trigger resizing. Hence, we discuss the consequences of our robust construction in real-world deployments below by analyzing how our modifications change the frequency of required resizing. 

Consider a standard hash table with~$I$ successful insertions and~$D$ successful deletions. For resizing to be necessary, it must be that~$\frac{I-D}{b}$ has exceeded $\alpha$. At some point, before resizing is triggered, if the rate of insertions and deletions are roughly equal, a structure could persist indefinitely without resizing.

Now consider a modification where deletions merely mark elements as deleted without allowing for the possibility of being replaced by fresh insertions. That is, we do not modify the insertion procedure to replace previously deleted elements. In this scenario, resizing occurs when~$\frac{I}{b}$ has surpassed~$\alpha$, even if only a few elements are actually represented in the structure. This could occur when an adversary inserts~$\approx \alpha \cdot b$ elements, then deletes nearly all of them\footnote{Of course, if an adversary deleted all elements, it would be trivial to flush the table and reinitialize the structure.}, and finally triggers a resizing with a few subsequent fresh insertions. Although this seems wasteful, it aligns with the resizing logic since the total insertions exceed the threshold. That is, a resizing is triggered only after the total number of insertions exceeds the threshold set by~$\alpha$ (regardless if a deletion has subsequently nullified them).

We would like our robust hash table to conserve the property where deletions free space, such that $\frac{I}{b} > \alpha$ does not necessarily trigger a resizing. Thus, in addition to marking deleted elements, we also prefer replacing said deleted elements with new insertions. This is desirable in the non-adversarial case (where insertions, deletions, and queries do not depend on the internal randomness of the structure, the internal state of the structure, or past operations), as one expects freshly inserted elements will eventually replace deleted elements.

Adversarial strategies can still trigger resizing with few non-deleted elements. For example, an adversary could insert~$I = \alpha \cdot b - 1$ elements, delete all but those in the least populated bucket, and with a~$1/b$ probability, trigger resizing with only those elements in that bucket remaining. While this requires the adversary to exceed the threshold number of insertions, making it marginally problematic in practice, the collection size at the time of resizing may be smaller than the non-adversarial threshold. In sum, while adversaries can still trigger small collection resizing under certain conditions, our approach ensures the hash table is provably robust and allows it to persist for extended periods without resizing if insertions and deletions are balanced in the non-adversarial setting.

%-------------------------------------------------------------------------------

%-------------------------------------------------------------------------------
\section{Robust Skip Lists}\label{sec:skiplist}
\subsection{Insecurity of Standard Skip Lists}\label{sec:gap-attack}

As noted in~\cite{pugh}, the heights of the elements in the skip list must be kept secret, or otherwise, the skip list can be degenerated by simply deleting all elements in the list that are not at height zero. In our security model, this attack is trivial. However, even when disallowing deletions (or using our modified deletion) functionality, an adaptive adversary can still degenerate the skip list using a powerful but subtle strategy. We call this the \emph{gap attack} and detail it next, but use an intuitive rather than a formal description.

We assume the skip list takes values from $\{0,1\}^n$, which we interpret as integers between $0$ and $2^n-1$. The gap attacker proceeds as in Figure~\ref{fig:gap}. It starts by inserting an element in the middle $M=2^{n-1}$ of the interval $[L,R]=[0,2^{n}]$. If this element gets assigned a height of $0$ in the skip list, i.e., is only inserted in the bottom list, then the attacker secures it by shifting the left bound $L$ of the interval to $M$, moving to that gap. In the other case, if the height is large, it ``gives up'' this part of the skip list and moves the right bound $R$ of the interval to $M$. Continue with the new interval $[L',R']$ of half the size until $n$ elements have been inserted.
\begin{figure}[thp]
%	\Wider[4em]{
		\centering
		\begin{pchstack}[boxed,center,space=0.5em]
            \procedure[linenumbering, headlinecmd={\vspace{.1em}\hrule\vspace{.1em}}]{Gap attack on skip list}{%
                L \gets 0, R\gets 2^{n}\\
                \pcfor i \gets 1 \ \mathbf{to} \ n\\
                \t \text{insert element $ M\gets (R+L)/2 $}\\
                \t \pcif \text{height}(M)=0 \pcthen L\gets M \ \pcelse R\gets M
            }
		\end{pchstack}
%	}
  \caption[The Gap Attack.]{The gap attack on skip lists, inserting $n$ elements from $\{0,1\}^n$.   
  } 
  \label{fig:gap}
\end{figure}

By construction, the value $M$ in each iteration is always an integer between $L$ and $R$. Moreover, at the end of each iteration, there are only elements of height $0$ in the interval $[0,L]$ (if any), and all elements of larger height in $[R,2^{n}]$ (if any), since we set the left resp.~right bound accordingly in each iteration. 
Hence, after $n$ iterations and $R-L=1$ we have all elements of height $0$ in $[0,L]$, and elements of larger height in $[L+1,2^{n}]$. 
In each iteration, we insert an element of height $0$ with constant probability $1-p$, which will eventually lie in the interval $[0,L]$. Therefore, the expected number of elements in $[0,L]$ is $(1-p)n$.
%
The resulting skip list is now highly degenerated in the interval $[0,L]$. Specifically, it corresponds to a simple linked list of average length $(1-p)n$ in this part. Hence, the search for the element $L$ takes linear time on average, whereas a regular skip list would yield a logarithmic average search time. This is an exponential blow-up in running time, which the gap attacker enforces.

\subsection{A Robust Construction}

\begin{figure*}[thp]
    %	\Wider[4em]{
    %	\Wider[4em]{
            \centering
            \begin{pchstack}[boxed,center,space=0.5em]
                \begin{pcvstack}[space=0.45em]
                        \procedure[linenumbering, headlinecmd={\vspace{.1em}\hrule\vspace{.1em}}]{$\Rep_{\key}(\setS)$}{%
                            \mathsf{h} \gets \schemefont{NewNode}(m,\star)\\
                            \llst.\hdr \gets \mathsf{h}, \;
                            \llst.\lvl \gets 1\\
                            \pcfor (x,v) \in \setS \\
                            \t \llst \gets \Up_{\key}(\llst,\ins_{(x,v)})\\						
                            \pcreturn \llst
                        }
                        \procedure[linenumbering, headlinecmd={\vspace{.1em}\hrule\vspace{.1em}}]{$\schemefont{NewNode}(\ell,(x,v))$}{%
                            \pccomment{array position $0$ is reserved for}\\
                            \pccomment{a deleted bit, key, value triple $(d,x,v)$}\\
                            \pccomment{accessible via $n.\delacc$, $n.\keyacc$ and $n.\valueacc$}\\
                            \pccomment{array positions $1 \ldots \ell$ are forward pointers}\\
                            \pccomment{level is accessible via $n.\lvl$}\\
                            \node \gets \kwnew\; [0,..,\ell]\\
                            \node[0] \gets (\bot, x,v)\\
                            \pcfor i \gets \ell \ \mathbf{downto} \ 1 \pcdo\\
                            \t \node[i] \gets \nlll\\
                            \pcreturn \node
                        }
                        \procedure[linenumbering, headlinecmd={\vspace{.1em}\hrule\vspace{.1em}}]{$\schemefont{RandomLevel}_{\key}(\boxed{x})$}{%
                            \boxed{\ell \gets R(\key,x,m,p)}\\
                            \boxed{\pcreturn \ell}\\
                            \ell \gets 1, \;
                            r \getsr [0,1)\\
                            \pcwhile r < p \ \mathbf{and} \ \ell < m \pcdo\\
                            \t \ell \gets \ell + 1, \; r \getsr [0,1)\\
                            \pcreturn \ell
                        }
                       \procedure[linenumbering, headlinecmd={\vspace{.1em}\hrule\vspace{.1em}}]{$\Qry(\llst,\qry_{x})$}{%
                          c \gets \llst.\hdr\\
                            \pcfor i \gets \llst.\lvl \ \mathbf{downto} \ 1 \pcdo\\
                            \t \pcwhile c[i] \neq \nlll \ \mathbf{and} \ c[i][0].\keyacc < x \pcdo \\
                            \t \t c \gets c[i]\\
                            c \gets c[1]\\
                            \pcif c \neq \nlll \ \mathbf{and} \  c[0].\keyacc = x \ \mathbf{and} \ c[0].\delacc \neq \bot \pcthen\\
                            \t \pcreturn c[0].\valueacc \\
                            \pcelse \\
                            \t \pcreturn \star
                        }
                \end{pcvstack}	
                \begin{pcvstack}[space=0.45em]
                        \procedure[linenumbering, headlinecmd={\vspace{.1em}\hrule\vspace{.1em}}]{$\Up_{\key}(\llst,\ins_{(x,v)})$}{%
                            \mathsf{u} \gets \kwnew  [1,..,m] \pccomment{local array of pointers}\\
                            c \gets \llst.\hdr\\
                            \pcfor i \gets \llst.\lvl \ \mathbf{downto} \ 1 \pcdo\\
                            \t \pcwhile c[i] \neq \nlll \ \mathbf{and} \ c[i][0].\keyacc < x \pcdo \\
                            \t \t c \gets c[i]\\
                            \t u[i] \gets c\\
                            c \gets c[1] \\
                            \pcif c \neq \nlll \ \mathbf{and} \  c[0].\keyacc = x \pcthen\\
                            \t c[0].\valueacc \gets v,\; 
                            c[0].\delacc = \bot\\
                            \pcelse\\
                            \t \ell \gets \schemefont{RandomLevel}_{\key}(\boxed{x})\\
                            \t \pcif \ell > \llst.\lvl \pcthen \\
                            \t\t \pcfor i \gets \llst.\lvl + 1 \ \mathbf{upto} \ \ell \pcdo\\
                            \t\t\t \mathsf{u}[i] \gets \llst.\hdr\\
                            \t\t \llst.\lvl \gets \ell\\
                            \t \mathsf{n} \gets \schemefont{NewNode}(\ell,(x,v))\\
                            \t \pcfor i \gets 1 \ \mathbf{upto} \ \ell \pcdo\\
                            \t \t \mathsf{n}[i]\gets \mathsf{u}[i][i] ,\;
                            u[i][i]  \gets \mathsf{n}\\
                            \pccomment{find layer $\ell-1$ middle element using tortoise and hare} \\
                            \t \mathsf{middle} \gets u[\ell], \; \mathsf{fast} \gets u[\ell]\\
                            \t \pcwhile \mathsf{fast}  \neq \mathsf{n}[\ell] \ \mathbf{and} \ \mathsf{fast}[\ell-1] \neq \mathsf{n}[\ell] \pcdo \\
                            \t \t \mathsf{middle} \gets \mathsf{middle}[\ell-1], \;
                            \mathsf{fast} \gets \mathsf{fast}[\ell-1][\ell-1]\\
                            \pccomment{swapping logic} \\
                            \t \pcif \ell > \mathsf{middle}.\lvl \pcthen\\
                            \t \t \mathsf{middle}.\mathsf{append}(\mathsf{n}[\ell]), \;
                            \mathsf{n} \gets \mathsf{n}[0:\ell-1]\\
                            \t \t u[\ell][\ell] \gets \mathsf{middle} \\
                            \pcreturn \llst
                        }
                        \procedure[linenumbering, headlinecmd={\vspace{.1em}\hrule\vspace{.1em}}]{$\Up(\llst,\del_{x})$}{%
                            c \gets \llst.\hdr\\
                            \pcfor i \gets \llst.\lvl \ \mathbf{downto} \ 1 \pcdo\\
                            \t \pcwhile c[i] \neq \nlll \ \mathbf{and} \ c[i][0].\keyacc < x \pcdo \\
                            \t \t c \gets c[i]\\
                            c \gets c[1]\\
                            \pcif c \neq \nlll \ \mathbf{and} \  c[0].\keyacc = x \pcthen\\
                            \t c[0].\delacc = \top\\
                            \pcreturn \llst
                        }
                \end{pcvstack}	
            \end{pchstack}
    %	}
      \caption[A Robust Skip List.]{A robust, possibly ``deterministic'' (and keyed) skip list structure $\SL[\boxed{R},m,p]$ admitting insertions, deletions, and queries for any~$x \in \univ$ for some well-ordered universe~$\univ$. The parameters are an integer $m \geq 0$ representing the maximum level of the structure, a fraction~$p \in (0,1)$ used for determining an element's random level, and, if using the deterministic version of the structure, a keyed function $R: \keys \by \univ \by \mathbb{Z}^{+} \by (0,1) \to [m]$ that maps an element to a level in accordance with the distribution imposed by~$m$ and~$p$. A concrete scheme is given by a particular choice of parameters. Subroutines used by the deterministic version of the structure appear in the boxed environment. We define the~$\mathsf{append}$ operation, such that it appends an element to the end of the node array.} 
      \label{fig:rsl}
    \end{figure*}

In Figure~\ref{fig:rsl}, we give a pseudocode description of the robust skip list structure. Notably, elements can be marked are deleted by marking a ``deleted'' bit~$d$, that is stored in the node as~$\top$. It is important to point out that deleting and reinserting an element does not change the associated height, as only the $d$ bit is flipped to~$\bot$. Importantly, we do not allow for deleted elements to be replaced by subsequent insertions due to the need to preserve order-sensitive operations, and the fact that such a mechanism could be used to accelerate the gap attack we present above. 

\begin{figure}[thp]
%	\Wider[4em]{
    \centering
    \begin{pchstack}[boxed,center,space=0.5em]
        \procedure[linenumbering, headlinecmd={\vspace{.1em}\hrule\vspace{.1em}}]{Swapping mechanism for robust skip lists}{%
            \pccomment{find layer $\ell-1$ middle element using tortoise and hare} \\
            \mathsf{middle} \gets u[\ell], \; \mathsf{fast} \gets u[\ell]\\
            \pcwhile \mathsf{fast}  \neq \mathsf{x}[\ell] \ \mathbf{and} \ \mathsf{fast}[\ell-1] \neq \mathsf{x}[\ell] \pcdo \\
            \t \mathsf{middle} \gets \mathsf{middle}[\ell-1], \;
            \mathsf{fast} \gets \mathsf{fast}[\ell-1][\ell-1]\\
            \pccomment{swapping logic} \\
            \pcif \ell > \mathsf{middle}.\lvl \pcthen\\
            \t \mathsf{middle}.\mathsf{append}(\mathsf{x}[\ell]), \;
            \mathsf{x} \gets \mathsf{x}[0:\ell-1]\\
            \t u[\ell][\ell] \gets \mathsf{middle}
        }
    \end{pchstack}
%	}
  \caption[Skip List Swapping Mechanism.]{The swapping mechanism for robust skip lists, which is invoked after a node $\mathsf{x}$  has been inserted on layer $\ell$ and update vector $u$ has been constructed during this process. 
  } 
  \label{fig:swap}
\end{figure}

We use a simple swapping mechanism to make the skip list robust (depicted in~\Cref{fig:swap}). When the skip list inserts an element (denoted as $\mathsf{x}$), it first performs a standard insertion and then invokes the swapping procedure using the update vector $u$ constructed during insertion. After node $\mathsf{x}$ is inserted on layer $\ell$, the mechanism counts nodes on layer $\ell-1$ between $u[\ell]$ and $\mathsf{x}[\ell]$ ($\mathsf{x}$'s successor on level $\ell$). 

The middle element is then identified in a single pass using the tortoise and hare algorithm \cite{knuth1971art}. If the middle element is not $\mathsf{x}$ itself (verified by checking $\ell < \text{middle.lvl}$), a height swap occurs: the middle element's height increases to level $\ell$ while node $\mathsf{x}$'s height decreases to level $\ell-1$, effectively exchanging their heights.

This mechanism locally balances the skip list, preventing adversaries from creating large sequences of same-height elements that would result in search path blowup. The gap attack specifically becomes highly infeasible, as elements of a fixed height can no longer be shifted toward one side of the data structure. Instead, heights are immediately swapped at the interval's midpoint, halving long sequences of elements on level $\ell-1$. Note, this approach effectively handles corner cases where $\ell = \text{list.header}$ or $n[\ell] = \text{null}$. Moreover, the interval typically contains a constant, denoted~$a$, number of nodes with overwhelming probability, ensuring the mechanism operates in constant time with high probability.

We will formally show that our robust skip list is secure via a number of intermediary lemmas. The first of which proves a necessary condition for degenerating a skip list (including our robust version). We specifically analyze the skip list from the point of view of being able to have infinite height (we rectify this with reality before delivering our final result). One examines the set of elements that appear in the skip list strictly below level~$L(n) = \log_{1/p}(n)$, where $n$ is the number of (``deleted'' or actual) elements in the skip list (i.e., its implied capacity), and the set of elements that appear at or above level~$L(n)$.

We first relate the length of a search path (i.e., the number of nodes to be visited) to the maximal width $w$ on each level below $L(n)$, where the maximal width describes the maximal number of level $i$ elements between level $i+1$ elements in the skip list over all levels $i=0,1,\dots, L(n)-1$.  Here, we call an element a \emph{level $i$ element} if the node's height is at least $i$. In particular, any level $j$ element is also a level $i$ element for $i\le j$. We say that the element is a \emph{max-level $i$ element} if it is a level $i$ element but not a level $i+1$ element.

\begin{lemma}[Necessary Condition for Degenerating a Skip List]\label{lemma:ncdsl}
%mf: rewrote it
Consider a skip list for parameter $p\in(0,1)$ holding $n$ elements (possibly inserted by an adaptive adversary). If on all levels $i\in\{0,1,\dots, L(n)-1\}$ the number of level $i$ elements between any pair of level $i+1$ elements is at most $w$, then any search path is of length at most $2w\log_{1/p}n$, or the total number of elements on or above level $L(n)$ exceeds $w\log_{1/p}n$.
%Consider levels~$i \in \{1,2,\ldots,L(n)\}$. Then it is necessary for the adversary to craft a skip list representation where for some level~$i$ there exists~$> c$ elements between two level~$i + 1$ elements for some small constant~$c$ for there to exist a search path for an element that is greater than~$c \log_{1/p} (n)$; \emph{or} the total size of the skip list strictly above level~$L(n)$ is~$> c \log (n)$ for this same small constant~$c$. 
\end{lemma}

The lemma states that, for the adversary to create a bad skip-list representation, it may either hope that many elements are assigned a height beyond $L(n)$---which is very unlikely since the heights are determined faithfully by the data structure---or it must ensure that there is a ``degenerated'' sub-lists exceeding the width $w$ on some level. The latter matches our gap attack in Section~\ref{sec:gap-attack}, where we followed this strategy, and the lemma states that this is indeed the only valid attack strategy.

\begin{proof}
Assume that on all levels $i$, there exists at most~$w$ elements between any two elements on level $i+1$, and 
%that there exists a level above and 
that the total size of the skip list on or above level~$L(n)$ is at most $ w \log_{1/p} n$. Then the search path below level~$L(n)$ is at most $w \log_{1/p} n$ because whenever we descend to a level $i$ (and the index to be searched is thus between the indexes of both level $i+1$ elements), we make at most $w$ steps on the level $i$. This bounds the total number of steps on all levels $i$ below $L(n)$ by $w\cdot L(n)=w\cdot \log_{1/p} n$.  In addition, on level $L(n)$ or above, the total number of elements is bounded by $w \log_{1/p} n$, such that even searching all these elements cannot increase the overall number of inspected elements by more than $w \log_{1/p} n$. This yields an overall length of the search part of $2w \log_{1/p} n$.
\end{proof}

We formulate the following game to bound the number of elements on max-level~$i$ between two level~$i+1$ elements for the robust skip list. We assume that the adversary can insert as many elements as they like (up to its insertion limit~$q_U$) and that the adversary can insert into any gap arbitrarily many times. Further, observe that the adversary cannot influence the height of any particular element or alter the heights that were chosen by deletion due to our special deletion method. Then, the ability of the adversary to accrue elements that exist on level~$i$ between two level~$i+1$ elements distills down to a coin-flipping game. 

Given the number of individual trials~$n$ and probability~$p$, the game is as follows. For each individual trial, a coin (that is~$\heads$ with probability~$p$ and is~$\tails$ with probability~$1-p$) is flipped until a tail appears, at which point the particular trial is concluded. The outcome of a trial is the total number of~$\heads$ that occurred during a particular trial. For instance, the outcome~$\tails$ maps to~$0$, while the outcome~$\heads,\heads,\tails$ maps to~$2$. 

The game keeps a sequence of all the outcomes. Say the sequence at a point in time~$t-1$ is~$o_1,o_2,\ldots,o_{t-1}$. The adversary is allowed to run the next trial~$t$ anywhere within the sequence. That is, they could dictate the outcome of trial~$t$ (the result of which they do not control, as coin flips determine this) at the beginning of the sequence (before~$o_1$), at the end of the sequence (after~$o_{t-1}$), or anywhere in between two adjacent~$o_{i-1},o_{i},i \leq t-1$. The trial is then run, the outcome recorded in the sequence, and the sequence relabeled (depending on where the adversary decided to place the outcome of the most recently run trial). 

For each possible trial outcome, we have the following ``halving'' behavior concerning runs (consecutive subsequences) of outcome~$i$ for each~$i \in \{0,1,\ldots \log_{1/p}(n)\}$. If the adversary is trying to extend a particular run, they always insert it at the beginning or end of the run. By inspection of our robust skip list structure, this strategy is optimal, as it maximizes the probability of extending a particular run (by minimizing the probability of halving). Given this, when an  adversary tries to extend a run of outcome~$i$, three possible outcomes can occur:

\begin{enumerate}
    \item if the outcome of this fresh trial is~$i$, then the run extends by length~$1$;
    \item if the outcome of this trial is~$i+1$;  the length of the run is halved (or more precisely, the updated run length is the ceiling of dividing the previous run length by $2$);
    \item if the outcome of the fresh trial is any other outcome; the run length remains the same. 
\end{enumerate}

Observe that this is precisely equivalent to the procedure for an adaptive adversary inserting elements into our robust skip list with probability parameter~$p$ and insertion budget (skip list capacity)~$n = q_U$. We specifically consider the scenario where the adversary tries to accrue elements that exist on level~$0$ between two level~$1$ elements. This is because the probability of the accruing elements on this level is maximized. Looking ahead, we will cast this run width accruing game as a stochasic process that is supermartingale, generalize our result for level~$0$ to all levels~$i \in \{0,1,\ldots,\log_{1/p}(n) - 1\}$, and combine them to get a bound on the maximum search path cost over the entire robust skip list.

\begin{lemma}[Bounding Layer Sequential Elements for the Robust Skip List]\label{lemma:rsllb}
Denote~$W_i$ the random variable describing the maximum sequence of elements that exist on max-layer~$i$ between two level~$i+1$ elements for~$i \in \{0,1,\ldots,\log{1/p}(n)-1\}$ for a robust skip list. For $\epsilon>0$ and $a=\frac{2(1+p)}{p}$ let~$W$ be the event that there exists $W_i> a(1+\epsilon)$ for some $i \in \{0,1,\ldots,L(n)-1\}$, then
\[\Pr[W]  \leq e^{(\lambda^{*}a) - (\epsilon\lambda^{*}a)},\]
where ~$\lambda^{*}$ is the maximal solution~$\lambda > 0$ to
$$(1-p)e^{\lambda} + p(1-p)e^{-\lambda \left(\frac{1}{p} + \frac{a}{2} \right)} + p^2 \leq 1 .$$
\end{lemma}

\begin{proof}
\textit{Defining the Probabilistic Process.} We begin by considering the adversary trying to accrue a run of outcome~$0$. Given the adversary can play many independent trials, indexed by integer~$t$ (and in reality bounded by~$n=q_U$), we define a counter tracking the run length~$X_t$, initialized to~$0$, that is updated as follows:

    \[ X_{t+1} = \begin{cases} 
          X_t +1 & \text{with probability } 1-p \\
           \left\lceil \frac{X_t}{2} \right\rceil & \text{with probability } p(1-p) \\
          X_t & \text{with probability } 1 - ((1-p) + p(1-p) = p^2 \\
       \end{cases}
    \]

For~$X_t = x$,
\begin{align*}
    \mathbb{E}[X_{t+1} | X_t = x] &= (1-p)(x+1) + p(1-p)(\frac{x}{2} +1) + p^2x \\
    &= \left((1-p)+\frac{p(1-p)}{2}+p^2\right)x + (1-p)+p(1-p) \\
    & = \left(1 - \frac{p}{2} + \frac{p^2}{2} \right)x + 1-p^2,
\end{align*}
where we approximate~$\left\lceil \frac{x}{2} \right\rceil$ as~$\frac{x}{2} + 1$.

Setting~$A = \left((1-p)+\frac{p(1-p)}{2}+p^2\right)$ and~$D = 1-p^2$, we can solve for a fix point (i.e., the steady-state solution where the drift is zero) by solving~$a = Aa + D = a(1-A) = D$:

$$a = \frac{D}{1-A} = \frac{1-p^2}{\frac{p(1-p)}{2}} = \frac{2(1+p)}{p}.$$

Therefore, for any,~$p$ the fixed point is~$a = \frac{2(1+p)}{p}$.\\

\textit{Bounding the Process for Outcome~$0$.} 
We next cast this process as martingale to be able to apply a concentration bound. Specifically, for trying to accrue a run of outcome~$0$, we define the process

$$M_t = e^{(\lambda(X_t - a))},$$

where~$\lambda > 0$ is a parameter to be selected. Our goal is to show that when~$X_t$ exceeds a certain threshold (say~$x \geq a + B$ for some constant~$B > 0$), the process~$M_t$ is supermartingale. That is for all~$x \geq a + B, \mathbb{E}[M_{t+1} | X_t = x] \leq M_t$. 

Using the update rule for our process, we have for~$X_t = x$:
\begin{align*}
    \mathbb{E}[M_{t+1} | X_t = x] &= (1-p)e^{\lambda((x+1)-a) + p(1-p)e^{\lambda}((x/2+1)-a) + p^2e^{\lambda(x-a)}} \\
    &= e^{\lambda(x-a)} \left( (1-p)e^{\lambda} + p(1-p)e^{\lambda(1-x/2)}+p^2 \right).
\end{align*}

Observe that since the term~$e^{\lambda1-x/2}$ decreases in~$x$, the worst case for~$x \geq a+B$ is exactly at~$x=a+B$. Hence, it suffices to have
$$(1-p)e^{\lambda} + p(1-p)e^{\lambda(1-a+B/2)}+p^2 \leq 1$$
for our stochastic process to satisfy the supermartingale condition. 

Further, we have~$\frac{a+b}{2} = \frac{1+p}{p} + \frac{B}{2}$ and $1 - \frac{a+B}{2} = -\frac{1}{p} - \frac{B}{2}$, thus, our condition simplifies to
$$(1-p)e^{\lambda} + p(1-p)e^{-\lambda \left( \frac{1}{p} - \frac{B}{2} \right)} + p^2 \leq 1.$$

For any fixed~$p \in (0,1)$ and chosen~$B>0$ (in practice we chose~$B$ to be a small constant that is~$\approx a$), one can solve for that largest~$\lambda$ that satisfies this inequality; denote this~$\lambda^{*} = \lambda(p,B)$. 

Now, define a stopping time
$$t_0 = \min \{ t \geq 0 : X_t \geq a + B \}.$$

At this stopping time, we have 
$$M_{t_0} = e^{\lambda^{*}(X_{t_0} - a)} \leq e^{\lambda^{*}B},$$
as~$X_{t_{0}} \geq a+B$. We then work with the stopped process~$M_{t \land t_0}$ (or more precisely with the process from time $t_0$ onward) and apply Ville's inequality~\cite{ville1939etude}. This yields for all~$k \geq 0$
$$\Pr \left[ \underset{0 \leq t_0 \leq n}{\max} X_t \geq a + k \right] \leq \frac{\mathbb{E}[M_{t_0}]}{e^{\lambda^{*}k}} \leq e^{\lambda^{*}B}e^{-\lambda^{*}k}.$$

Define~$C = e^{\lambda^{*}B}$, we then obtain the concentration bound for outcome~$0$ for all~$k \geq 0$:
$$\Pr [X_n \geq a + k] \leq Ce^{-\lambda^{*}k}.$$

Recasting in the multiplicative form (as~$(1+\epsilon)a = a + \epsilon a$), for any~$\epsilon > 0$ we have
$$\Pr[X_t \geq (1 + \epsilon)a] \leq Ce^{-\lambda^{*} \epsilon a}.$$\\

\textit{Lifting Result to All Outcomes.} Next, we use a chaining argument to lift our result to the maximum over all outcomes~$ j \in \{ 0,1,\ldots,\log_{1/p}(n)-1 \} $.

Let~$X_{n}^{(j)}$ denote the run length process for outcome \\ $j \in \{0,1,\ldots,\log_{1/p}(n)-1\}$. Observe that the probability of outcome~$j$ is~$p^{j}(1-p)$ and the outcome~$j+1$ is~$p^{j+1}(1-p)$. Thus, the ratio is
$$\frac{p^{j}(1-p)}{p^{j+1}(1-p)}=\frac{1}{p},$$
which is independent of~$j$. In turn, the fixed point~$a = \frac{2(1+p)}{p}$ is identical (up to a constant additive error) for every outcome~$j$. Therefore, for each~$j$, we have
$$\Pr \left[ \underset{0 \leq t_0 \leq n}{\max} X_t^{(j)} \geq a + k \right] \leq Ce^{-\lambda^{*}k}.$$

A naive union bound would suggest
$$\Pr \left[ \max_{j=0}^{\log_{1/p}(n)-1} \underset{0 \leq t_0 \leq n}{\max} X_t^{(j)} \geq a + k \right] \leq \log_{1/p}(n)Ce^{-\lambda^{*}k}.$$ 

However, using a standard chaining and peeling argument~\cite{bound_book} we can show that in fact, there exist constants~$C',\lambda' > 0$ (depending only on~$p$) such that 
$$\Pr \left[ \max_{j=0}^{\log_{1/p}(n)-1} \underset{0 \leq t_0 \leq n}{\max} X_t^{(j)} \geq a + k \right] \leq C'e^{-\lambda^{'}k},$$

or equivalently, for any~$\epsilon > 0,$

$$\Pr \left[ \max_{j=0}^{\log_{1/p}(n)-1} \underset{0 \leq t_0 \leq n}{\max} X_t^{(j)} \geq (1 + \epsilon)a  \right] \leq C'e^{-\lambda^{'} \epsilon a}.$$

In practice, we simply take~$C' \approx C$ and~$\lambda' \approx \lambda^{*} = \lambda(p,B)$ (the same constants as above for outcome~$0$), as there is at most a negligible difference between the bound for different outcomes in~$\{0,1,\ldots,\log_{1/p}(n)-1\}$ and the bound is maximized at outcome~$0$. Then, by choosing~$B=a$, we obtain our result in the lemma. 
\end{proof}

\begin{lemma}[Overall Robust Skip List Search Path Cost]\label{lemma:rslspc}
 For $\epsilon>0$ and $a=\frac{2(1+p)}{p}$, let~$S$ be the total search path cost of the robust skip list, , then \[\Pr[S \geq a(1+\epsilon) \log_{1/p}(n)]  \leq e^{(\lambda^{*}a) - (\epsilon\lambda^{*}a)},\]
 where~$\lambda^{*}$ is the maximal solution~$\lambda > 0$ to
 $$(1-p)e^{\lambda} + p(1-p)e^{-\lambda \left(\frac{1}{p} + \frac{a}{2} \right)} + p^2 \leq 1 .$$
\end{lemma}

\begin{proof}
The lemma directly follows from~\cref{lemma:rsllb}, and the fact that
\begin{align*}
    S &= \sum_{j=0}^{\log_{1/p}(n)} X^{(j)} \\
      &\leq \sum_{j=0}^{\log_{1/p}(n)} max_{j=0}^{\log_{1/p}(n)-1} \underset{0 \leq t_0 \leq n}{\max} X_t^{(j)}.
\end{align*}
\end{proof}

Next, we bound the number of elements in a skip list above level~$L(n) = \log_{1/p}(n)$, addressing the second point in~\cref{lemma:ncdsl}. 

\begin{lemma}[Bound on the Size of the list Above Level~$L(n)$]\label{lemma:lvlbound}
    Given a (robust) skip list with probability parameter~$p$, let $H$ be the number of elements that appear at heights~$\geq L(n) = \log_{1/p}(n)$. That is, $H$ counts the total occurrences of elements at or above height~$L(n)$ resp.~the total size of the skip list at or above height~$L(n)$. Then,
    \[ \mathbb{E}[H] = \frac{1}{1-p}, \]
    and
    \[ \Pr\left[H \geq w \cdot \log_{1/p}(n)\right] \leq e^{-\frac{w\log_{1/p}(n)-1}{3}}. \]
\end{lemma}

\begin{proof}
    Let $H_i\sim \operatorname{Bin}(n,p^i)$ denote the random variable describing the number of elements on level $i$ among the $n$ elements, such that $\mathbb{E}[H_i]=np^i$.
    Let $H = \sum_{i \ge L(n)} H_i$ be the total number of times an element appears at some level in all levels above level $L(n) = \log_{1/p}(n)$. Then,
    \[ \mathbb{E}[H] = \sum_{i \ge L(n)} np^{i} = np^{L(n)} \sum_{j \geq 0} p^{j} = np^{L(n) } \frac{1}{1-p}. \]
    Now, observe that $p^{L(n)} = p^{\log_{1/p}(n)} = n^{\log_{1/p} p} = n^{-1}$, in turn $np^{L(n) } = n \cdot p^{L(n)} = n \cdot \frac{1}{n}  = 1$. Therefore, the expected number of elements that appear at any level on or above level~$L(n)$ is
    \[ \mathbb{E}[H] = \frac{1}{1-p}. \]
    We then obtain a tail bound via the standard Chernoff bound, solving for a value $\delta$\footnote{Here~$\delta$ refers to the usual difference from the mean in the Chernoff bound, not the parameter of the AAPC security notion.} such that~$(1+\delta)\left(\frac{1}{1-p}\right) = w \log_{1/p}(n)$. This completes the proof. The value $\delta=(1-p)w\log_{1/p}(n)-1$ works, and we get the upper bound of $\exp\left(-\frac{\delta}{3(1-p)}\right)$.
\end{proof}

In our robust skip list, we define a maximum level~$m$. This, in turn, defines the capacity of the list~$n$ by solving~$m = \log_{1/p}(n) = L(n)$. So, in reality, this result actually reflects the maximum number of elements on level~$L(n)$. To win in our game, the adversary must either craft a structure such that the total search path cost below level~$L(n)$ exceeds~$w \log_{1/p}(n)$ or the above ``bad'' event happens where there exists more than~$w \log_{1/p}(n)$ elements on level~$L(n)$. Combining these results gives us the following theorem.

\begin{theorem}[Robust Skip List AAPC Security Result]\label{thm:rslsr}
    Let~$\Pi$ be the robust skip list from Figure~\ref{fig:rsl} with parameters~$p \in [0,1]$ and $m = \log_{1/p}(q_U)$. For integers~$q_U,q_Q,t \geq 0$, it holds that~$\Pi$ is~$(\phi,\beta,\epsilon,\delta,t)$-conserved with $\phi$ being the Maximum Search Path Cost function (Figure~\ref{fig:sl-cost}), $\beta = c \log_{1/p}(n)$, $\epsilon > 0 $, and 
    $$\delta = e^{(\lambda^{*}a) - (\epsilon\lambda^{*}a)} + e^{-\frac{a\log_{1/p}(n)-1}{3}},$$
    where~$a=\frac{2(1+p)}{p}$, $c = a(\epsilon + 1)$, and~$\lambda^{*}$ is the maximal solution~$\lambda > 0$ to
 $$(1-p)e^{\lambda} + p(1-p)e^{-\lambda \left(\frac{1}{p} + \frac{a}{2} \right)} + p^2 \leq 1 .$$
\end{theorem}

\begin{proof}
    The theorem directly follows from the observation deletions do not help the adversary as with our robust hash table and \Cref{lemma:ncdsl}, \Cref{lemma:rslspc}, and~\Cref{lemma:lvlbound}.
    \qed
\end{proof}

To give a concrete illustration of this bound, suppose we had~$n = 2^{32}$, $p=\frac{1}{2}$. Then our fixed point~$a = 6$, and solving for~$\lambda^{*}$ numerically yields~$\lambda^{*} \approx 0.34$. Then, choosing~$\epsilon = 8$ (hence, $c=54$), the probability that the maximum search cost path exceeds~$54 \log_{1/p}(n)$ is less than or equal to~$\delta =  6.28 \times 10^{-7}$. While a constant~$c = 54$ may appear large, consider that that,~$\lambda^{*}$ solely depends on~$p$. In turn, for any fixed~$\epsilon$ this bound is constant as~$n\to\infty$, showing that our adaptive search path is indeed~$O(\log n)$. We further, remark that this constant is likely ``artificially'' large, in the sense that the stochastic process we bound is complex, leaving us only to be able to use blunt Markov-like concentration bounds.

\subsection{Robust Skip Lists in Real World Deployments}

Skip lists, like hash tables, have an explicit capacity for a set number of elements and require resizing when exceeded. While skip lists do not require upfront memory allocation, they require setting a maximum node height of $m=\log_{\frac{1}{p}} n$ for expected $n$ insertions. Exceeding $n$ insertions necessitates resizing as the probabilistic guarantees otherwise deteriorate~\cite{pugh}. Our security analysis avoids considering attacks that trigger re-initialization by enforcing~$m = \log_{\frac{1}{p}} q_U$/
We now evaluate how our modifications affect resizing frequency in practice.

Standard skip lists with $I$ successful insertions and $D$ successful deletions require resizing when $\log_{\frac{1}{p}} (I-D) > m$. Previously, structures could operate indefinitely without resizing if insertion and deletion rates remained balanced. Our modified structure, which merely marks elements as "deleted" without allowing replacement, requires resizing when $\log_{\frac{1}{p}} (I) > m$ regardless of remaining elements. This allows adversaries to trigger early resizing by inserting approximately $\left(\frac{1}{p}\right)^m$ elements, deleting most, then forcing a resize with few additional insertions.

Unlike hash tables, we cannot replace deleted nodes without creating a security vulnerability where adversaries could manipulate the skip list by strategically deleting elements and inserting new ones, effectively repositioning unfavorable heights in other parts of the skip list -- essentially enhancing our gap attack. While our approach requires more frequent resizing, this represents an essential trade-off ensuring provable robustness against adaptive adversarial attacks while preserving expected performance characteristics.

%-------------------------------------------------------------------------------

%-------------------------------------------------------------------------------
\section{Robust Treaps}\label{sec:treaps}
\subsection{(In)Security of the Standard Treap}
Unlike other probabilistic data structures in this study, treaps (without deletions) demonstrate intrinsic security against search path cost blow-up. However, adaptive adversaries can still mount attacks that force certain elements to be near the root with high probability. 

Consider a lottery system designed to select a number of winners with uniform probability from a participant pool. Imagine, the implementation uses a treap data structure with an in-order traversal limited to a constant path length, thereby theoretically ensuring equal selection probability for all participants.

However, this implementation contains a critical security vulnerability against adaptive adversaries. While attackers cannot directly manipulate the random priority values assigned to entries, they can execute a more sophisticated attack by strategically inserting elements with carefully chosen keys positioned adjacent to a target element. By continuing this insertion pattern until placing an element with exceptionally low priority, they force the treap to perform rotation operations that elevate their target element toward the root. Since elements closer to the root are more likely to be selected during the limited-depth traversal, this compromises the lottery's fairness.

Concretely, let $x_1, x_2, \ldots, x_{j-1}$ be the keys inserted in sorted order with associated priorities 
$$r^{x_1}, r^{x_2}, \ldots, r^{x_{j-1}},$$
drawn independently from the uniform distribution on $[0,1]$. An adaptive adversary selects an arbitrary target element $x_i$. The adversary then repeatedly inserts new elements into the gaps between $x_i$ and $x_{i+1}$ and between $x_{i-1}$ and $x_i$ until obtaining an exceptionally low priority value. This is expected to occur after a linear number of insertions.

After inserting, let $S^n_{x_i}$ denote the search path to $x_i$. Since
\begin{align*}
S^n_{x_i} &= |\{\text{records in sequence } r^{x_i},r^{x_{i-1}},\ldots,r^{x_1}\}| + \\&|\{\text{records in sequence } r^{x_i},r^{x_{i+1}},\ldots,r^{x_n}\}| - 1,
\end{align*}
and the number of records in these intervals is constant (as the neighboring elements to $x_i$ have exceptionally low priorities for which there exists only a constant number of nodes with lower priorities), $x_i$ now resides near the top of the treap with high probability.

Importantly, for our purposes, treaps maintain their expected $O(\log n)$ operational complexity against adaptive adversaries only when our modified deletion procedure is applied. Without it, an adversary could simply re-insert (and delete) an element until obtaining a favorable priority, making degeneration attacks trivial.

This resistance to performance degradation attacks under lazy deletion represents a significant finding, as all other PSDS examined proved vulnerable. The treap's rebalancing mechanism, based on previously sampled priorities, provides a natural defense against malicious attempts to create operation sequences that would otherwise lead to worst-case runtime scenarios.

\subsection{A Robust Construction}

\begin{figure*}[thp]
    %	\Wider[4em]{
            \centering
            \begin{pchstack}[boxed,center,space=0.5em]
                \begin{pcvstack}[space=0.45em]
                        \procedure[linenumbering, headlinecmd={\vspace{.1em}\hrule\vspace{.1em}},codesize=\footnotesize]{$\Rep_{K}(\setS)$}{%
                            \ttree.\rt \gets \nlll \\
                            \pcfor (x,v) \in \setS \pcdo \\
                            \t \ttree \gets \Up_{K}(\ttree,\ins_{(x,v)})\\						
                            \pcreturn \ttree
                        }
                        \procedure[linenumbering, headlinecmd={\vspace{.1em}\hrule\vspace{.1em}},codesize=\footnotesize]{$\schemefont{RandomPriority}_{K}(\boxed{x})$}{%
                            \boxed{p \gets R(K,x)}\\
                            \boxed{\pcreturn p}\\
                            p \getsr (0,1)\\
                          \pcreturn p
                        }
                        \procedure[linenumbering, headlinecmd={\vspace{.1em}\hrule\vspace{.1em}},codesize=\footnotesize]{$\schemefont{NewNode}((x,v),p)$}{%
                            \pccomment{array position $0$ is reserved for} \\
                            \pccomment{a deleted bit, key, value triple $(d,x,v)$}\\
                            \pccomment{accessible via $n.\delacc$,, $n.\keyacc$ and $n.\valueacc$}\\
                            \pccomment{array positions $2, 3$ are child pointers and $1$ is priority}\\
                            \node \gets [(\bot, x,v),p,\nlll,\nlll]\\
                          \pcreturn \node
                        }
                       \procedure[linenumbering, headlinecmd={\vspace{.1em}\hrule\vspace{.1em}},codesize=\footnotesize]{$\Qry(\ttree,\qry_{x})$}{%
                          \ttree.\rt \gets \Qry^{\text{rec}}(\ttree.\rt,\qry_{x}) \\
                          \pcreturn \ttree
                        }
                       \procedure[linenumbering, headlinecmd={\vspace{.1em}\hrule\vspace{.1em}},codesize=\footnotesize]{$\Qry^{\text{rec}}(c,\qry_{x})$}{%
                          \pcif c = \nlll \pcthen \\
                            \t \pcreturn \star \\
                          \pcif c[0].\keyacc = x \pcthen \\
                            \t \pcreturn c[0].\keyacc \\
                            b \gets (x > c[0].\keyacc)\\
                            \pcreturn \Qry^{\text{rec}}(c[2+b],\qry_{x})
                        }
                        \procedure[linenumbering, headlinecmd={\vspace{.1em}\hrule\vspace{.1em}},codesize=\footnotesize]{$\schemefont{Rotate}(c,b)$}{%
                             \tmp \gets c[2+b][3-b] \\
                             c[2+b][3-b] \gets c \\
                             c[2+b] \gets \tmp \\
                            \pcreturn \tmp
                        }
                \end{pcvstack}	
                \begin{pcvstack}[space=0.45em]
                        \procedure[linenumbering, headlinecmd={\vspace{.1em}\hrule\vspace{.1em}},codesize=\footnotesize]{$\Up_{K}(\ttree,\ins_{(x,v)})$}{%
                            \ttree.\rt \gets \Up^{\text{rec}}_{K}(\ttree.\rt,\ins_{(x,v)}) \\
                            \pcreturn \ttree
                        }
                        \procedure[linenumbering, headlinecmd={\vspace{.1em}\hrule\vspace{.1em}},codesize=\footnotesize]{$\Up^{\text{rec}}_{K}(c,\ins_{(x,v)})$}{%
                            \pcif c = \nlll \pcthen \\
                             \t p \gets \schemefont{RandomPriority}_{K}(\boxed{x})\\
                            \t \pcreturn \schemefont{NewNode}((x,v),p) \\
                            \pcif c[0].\keyacc = x \pcthen \\
                            \t c[0].\valueacc \gets v, \; 
                            c[0].\keyacc \gets \bot \\
                            \t \pcreturn c \\
                            b \gets (x > c[0].\keyacc) \\
                            c[2+b] \gets \Up^{\text{rec}}_{K}(c[2+b],\ins_{(x,v)}) \\ 
                            \pccomment{maintain MIN Heap property}\\
                            \pcif c[1] > c[2+b][1] \pcthen \\
                            \t c \gets \schemefont{Rotate}(c,b) \\
                            \pcreturn c
                        }
                        \procedure[linenumbering, headlinecmd={\vspace{.1em}\hrule\vspace{.1em}},codesize=\footnotesize]{$\Up_{K}(\ttree,\del_{x})$}{%
                             \Up^{\text{rec}}_{K}(\ttree.\rt,\del_{x}) \\
                            \pcreturn \ttree
                        }
                        \procedure[linenumbering, headlinecmd={\vspace{.1em}\hrule\vspace{.1em}},codesize=\footnotesize]{$\Up^{\text{rec}}(c,\del_{x})$}{%
                            \pcif c = \nlll \pcthen \\
                            \t \pcreturn \\
                            \pcif c[0].\keyacc = x \pcthen \\
                            \pccomment{Remove node} \\
                            \t c[0].\delacc \gets  \top \\
                            \pcelse \\
                            \t b \gets (x > c[0].\keyacc) \\
                            \t \Up^{\text{rec}}_{K}(c[2+b],\del_{x}) \\ 
                            \pcreturn
                        }
                \end{pcvstack}	
            \end{pchstack}
    %	}
      \caption[A Robust Treap.]{A robust, possibly ``deterministic'' (and keyed) robust MIN treap structure $\TR[\boxed{R}]$ admitting insertions, deletions, and queries for any~$x \in \univ$ for some well-ordered universe~$\univ$. The parameter is a keyed function $R: \keys \by \univ \to (0,1))$ that assigns an element a random priority. Subroutines used by the deterministic version of the structure appear in the boxed environment. Let $\schemefont{MinPrioChild}(c)$ denote the function that returns the child index (0 or 1) of node $c$ with the minimum priority, or null if $c$ has no children.} 
      \label{fig:rtreap}
    \end{figure*}

We give a pseudocode description of the robust treap using our modified deletion procedure in~\Cref{fig:rtreap}. We will formally show the security (with regard to the maximal search path) of our modified-deletion treap. However, we first formalize a view of the treap's representation via a stochastic process. We start by analyzing the representation formed by a non-adaptive adversary and the subsequent maximum search path cost. 

Consider a treap containing $n$ elements inserted by a non-adaptive adversary, i.e., selected uniformly at random from the universe of all possible elements. Consider all inserted elements in the sorted order of their key value $x_1 \leq x_2 \leq \ldots \leq x_n$. Each key~$x_i$ is assigned a random priority~$r^{x_{i}}$ drawn independently from the uniform distribution on~$[0,1]$. 

Let $S^n_{x_i}$ denote the random variable representing the search path length for a fixed element $x_i$. The search path to element $x_i$ consists of all ancestors of $x_i$ in the treap structure. From Aragon and Seidel~\cite{aragon1989randomized}, $x_j$ is an ancestor of $x_i$ if and only if $x_j$ has the lowest priority among all elements between $x_i$ and $x_j$ (inclusive). Specifically:

\begin{itemize}
    \item If $j > i$, then $x_j$ is an ancestor of $x_i$ if and only if $r^{x_j} = \min\{r^{x_i}, r^{x_{i+1}}, \ldots, r^{x_j}\}$
    \item If $j < i$, then $x_j$ is an ancestor of $x_i$ if and only if $r^{x_j} = \min\{r^{x_j}, r^{x_{j+1}}, \ldots, r^{x_i}\}$
\end{itemize}

This means that an element is an ancestor of $x_i$ precisely when its priority is a minimum value -- what we will refer to as a "record" -- in one of two sequences extending from $x_i$. Hence, we can interpret $S^n_{x_i}$ as:

\begin{align*}
S^n_{x_i} &= |\{\text{records in sequence } r^{x_i},r^{x_{i-1}},\ldots,r^{x_1}\}| + \\&|\{\text{records in sequence } r^{x_i},r^{x_{i+1}},\ldots,r^{x_n}\}| - 1,
\end{align*}

where the subtraction of~$1$ accounts for $x_i$ being counted in both sequences. 

A classical fact about random permutations is the behavior of records. For a sequence of~$k$ i.i.d. uniformly distributed random variables, the probability that the~$j\text{-th}$ element is a record (i.e., it is less all~$j-1$ preceding values is exactly~$\frac{1}{j}$). More precisely, define the following indicator variables for a given sequence:

$$ I_j = 
\begin{cases}
    1, & \text{if the } j\text{-th} \text{ element is a record,}\\
    0, & \text{otherwise.}
\end{cases}
$$

Then we have~$\mathbb{E}[I_j] = \frac{1}{j}$. For a sequence of length~$k$, the total number of records is~$R_k = \sum_{j=1}^{k} I_j$ and its expectation is~$\mathbb{E}[R_k] = \sum_{j=1}^{k} \frac{1}{j} = H_k$, where~$H_k$ is the $k\text{-th}$ harmonic number. In our context, when considering the ``leftward'' sequence of priority values~$L_i$ (of length~$i$) and the ``rightward'' sequence if priority~$R_i$ (of length~$n-i+1$) with respect to key at index ~$i$, we have~$\mathbb{E}[L_i] = H_i$ and~$\mathbb{E}[R_i] = H_{n-i+1}$. 

Thus, for a fixed~$x_i$,
$$\mathbb{E}[S^{n}_{x_i}] = \mathbb{E}[L_i + R_i - 1] = H_i + H_{n-i+1} - 1.$$

Nothing, that for any~$i$ it must be that~$ H_i \leq H_n$ and $H_{n-i+1} \leq H_n$, and the well known fact~$H_n \leq \ln(n) + 1$, we have
\begin{align*}
    \mathbb{E}[S^{n}_{x_i}] &\leq 2H_n - 1 \\
    &\leq 2\ln(n) + 1.
\end{align*}

We next argue that even when an adaptive adversary determines the insertions, each inserted element's probability of forming a record remains exactly~$\frac{1}{j}$ when it is the~$j\text{-th}$ element inserted -- exactly the same as the non-adaptive case. Even though an adaptive adversary can observe all previous outcomes and choose the next element adaptively (that is, select the key value so it falls into any ``gap'' of existing key values, like in the case of the skip list in~\Cref{sec:skiplist}), the new priority is still drawn uniformly and independently from~$[0,1]$. The joint distribution of the prior priorities is unchanged. We formalize this idea in the following lemma. 

\begin{lemma}[Invariant Record Probability under Adaptive Insertion]\label{lemma:tre}
Let $x_1, x_2, \ldots, x_{j-1}$ be the keys inserted in sorted order with associated priorities 
$$r^{x_1}, r^{x_2}, \ldots, r^{x_{j-1}},$$
drawn independently from the uniform distribution on $[0,1]$. An adaptive adversary (chooses a gap (i.e., a position between any two or before/after these keys) into which to insert a new key $x_j$. The new key receives an independent priority $r^{x_j} \sim \mathsf{U}[0,1]$. After relabeling the keys according to their inherent order, let the sorted sequence of priorities (of all $j$ keys) be
$$r_{(1)} \le r_{(2)} \le \cdots \le r_{(j)}.$$

Then, even conditioned on the past $\sigma$-algebra $\mathcal{F}_{j-1}$ (which contains the ordered priority values and all adversarial decisions regarding the first $j-1$ insertions), we have
$$\Pr\Bigl(r^{x_j} = r_{(1)} \,\Big|\, \mathcal{F}_{j-1}\Bigr) = \frac{1}{j}.$$

\end{lemma}

\begin{proof}
Condition on the $\sigma$-algebra $\mathcal{F}_{j-1}$; that is, assume the priorities
$$r^{x_1}, r^{x_2}, \ldots, r^{x_{j-1}}$$
are fixed and rearranged in increasing order:
$$r_{(1)} \le r_{(2)} \le \cdots \le r_{(j-1)}.$$

An adaptive adversary may insert the new key~$x_j$ in any gap between any two keys in the current sequence (or before the smallest or after the largest). Still, such a decision affects only the position of the key in the \emph{key order} and does not alter the statistical properties of the newly drawn priority.

The new priority $r^{x_j}$ is drawn independently from $\mathsf{U}[0,1]$. Thus, when the new key is inserted, the complete set of $j$ priorities is
$$\{r^{x_j}, r_{(1)}, r_{(2)}, \ldots, r_{(j-1)}\}.$$

Since the first $j-1$ values are already fixed and $r^{x_j}$ is independent and uniformly distributed over $[0,1]$, the resulting set of $j$ priorities is exactly equivalent to a set of $j$ independent uniform samples upon relabeling.

In any sequence of $j$ i.i.d. $\mathsf{U}[0,1]$ random variables, symmetry implies that the probability that any particular one (here, the newly inserted element) is the minimum is exactly $1/j$. Formally, we have
$$\Pr\Bigl(r^{x_j} = \min\{r^{x_j},r_{(1)}, r_{(2)}, \ldots, r_{(j-1)}\}\,\Big|\, \mathcal{F}_{j-1} \Bigr) = \frac{1}{j}.$$

Thus, regardless of where the adversary chooses to insert $x_j$, the probability that $x_j$ is a record (i.e., its priority is the smallest among the first $j$ keys) remains equal to $1/j$, as in the non-adaptive setting.
\end{proof}

\begin{theorem}[Treap AAPC Result]\label{thm:tsb}
Let~$\Pi$ be the robust treap from Figure~\ref{fig:rtreap}. For integers~$q_U,q_Q,t \geq 0$, it holds that~$\Pi$ is~$(\phi,\beta,\epsilon,\delta,t)$-conserved with $\phi$ being the Maximum Search Path Cost function (Figure~\ref{fig:t-cost}), $\beta = 2\ln n + 1$, any~$\epsilon > 0$ and 
    
   $$\delta = n e^{-\frac{\epsilon^2 H_{n}}{2(1 + \epsilon)}},$$

where~$n=q_U$ and~$H_n$ is the~$n\text{-th}$ harmonic number.
\end{theorem}

\begin{proof}
Observe that deletions do not help the adversary, as by construction, they at most relabel an existing entry and cannot possibly extend the longest path. Therefore, we consider a treap of~$n=q_U$ keys (i.e., a treap with the maximal number of insertions made) built by an adaptive adversary.  

\textit{Casting the Insertion Process as a Doob Martingale.}\\
For a fresh key inserted at step~$j$, take the indicator variable~$I_j$ as defined above. Then, conditioned on the past $\sigma$-algebra $\mathcal{F}_{j-1}$ (which contains the ordered priority values and all adversarial decisions regarding the first $j-1$ insertions), and letting
$$m_{j-1} := \min\{r^{x_1}, r^{x_2}, \ldots, r^{x_{j-1}}\} \quad (\text{with } m_0 = 1),$$
it is easy to see that
$$\Pr(I_j = 1 | \mathcal{F}_{j-1}) = \Pr(r^{x_{j}} < m_{j-1} | \mathcal{F}_{j-1}) = m_{j-1}.$$

From~\cref{lemma:tre}, we have that even under adaptive insertions, the unconditional expectation remains
$$\mathbb{E}[m_{j-1}] = \frac{1}{j}.$$

Then, if letting~$X_n$ denote the total number of records over all~$n$ insertions, the unconditional expected number of records is
$$\mathbb{E}[X_n] = \sum_{j=1}^{n} \mathbb{E}[I_j] =  \sum_{j=1}^{n} \frac{1}{j} = H_n,$$
where~$H_n$ is the~$n\text{-th}$ harmonic number. 

From our above analysis, we have that the search path length~$S^{n}_x$ for key~$x$ is bounded in terms of the number of records~$X_n$ by
$$S_{n}^x \leq 2X_n -1.$$ 

Thus, if we can show that~$X_n$ is concentrated around~$H_n,$ we also have a bound on the search cost for a particular element. To do this, define the Doob martingale
$$M_j = \sum_{i=1}^{j} (I_i - \mathbb{E}[I_i | \mathcal{F}_{i-1}]), \quad j=0,1,2,\ldots,n,$$
with~$M_0=0.$ By construction,~$\{M_j\}$ is a martingale relative to the filtration~$\{F_j\}$. 

Next, observe that the martingale difference satisfy
$$D_j = M_j - M_{j-1} = I_j - \mathbb{E}[I_i | \mathcal{F}_{i-1}].$$
Since $I_j \in \{0,1\}$ and $\mathbb{E}[I_i | \mathcal{F}_{i-1} \in [0,1]$, we have~$|D_j| \leq 1$. 

Since $I_j\in\{0,1\}$ is a Bernoulli random variable with parameter~$m_{j-1}$, its conditional expectation is
$$
\mathbb{E}[I_j\mid \mathcal{F}_{j-1}] = m_{j-1},
$$
and the conditional variance is computed as:
\begin{align*}
\mathrm{Var}(I_j\mid \mathcal{F}_{j-1}) &= \mathbb{E}\Bigl[(I_j - m_{j-1})^2 \,\big|\, \mathcal{F}_{j-1}\Bigr] \\
&= m_{j-1}(1-m_{j-1}).
\end{align*}

Now, note that subtracting the constant $\mathbb{E}[I_j\mid \mathcal{F}_{j-1}]$ does not change the variance. That is,
\begin{align*}
    \mathrm{Var}(D_j\mid \mathcal{F}_{j-1}) &= \mathrm{Var}(I_j - \mathbb{E}[I_j\mid \mathcal{F}_{j-1}]\mid \mathcal{F}_{j-1})\\
&= \mathrm{Var}(I_j\mid \mathcal{F}_{j-1})\\
&= m_{j-1}(1-m_{j-1}).
\end{align*}

Thus, computing the predictable quadratic variation, we have
$$V_n = \sum_{j=1}^{n} \mathrm{Var}(D_j\mid \mathcal{F}_{j-1}) = \sum_{j=1}^{n} m_{j-1}(1-m_{j-1}).$$

Since $m_{j-1}(1-m_{j-1}) \leq m_{j-1}$ (because $1-m_{j-1}\le 1$ for all $m_{j-1}\in [0,1]$), we obtain
$$V_n \le \sum_{j=1}^{n} m_{j-1}.$$

Further, as~$E[m_{j-1}] = \frac{1}{j}$, we have
$$\mathbb{E}[V_n] \leq \sum_{j=1}^{n} \frac{1}{j} = H_n.$$

\textit{Applying A Concentration Bound.}\\

Freedman's inequality~\cite{freedman1975tail} states that if~$\{M_j\}$ is a martingale with a difference bounded by~$1$ with predictable quadratic variation~$V_n$, then for any~$a,b>0$,
$$\Pr(M_{n} \geq a \text{ and } V_n \leq b) \leq e^{-\frac{a^{2}}{2(a+b)}}.$$

We set~$b = H_n$ (as typically~$V_n$ will not exceed~$H_n$ by much) and chose~$a = \epsilon H_n$, where~$\epsilon > 0$ is our parameter from our security statement. 

Then, Freedman's inequality gives us
$$\Pr(M_n \geq \epsilon H_n) \leq  e^{-\frac{\epsilon^2H_{n}^{2}}{2(\epsilon H_n+H_n)}} = e^{-\frac{\epsilon^2 H_{n}}{2(1 + \epsilon)}}.$$

Since
$$X_n = \sum_{j=1}^{n} I_j = M_n + \sum_{j=1}^{n} \mathbb{E}[I_j | \mathcal{F}_{j-1}],$$
and~$\sum_{j=1}^{n} \mathbb{E}[I_j | \mathcal{F}_{j-1}]$ has expectation~$H_n$, the above inequality shows that with probability at least~$1 - e^{-\frac{\epsilon^2 H_{n}}{2(1 + \epsilon)}}$ we have~$X_n \leq (1+ \epsilon)H_n$. Further, recalling that~$S^{n}_{x} \leq 2X_n -1$, this implies with the same probability, $S^{n}_{x} \leq 2(1+\epsilon)H_n-1$.

\textit{Bounding the Search Cost Path Over All Elements.} \\

Let~$E_x$ be the event that the search path cost for a fixed element exceeds the threshold~$T = 2(1+\epsilon) \ln(n) + 1$. 

From the above, we have that
$$ \Pr(E_x) \leq e^{-\frac{\epsilon^2 H_{n}}{2(1 + \epsilon)}},$$

as~$H_n \leq ln(n) + 1$.

Then, applying a standard union bound over all the~$n$ elements in the treap, the event that there exists some element with a search path cost exceeding~$T$ is bounded by 
$$\Pr \left(\bigcup_{x \in \{x_1,\ldots,x_n \}} E_x  \right) \leq n e^{-\frac{\epsilon^2 H_{n}}{2(1 + \epsilon)}}.$$
\end{proof}

To give a concrete illustration of this bound, suppose we had~$n = 2^{32}$ and select~$\epsilon = 5$. Our expected search path cost is~$2 \ln(2^{32}) + 1 \approx 45.36$, and leveraging our results from~\Cref{thm:tsb} the probability the maximum search cost path exceeds this by five times is~$\leq \delta = 2^{32} \cdot e^{-\frac{25 H_{2^{32}}}{12}} \approx 6.65 \times 10^{-12}$.

\subsection{Robust Treaps in Real World Deployments}

Unlike hash tables and skip lists, treaps operate without an implicit maximum capacity and typically don't require resizing operations. However, our modified structure -- which only marks elements as ``deleted'' without allowing replacement -- may still necessitate periodic treap re-initialization to reclaim memory occupied by deleted elements. This limitation exists because allowing the replacement of deleted nodes would create a security vulnerability, enabling attackers to strategically shift unfavorable priorities to different parts of the treap. Similar to our skip list approach, we cannot simply reuse deleted nodes without compromising security. While this design choice increases maintenance overhead compared to standard treaps, it represents an essential trade-off that ensures provable robustness against adversarial attacks while maintaining the treap's expected performance characteristics in adversarial environments.
%-------------------------------------------------------------------------------

%-------------------------------------------------------------------------------
\section{Experimental Results}\label{sec:experimental}
\begin{figure}[h]
    \centering
    \includegraphics[width=\textwidth]{chapters/ch5_skipping/ch5_images/datastructure_performance.png}
    \caption[Non-adaptive PSDS Results.]{Maximum and average hop count in the non-adaptive setting, displayed on a linear scale.}
    \label{fig:runtimes_nonadaptive}
\end{figure}

\begin{figure}[h]
    \centering
    \includegraphics[width=\textwidth]{chapters/ch5_skipping/ch5_images/adaptive_datastructure_performance.png}
    \caption[Adaptive PSDS Results.]{Maximum and average hop count in the adaptive setting, displayed on a logarithmic scale.}
    \label{fig:runtimes_adaptive}
\end{figure}


We conducted experiments to empirically validate our analytical results. Our first experiment tested whether robust data structures offer benefits in non-adversarial settings. Using a dataset of 10 million usernames~\cite{10-mio}, we randomly inserted 1,000 usernames into each data structure. We measured performance by counting hops (forward movements between nodes). The hash table's load factor was limited to 0.7 \cite{mcclellan1974art}, and the skip list's maximum height was set to $log_{2} n$. We used Python's built-in hash function, which is vulnerable to multi-collision attacks. Results were averaged over 100 trials.

As shown in \Cref{fig:runtimes_nonadaptive}, the robust skip list consistently required fewer mean and maximum hops than its standard counterpart, demonstrating benefits even in non-adversarial settings with only constant overhead. The robust hash table showed comparable performance to the original structure. We benchmarked an unmodified treap implementation given its inherent adversarial robustness.

Our second experiment evaluated performance under adaptive adversarial conditions. We implemented a hash collision attack on hash tables and a gap attack on skip lists, averaging results over 100 trials. Treaps were excluded due to their established inherent robustness.

For the hash collision attack, we pre-calculated bucket values to deliberately insert all elements into a single bucket, creating worst-case conditions for standard hash tables. For the robust implementation, we used a random key for pre-calculation, since the actual secret key would be unknown to an attacker.

For the gap attack against skip lists, we tested two variants: a restricted version using the same username dataset and an unrestricted version using integers within the range $[0, 10^{100}]$\footnote{While this serves primarily as a proof of concept, such a vast interval could realistically be achieved using a 20-character limit with Unicode encoding. We emphasize that significant runtime degradation can be observed even with substantially smaller intervals.}. We report the top $1\%$ of outcomes with respect to the maximum hop count.

Results in~\Cref{fig:runtimes_adaptive} confirm that adversarial attacks significantly degrade standard implementations, while robust counterparts maintain consistent performance. The robust skip list maintained an average maximum hop count of 27.36, compared to 33.17 for the non-robust implementation under non-adaptive conditions. Under adaptive settings, the non-robust implementation degraded to 35.61 maximum average hops, and further to 202.71 hops when using the larger integer range. This validation confirms our theoretical findings on adversarial robustness, and also suggests that our remark regarding the artificial ``looseness'' of the bound carries weight.
%-------------------------------------------------------------------------------
\chapter{Conclusion and Future Work}

\authorRemark{The following needs to be re-organized and re-written.}

In \cite{HassidimKMMS20}, the authors consider adding robustness to streaming algorithms using differential privacy.
Meanwhile, Hardt and Woodruff~\cite{hardt2013}, Cohen et al.~\cite{cohen2022robust} and Ben-Eliezer et al.\cite{BenEliezer2022} have shown that linear sketches (including CMS but not HK) are not ``robust" to well-resourced adaptive attacks, when it comes to various $L_p$-norm estimation tasks, e.g., solving the $k$-heavy-hitters problem relative to the $L_2$-norm.  
These works are mostly of theoretical importance, whereas we aim to give concrete  attacks and results that are (more) approachable for practitioners.

We made a comprehensive security analysis of the Redis CPDS suite, developing 10 different attacks across four CPDS. 
Our attacks can be used to cause severe disruptions to the performance of systems relying on these CPDS, ranging from mis-estimation of data statistics to triggering denial-of-service attacks. 
Our work illustrates the importance of low-level algorithmic choices and the dangers of using weak hash functions in CPDS. 

Our work opens up interesting directions for future work. Various other PDS suites exist in the wild, such as in Google BigQuery and Apache Spark, and could also be subjected to detailed security analysis as we have done for Redis here. 
Methods to provably protect PDS against attacks have been proposed in~\cite{NaorY15,clayton2019,FPUV22,PatersonR22}. However, these analyses tend to focus on textbook versions of the CPDS. 
Adapting these analyses to cater to the specifics of different implementations would help improve confidence in the deployed variants.

At a higher level, there still seems to be a lack of understanding  in the broader developer community about the risks of using CPDS in potentially adversarial settings. 
Work is needed to educate developers about these risks; we hope this paper can play a part in this effort. 
As an alternative, in an effort to shield developers from these risks, one could develop new CPDS implementations that are secure by default and package them in the form of easily consumed libraries with safe APIs. Such an effort could leverage the experience that the research community has gained from developing ``safe by default'' cryptographic libraries.

In this work, we conducted the first systematic analysis of probabilistic skipping-based data structures -- specifically hash tables, skip lists, and treaps -- in adaptive adversarial settings. Further, we established formal security notions and provide provably secure variants of each structure. Moreover, we uncovered innate vulnerabilities in the standard constructions of hash tables and skip lists that allowed adversaries to carry out attacks that degraded the expected performance of these structures exponentially.  Remarkably, we found that (insertion-only) treaps demonstrated inherent resilience against adaptive adversarial manipulation.

While our theoretical bounds provide rigorous security guarantees, there remains scope for developing tighter bounds. Further future research directions include extending our analysis to related data structures such as zip trees \cite{tarjan2021zip}, zip-zip trees \cite{gila2023zip}, skip graphs \cite{aspnes2007skip}, and randomized meldable heaps \cite{gambin1998randomized}. Moreover, a more sophisticated approach to deletions could potentially eliminate memory overhead while maintaining security guarantees. One promising direction involves localized re-initializations that would allow portions of the data structure to be safely rebuilt without compromising robustness. Additionally, it may be of interest to apply our AAPC framework to more classes of data structures and properties, as it is inherently quite extensible.

\end{document}

